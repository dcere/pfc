\chapter{Conclusiones}
\label{cap:conclusiones}


En este proyecto fin de carrera se ha creado un modelo de recursos distribuidos para facilitar la puesta en marcha y la automatización de distintas infraestructuras distribuidas. \\

Para comprobar la validez del modelo creado se ha extendido la herramienta de gestión de configuración Puppet añadiéndole el recurso sistema distribuido o \emph{cloud}. A la hora de definir el recurso sistema distribuido se han tenido en cuenta las características propias de este tipo de recursos, como la disponibilidad, las prestaciones y las dependencias y los conceptos de iteración y convergencia que las herramientas de configuración proveen. \\

Posteriormente se ha comprobado la extensión realizada haciendo que la herramienta de gestión de configuración Puppet sea capaz de administrar infraestructuras distribuidas de ejecución de trabajos como AppScale y TORQUE. Sin embargo no se han descuidado el resto de aspectos de estas infraestructuras. En concreto, en la infraestructura AppScale se soportan los dos tipos de despliegue y la totalidad de los roles posibles, siendo por lo tanto capaz de gestionar al completo una infraestructura de este tipo, y no sólo la parte de ejecución de trabajos. \\

Además, no se ha comprobado la validez del modelo exclusivamente con infraestructuras de ejecución de trabajos, sino que se ha constatado que también es válido para infraestructuras más generales como por ejemplo la de servicios web en tres niveles. Se puede decir, por consiguiente, que se ha ido más allá del mero cumplimiento de los objetivos del proyecto. \\

En cuanto al trabajo futuro, al ser este un primer prototipo, se encuentra abierto a multiples ampliaciones. El paso más evidente sería integrar este modelo dentro de la herramienta Puppet. Sería especialmente interesante modificar la gramática de descripción de recursos para diferenciar entre los recursos locales y los distribuidos. Posteriormente, se deberían incorporar las funcionalidades básicas de gestión de recursos distribuidos. De esta manera los futuros administradores de este tipo de infraestructuras contarían de manera automática con un conjunto de funciones que ayudan en la administración de sistemas distribuidos en Puppet.

La validez del trabajo realizado se ha comprobado únicamente sobre máquinas virtuales con un sistema operativo con núcleo Linux. Son de sobra conocidas las diferencias que existen entre las distintas distribuciones de Linux, por no hablar ya de las diferencias entre sistemas operativos completamente distintos. Realizar proveedores que sean válidos para cualquier sistema operativo es el segundo paso que debería realizarse. Además, estas máquinas virtuales se alojan en ordenadores de un laboratorio en la EINA; utilizar máquinas virtuales alojadas en proveedores como Amazon o Rackspace supondría un broche de oro. \\

Por último, y a modo de valoración personal, este proyecto me ha permitido aprender una gran cantidad de nuevas tecnologías; conocer una herramienta de gestión de configuración como Puppet, que es de las que más relevancia está tomando; estudiar las infraestructuras de AppScale, TORQUE y de servicios web e introducirme en un nuevo lenguaje de programación como Ruby que aporta conceptos muy interesantes y distintos a los lenguajes estudiados a lo largo de la carrera. Aunque en ocasiones la documentación fuese más escasa de lo deseado, la oportunidad de aprender nuevos conceptos y tecnologías es algo que personalmente valoro muy positivamente de este proyecto.
