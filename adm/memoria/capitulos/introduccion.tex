\chapter{Introducción}
\label{cap:introduccion}

{\sf

Durante los últimos años la computación en la nube ha ido ganando importancia de manera progresiva. La capacidad de usar la computación como un servicio permite a los usuarios finales de una aplicación acceder a ésta a través de un navegador web, una aplicación móvil o un cliente de escritorio mientras que la lógica de la aplicación y los datos se encuentran en servidores situados en una localización remota. Las aplicaciones en la nube tratan de proporcionar al usuario el mismo servicio y rendimiento que las aplicaciones instaladas localmente en su ordenador.

Explicación de ejemplos de infraestructuras distribuidas.
 - Clásicas: Condor, Torque
 - AppScale
 - Servicios web 3 capas

De la misma manera las herramientas de administración de sistemas también han experimentado un considerable avance. Con entornos cada vez más heterogéneos y complejos, la administración de sistemas de manera manual ya no es una opción.

Las herramientas de administración de sistemas carecen de funcionalidad para administrar infraestructuras distribuidas.

La administración de un cloud es la administración de las MV que forman los nodos del cloud. La administración es puramente software.

\section{Contexto del proyecto}

Para la realización de este proyecto de fin de carrera se ha hecho uso del laboratorio 1.03b de investigación que el Departamento de Informática e Ingeniería de Sistemas posee en la Escuela de Ingeniería y Arquitectura de la Universidad de Zaragoza.

\section{Objetivos}

El objetivo de este proyecto es proporcionar una herramienta que facilite la puesta en marcha de infraestructuras distribuidas y su posterior mantenimiento. Las tareas principales en las que se puede dividir este proyecto son:

\begin{enumerate}
\item Estudio de algunas de las infraestructuras distribuidas existentes profundizando en la parte relativa a la ejecución de trabajos distribuidos.
\item Análisis de las herramientas de administración de virtualización \emph{hardware}.
\item Investigación de las herramientas de gestión de configuración existentes más relevantes y elección de aquella que mayor facilidad de integración y uso proporcione.
\item Extensión de la herramienta de gestión de configuración para que soporte la puesta en marcha y el mantenimiento de un sistema de ejecución de trabajos distribuidos.
\end{enumerate}

\section{Organización de la memoria}

El resto de este documento queda organizado de la siguiente manera:
\begin{description}
\item[Capítulo \ref{cap:trabajo}] Trabajo realizado.
\item[Capítulo \ref{cap:conclusiones}] Conclusiones.
\end{description}


\section{Agradecimientos}

}
