\chapter{Introducción}
\label{cap:introduccion}

{\sf

Introduccion.

\section{Contexto del proyecto}

Para la realización de este proyecto de fin de carrera se ha hecho uso del laboratorio 1.03b de investigación que el Departamento de Informática e Ingeniería de Sistemas posee en la Escuela de Ingeniería y Arquitectura de la Universidad de Zaragoza.

\section{Objetivos}

El objetivo de este proyecto es proporcionar una herramienta que facilite la puesta en marcha de infraestructuras distribuidas y su posterior mantenimiento. Las tareas principales en las que se puede dividir este proyecto son:

\begin{enumerate}
\item Estudio de algunas de las infraestructuras distribuidas existentes profundizando en la parte relativa a la ejecución de trabajos distribuidos.
\item Análisis de las herramientas de administración de virtualización \emph{hardware}.
\item Investigación de las herramientas de gestión de configuración existentes más relevantes y elección de aquella que mayor facilidad de integración y uso proporcione.
\item Extensión de la herramienta de gestión de configuración para que soporte la puesta en marcha y el mantenimiento de un sistema de ejecución de trabajos distribuidos.
\end{enumerate}

\section{Organización de la memoria}

El resto de este documento queda organizado de la siguiente manera:
\begin{description}
\item[Capítulo \ref{cap:trabajo}] Trabajo realizado.
\item[Capítulo \ref{cap:conclusiones}] Conclusiones.
\end{description}


\section{Agradecimientos}

}
