\chapter{Validación de la solución planteada}
\label{cap:validacion}


Para validar la solución desarrollada, se ha hecho uso del laboratorio 1.03b que el Departamento de Informática e Ingeniería de Sistemas posee en la Escuela de Ingeniería y Arquitectura de la Universidad de Zaragoza. Los ordenadores de este laboratorio poseen procesadores con soporte para virtualización, lo que hace posible la creación de máquinas virtuales para simular los nodos que forman cada una de las infraestructuras distribuidas.

Antes de empezar con las pruebas hay que configurar el entorno en el que se realizarán. En particular, y dado que las máquinas virtuales necesitan conectarse a internet para la descarga e instalación de paquetes, el uso de un servidor de DNS es bastante recomendable. De este modo, podemos usar direcciones IP públicas (por ejemplo, las de las máquinas del laboratorio que no se estén usando en ese momento) para nuestras máquinas virtuales. El servidor DNS se usa también para hacer la resolución de nombres, tanto normal como inversa, que requieren AppScale y TORQUE para su correcto funcionamiento.


%%%%%%%%%%%%%%%%%%%%%%%%%%%%%%%%%%%%%%%%%%%%%%%%%%%%%%%%%%%%%%%%%%%%%%%%%%%%%%%%
\section{Pruebas comunes a todas las infraestructuras}

En todas y cada una de las infraestructuras se han realizado las siguientes pruebas para comprobar el correcto funcionamiento del proveedor distribuido:

\begin{itemize}
\item Apagado de una máquina virtual que había empezado encendida y no era líder: el líder provee una nueva máquina virtual que sustituye a la que ha fallado.
\item Apagado de una máquina virtual que no había empezado encendida y no era líder: el líder proporciona una nueva máquina virtual que sustituye a la que ha fallado.
\item Apagado de una máquina virtual que había empezado encendida y era líder: se elige a un nuevo líder el cual provee una nueva máquina virtual que sustituye a la que ha fallado.
\item Puesta en marcha de la infraestructura desde una máquina que no pertenece al \emph{cloud}: se informa de qué máquina perteneciente al \emph{cloud} se ha iniciado para continuar la puesta en marcha desde ella.
\item Puesta en marcha de la infraestructura desde una máquina que pertenece al \emph{cloud}: se crea el \emph{cloud}.
\end{itemize}

Estas pruebas tienen como objetivo simular los fallos que se podrían encontrar en una ejecución en un entorno de producción para demostrar la tolerancia a fallos del sistema.


%%%%%%%%%%%%%%%%%%%%%%%%%%%%%%%%%%%%%%%%%%%%%%%%%%%%%%%%%%%%%%%%%%%%%%%%%%%%%%%%
\section{Pruebas específicas de la infraestructura AppScale}

Para probar la infraestructura AppScale se han usado tres máquinas virtuales alojadas en dos máquinas físicas. Una de las máquinas virtuales actúa como nodo maestro y las otras dos actúan como nodos abiertos. Además de las pruebas comunes a todas las infraestructuras se han realizado una serie de pruebas particulares para comprobar el provedor de la infraestructura AppScale:

\begin{itemize}
\item Parada del proceso controlador: el proceso que lo está monitorizando (god \ref{anx:inst-god}) se da cuenta y crea un nuevo proceso controlador.
\item Parada del proceso que monitoriza al controlador: el proveedor (puppet) se da cuenta y crea un nuevo proceso de monitorización.
\end{itemize}


%%%%%%%%%%%%%%%%%%%%%%%%%%%%%%%%%%%%%%%%%%%%%%%%%%%%%%%%%%%%%%%%%%%%%%%%%%%%%%%%
\section{Pruebas específicas de la infraestructura TORQUE}

Para probar la infraestructura TORQUE se han usado cuatro máquinas virtuales alojadas en dos máquinas físicas. Una de las máquinas virtuales actúa como nodo maestro y las otras tres actúan como nodos de computación. Además de las pruebas comunes, para comprobar el provedor de la infraestructura TORQUE se han realizado las siguientes pruebas:

\begin{itemize}
\item Parada del proceso de autenticación (trqauthd) en el nodo maestro: el proceso que lo está monitorizando (god) se da cuenta y crea un nuevo proceso de autenticación.
\item Parada del proceso servidor (pbs\_server) en el nodo maestro: el proceso que lo está monitorizando (god) se da cuenta y crea un nuevo proceso servidor.
\item Parada del proceso planificador (pbs\_sched) en el nodo maestro: el proceso que lo está monitorizando (god) se da cuenta y crea un nuevo proceso planificador.
\item Parada del proceso de ejecución de trabajos (pbs\_mom) en un nodo de computación: el proceso que lo está monitorizando (god) se da cuenta y crea un nuevo proceso de ejecución de trabajos.
\item Parada del proceso que monitoriza al proceso de autenticación en el nodo maestro: el proveedor (puppet) se da cuenta y crea un nuevo proceso de monitorización.
\item Parada del proceso que monitoriza al proceso servidor en el nodo maestro: el proveedor (puppet) se da cuenta y crea un nuevo proceso de monitorización.
\item Parada del proceso que monitoriza al proceso planificador en el nodo maestro: el proveedor (puppet) se da cuenta y crea un nuevo proceso de monitorización.
\item Parada del proceso que monitoriza al proceso de ejecución de trabajos en un nodo de computación: el proveedor (puppet) se da cuenta y crea un nuevo proceso de monitorización.
\end{itemize}


%%%%%%%%%%%%%%%%%%%%%%%%%%%%%%%%%%%%%%%%%%%%%%%%%%%%%%%%%%%%%%%%%%%%%%%%%%%%%%%%
\section{Pruebas específicas de la infraestructura web de tres niveles}

Para probar la infraestructura web se han usado cuatro máquinas virtuales repartidas entre dos máquinas físicas. Una máquina virtual actúa como balanceador de carga, dos actúan como servidores web y la última lo hace como base de datos. Las pruebas que se han realizado para comprobar el correcto funcinamiento del proveedor de la infraestructura web han sido:

\begin{itemize}
\item Parada del proceso balanceador de carga: el proceso que lo está monitorizando (puppet) se da cuenta y crea un nuevo proceso balanceador de carga.
\item Parada del proceso servidor web: el proceso que lo está monitorizando (puppet) se da cuenta y crea un nuevo proceso servidor web.
\item Parada del proceso base de datos: el proceso que lo está monitorizando (god) se da cuenta y crea un nuevo proceso base de datos.
\item Parada del proceso que monitoriza al proceso balanceador de carga: el proveedor (puppet) se da cuenta y crea un nuevo proceso de monitorización.
\item Parada del proceso que monitoriza al proceso servidor web: el proveedor (puppet) se da cuenta y crea un nuevo proceso de monitorización.
\item Parada del proceso que monitoriza al base de datos: el proveedor (puppet) se da cuenta y crea un nuevo proceso de monitorización.
\end{itemize}
