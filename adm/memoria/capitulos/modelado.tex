\chapter{Modelado de configuración automática de infraestructuras distribuidas}
\label{cap:modelado}

[Revisar]\\

La extensión a la herramienta de configuración Puppet se ha hecho mediante el uso de tipos y proveedores personalizados. Mediante la definición de un nuevo tipo estamos ampliando la cantidad de recursos que Puppet puede administrar; pero para que Puppet sepa cómo administrar ese nuevo recurso debemos proporcionar un proveedor en el que se le indique lo que tiene que hacer.


\section{Modelado de recursos y configuración en Puppet}

Puppet usa un lenguaje declarativo para modelar los distintos elementos de configuración, que en la terminología de Puppet se llaman recursos. Mediante el uso de este lenguaje se indica en qué estado se quiere que se mantenga el recurso y será tarea de Puppet el encargarse de que así sea. Cada recurso está compuesto de un tipo (el tipo de recurso que estamos gestionando), un título (el nombre del recurso) y una serie de atributos (los valores que especifican el estado del recurso). Por ejemplo, para modelar un recurso de tipo fichero podríamos usar algo similar a:

\begin{lstlisting}
file {'testfile':
  path    => '/tmp/testfile',
  ensure  => present,
  mode    => 0640,
  content => "I'm a test file.",
}
\end{lstlisting}

La agrupación de uno o más recursos en un fichero de texto da lugar a un manifiesto. En general, un manifiesto contiene la información necesaria para realizar la configuración de un nodo. Cuando a Puppet se le da la orden de aplicar un manifiesto los pasos que hace son:
\begin{itemize}
\item Interpretar y compilar la configuración.
\item Comunicar la configuración compilada al nodo.
\item Aplicar la configuración en el nodo.
\item Enviar un informe con los resultados.
\end{itemize}


\section{Modelado de recursos distribuidos: el recurso \emph{cloud}}

El modelado de un recurso distribuido plantea ciertos desafíos al modelo anterior: se puede pensar que para modelar un recurso distribuido basta con que Puppet envíe a cada nodo la configuración necesaria para garantizar el comportamiento deseado, pero, ¿qué pasa cuando ese nodo falla? Si no hacemos nada el recurso dejará de mantenerse en el estado deseado. A la hora de administrar un recurso distribuido hay que asegurarse por lo tanto de que los nodos están operativos y cumpliendo con su función.
Asimismo, un recurso distribuido puede presentar elementos comunes para todos los nodos del sistema. Un recurso clásico de Puppet no presenta este problema, ya que o el nodo posee ese recurso, o no lo posee. Se puede hacer una analogía con la programación orientada a objetos en la que el recurso distribuido sería una metaclase...

Para modelar un recurso \emph{cloud}, se han considerado como indispensables los siguientes parámetros:
\begin{itemize}
\item Nombre: Para identificar al \emph{cloud} de manera única.
\item Tipo: Para describir qué tipo de \emph{cloud} es. [Modificar]
\item Fichero de direcciones IP: Para describir qué dirección IP está asociada a cada nodo del \emph{cloud}.
\item Fichero de imágenes de disco: Para asignar a cada nodo la correspondiente imagen de disco duro.
\item Fichero de dominio: Para definir una máquina virtual especificando sus características \emph{hardware}.
\item Conjunto de máquinas físicas: Para indicar qué máquinas físicas pueden ejecutar las máquinas virtuales definidas.
\end{itemize}


\section{Diseño del proveedor}

Una vez modelado el recurso \emph{cloud}, queda como tarea proporcionar un proveedor que se encargue de llevar el \emph{cloud} al estado que se le indique desde el manifiesto Puppet. Un \emph{cloud} podrá estar en dos estados: funcionando o parado. Si decidimos que el \emph{cloud} tiene que estar funcionando, el proveedor se encargará de todos los pasos necesarios para llevar al \emph{cloud} a ese estado. A grandes rasgos, estos son los pasos que realiza para ponerlo en marcha:
\begin{itemize}
\item Comprobación de la existencia del \emph{cloud}: si existe se realizarán tareas de mantenimiento, si no existe se creará.
\item Comprobación del estado del conjunto de máquinas físicas: si una máquina física debe ejecutar muchas máquinas virtuales, existe un riesgo de pérdida de rendimiento.
\item Obtención de las direcciones IP de los nodos y los roles que les han sido asignados.
\item Comprobación del estado de las máquinas virtuales: si están funcionando se monitorizan, mientras que si no están funcionando hay que definir una nueva máquina virtual y ponerla en funcionamiento.
\item Cuando todas las máquinas virtuales estén funcionando se procede a inicializar el \emph{cloud}.
\item Operaciones de puesta en marcha particulares a cada tipo de \emph{cloud}.
\end{itemize}

De la misma manera, si decidimos que el \emph{cloud} tiene que estar parado, el proveedor se encargará de realizar los pasos necesarios para ello. Estos pasos son:
\begin{itemize}
\item Comprobación de la existencia del \emph{cloud}: si existe se procederá a su parada.
\item Operaciones de parada particulares a cada tipo de \emph{cloud}.
\item Apagado y borrado de las definiciones de las máquinas virtuales creadas explícitamente para este \emph{cloud}.
\item Parada de las funciones de automantenimiento de los nodos.
\item Eliminación de los ficheros internos de gestión del \emph{cloud}.
\end{itemize}
