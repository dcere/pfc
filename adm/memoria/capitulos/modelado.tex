\chapter{Modelado de recursos distribuidos con Puppet}
\label{cap:modelado}

[Revisar]\\

Las herramientas de gestión de la configuración se han centrado en la gestión de recursos de manera local a un nodo. Por otra parte la automatización existente en la administración de infraestructuras distribuidas es de bajo nivel, no yendo mucho más allá de la gestión de máquinas virtuales. En este capítulo veremos cómo se pueden crear recursos distribuidos en la herramienta de gestión de la configuración Puppet y cómo ésta puede ser usada para automatizar una administración de más alto nivel en infraestructuras distribuidas que tenga en cuenta conceptos como el de disponibilidad o el de prestaciones.


\section{Configuración de recursos distribuidos}

En Puppet, la definición clásica de recursos se presupone dentro del ámbito local del nodo. Es decir, para cada nodo especificamos qué recursos debe contener y cuál debe ser su estado. Dentro de este tipo de recursos se encuentran, entre otros, el recurso usuario y el recurso fichero. Sin embargo, el modelado de un recurso sistema distribuido plantea ciertos desafíos al ejemplo anterior, ya que no está pensado teniendo en cuenta la problemática asociada a los sistemas distribuidos. \\

%Ahora bien, este proveedor no es un proveedor al uso: debe lidiar con la problemática asociada a los sistemas distribuidos. Entre otras cosas debe tener en cuenta que puede haber dependencias entre los nodos, que cada nodo puede cumplir un papel distinto dentro del sistema y que los nodos pueden fallar. Además, debe resolver los fallos con la mayor transparencia posible, es decir, con la menor intervención humana posible.\\

Al modelar un recurso sistema distribuido, deben tenerse en cuenta las características propias de este tipo de recursos, como la disponibilidad, las prestaciones y las dependencias. La disponibilidad contempla los fallos que se pueden dar en una infraestructura distribuida y en ella estarían incluidos los fallos de procesos y los fallos de máquinas. Las prestaciones contemplan los servicios ofrecidos y dentro de ellas tendríamos la creación de máquinas para repartir la carga. Las dependencias contemplan la necesidad de que un servicio esté funcionando para que otro pueda hacerlo. Asimismo, un recurso sistema distribuido puede presentar elementos comunes con otros recursos sistema distribuido, tales como una monitorización básica. La presencia de elementos comunes entre los recursos clásicos de Puppet, por ejemplo un fichero y un usuario, no es tan corriente. \\

A la hora de definir un recurso sistema distribuido tenemos que presentarlo como un único sistema coherente, es decir, como una única abstracción, y por lo tanto no vale con describir un recurso sistema distribuido como una colección de recursos clásicos de Puppet. Afortunadamente, Puppet proporciona los medios para crear nuevos tipos de recurso, y se puede crear un nuevo tipo de recurso con sus atributos correspondientes para definir los recursos sistema distribuido. \\


%%%%%%%%%%%%%%%%%%%%%%%%%%%%%%%%%%%%%%%%%%%%%%%%%%%%%%%%%%%%%%%%%%%%%%%%%%%%%%%%
\section{Modelización en Puppet}
\label{sec:modelado-puppet}

Puppet puede ser extendido para incluir la definición de nuevos recursos. Para ello hay que proporcinarle como mínimo dos ficheros: uno en el que se define el recurso y otro en el que se define cómo gestionar ese recurso. Al fichero en el que se define el recurso se le llama tipo y al fichero en el que se define cómo gestionarlo se le llama proveedor. Es decir, el tipo se encarga del ``qué'' y el proveedor se encarga del ``cómo''.\\

Para definir un recurso sistema distribuido, al que también llamaremos \emph{cloud}, se han considerado como fundamentales los siguientes atributos:

\begin{itemize}
\item Nombre: Para identificar al recurso de manera única.
\item Fichero de dominio: Para definir una plantilla de creación de máquinas virtuales especificando sus características \emph{hardware}.
\item Conjunto de máquinas físicas: Para indicar qué máquinas físicas pueden ejecutar las máquinas virtuales definidas.
\end{itemize}

Estos atributos se obtienen mediante la observación de los elementos comunes a todo recurso sistema distribuido y forman, por tanto, el núcleo de un recurso \emph{cloud} genérico. Además de estos atributos cada subtipo de recurso sistema distribuido (AppScale, TORQUE...) puede añadir los que considere necesarios para una completa especificación del recurso.


%%%%%%%%%%%%%%%%%%%%%%%%%%%%%%%%%%%%%%%%%%%%%%%%%%%%%%%%%%%%%%%%%%%%%%%%%%%%%%%%
\subsection{Patrón de diseño del proveedor}
\label{sec:modelado-proveedor}

En Puppet, el proveedor es el encargado de llevar al recurso al estado que se le indique en el manifiesto. Típicamente el proveedor posee las funciones necesarias para crear un nuevo recurso y para destruirlo. Para llevar a cabo estas funciones en un recurso de tipo \emph{cloud} el proveedor se apoya en tres grupos de funciones: puesta en marcha de un \emph{cloud}, monitorización de un \emph{cloud} y parada de un \emph{cloud}. Las funciones de los dos primeros grupos se usan a la hora de crear un nuevo recurso de tipo \emph{cloud} mientras que las del último grupo se usan a la hora de parar un \emph{cloud} ya existente. \\

Puppet no contempla la herencia entre tipos o entre proveedores de distintos tipos. Por lo tanto, las funciones anteriormente especificadas no se encuentran dentro de ningún proveedor concreto (no existe, como tal, el proveedor de un recurso \emph{cloud} genérico), sino que se presentan como una clase Ruby con unas funcionalidades básicas que los distintos proveedores de recursos sistema distribuido pueden usar para facilitar el desarrollo de los mismos. \\

Las operaciones de puesta en marcha son las encargadas de poner en funcionamiento el \emph{cloud} especificado en el manifiesto. Las más importantes son:
\begin{itemize}
\item Inicio como líder: Función de puesta en marcha que realizará el nodo líder del \emph{cloud}. Ésta es la función más importante dentro de las funciones de inicio del proveedor ya que es la que se encarga de iniciar el \emph{cloud}. A grandes rasgos, los pasos que realiza son:

   \begin{enumerate}
   \item Comprobación de la existencia del \emph{cloud}: si no existe se creará.
   \item Comprobación del estado del conjunto de máquinas físicas.
   \item Obtención de las direcciones IP de los nodos y los roles que les han sido asignados.
   \item Comprobación del estado de las máquinas virtuales: si están funcionando se monitorizan, mientras que si no están funcionando hay que definir una nueva máquina virtual y ponerla en funcionamiento. Las funciones de monitorización incluyen el envío de un fichero mediante el cual cada nodo se autoadministre la mayor parte posible.
   \item Cuando todas las máquinas virtuales estén funcionando se procede a inicializar el \emph{cloud}.
   \item Operaciones de puesta en marcha particulares dependiendo de cada tipo de \emph{cloud}.
   \end{enumerate}

\item Inicio como nodo común: Función de puesta en marcha que realizarán los nodos comunes del \emph{cloud}.
\item Inicio como nodo externo: Función de puesta en marcha que realizará un nodo no perteneciente al \emph{cloud}.
\end{itemize}

La monitorización del \emph{cloud} únicamente la llevará a cabo uno de los nodos, al que llamaremos líder, ya que sería redundate que más de un nodo se encargara de comprobar el estado global del \emph{cloud}. Por tanto, sólo hay una función importante:
\begin{itemize}
\item Monitorización como líder: Función de monitorización que realizará el nodo líder del \emph{cloud}.
\end{itemize}

Para elegir al nodo líder de entre todos los nodos del \emph{cloud} se utiliza el algoritmo peleón (en inglés \emph{Bully algorithm}). En este algoritmo todos los nodos tratan periódicamente de convertirse en el líder; si hay un líder, impedirá que otro nodo le quite el liderazgo, y si no lo hay, uno de los restantes nodos se convertirá en líder. Para ayudar a la implementación de este algoritmo se proporcionan las funciones de elección de líder, de las cuales las más importantes son:
\begin{itemize}
\item Lectura y escritura de identificador: Funciones de lectura y escritura del identificador del nodo actual.
\item Lectura y escritura de identificador de líder: Funciones de lectura y escritura del identificador del nodo líder.
\item Escritura de identificador e identificador de líder remoto: Funciones de escritura del identificador y del identificador del líder en un nodo distinto del actual.
\end{itemize}

Por último, es posible que en algún momento se desee parar por completo el funcionamiento del \emph{cloud}. Las operaciones de parada de \emph{cloud} más importantes son:
\begin{itemize}
\item Apagado de máquinas virtuales: Función de apagado de las máquinas virtuales que forman el \emph{cloud}.
\item Borrado de ficheros: Función de eliminación de todos los ficheros internos de gestión del \emph{cloud}.
\end{itemize}

Aunque no se proporcionan como funciones, hay que tener en cuenta que cada tipo de \emph{cloud} puede tener sus propias funciones de parada. Estas funciones de parada deben realizarse antes de apagar las máquinas virtuales. De forma general, los pasos que hay que hacer a la hora de parar un \emph{cloud} son:
[Revisar este orden]
\begin{enumerate}
\item Comprobación de la existencia del \emph{cloud}: si existe se procederá a su parada.
\item Operaciones de parada particulares a cada tipo de \emph{cloud}.
\item Apagado de las máquinas virtuales creadas explícitamente para este \emph{cloud}.
\item Parada de las funciones de automantenimiento de los nodos.
\item Eliminación de los ficheros internos de gestión del \emph{cloud}.
\end{enumerate}


%%%%%%%%%%%%%%%%%%%%%%%%%%%%%%%%%%%%%%%%%%%%%%%%%%%%%%%%%%%%%%%%%%%%%%%%%%%%%%%%
\subsection{\emph{Framework} de implementación}
\label{sec:modelado-framework}

%Las operaciones de puesta en marcha que se proporcionan son:
%\begin{itemize}
%\item Inicio como líder: \texttt{leader\_start}.
%\item Inicio como nodo común: \texttt{common\_start}.
%\item Inicio como nodo externo: \texttt{not\_cloud\_start}.
%\end{itemize}

%La función de monitorización que se proporciona es:
%\begin{itemize}
%\item Monitorización como líder: \texttt{leader\_monitoring}.
%\end{itemize}

%Las funciones de elección de líder que se proporcionan son:
%\begin{itemize}
%\item Lectura y escritura de identificador: \texttt{get\_id} y \texttt{set\_id}.
%\item Lectura y escritura de identificador de líder: \texttt{get\_leader} y \texttt{set\_leader}.
%\item Escritura de identificador e identificador de líder remoto: \texttt{vm\_set\_id} y \texttt{vm\_set\_leader}.
%\end{itemize}

%Las funciones de parada de \emph{cloud} que se proporcionan son:
%\begin{itemize}
%\item Apagado de máquinas virtuales: \texttt{shutdown\_vms}.
%\item Borrado de ficheros: \texttt{delete\_files}.
%\end{itemize}

La mejor manera de crear un tipo y un proveedor en Puppet es hacerlo usando el concepto de módulo que ya dispone Puppet. Los módulos deben organizarse siguiendo una estructura predefinida. En esta estructura hay que colocar el código del tipo y del proveedor en la parte correspondiente. Si se crea el tipo \texttt{tipo} y el proveedor \texttt{proveedor} dentro del módulo \texttt{modulo} la estructura sería similar a la siguiente:

\begin{lstlisting}
modulo/
modulo/manifests
modulo/manifests/init.pp
modulo/files
modulo/templates
modulo/lib/facter
modulo/lib/puppet/type
modulo/lib/puppet/type/tipo.rb
modulo/lib/puppet/provider
modulo/lib/puppet/provider/tipo/proveedor.rb
modulo/lib/puppet/parser/functions
\end{lstlisting}

Al ser el recurso sistema distribuido un recurso genérico, dicha organización es excesiva, puesto que carece de la mayoría de los elementos ahí definidos. Bastará por tanto con la siguiente estructura:

\begin{lstlisting}
generic-cloud/provider
generic-cloud/provider/cloud.rb
\end{lstlisting}

Dentro del fichero \texttt{cloud.rb} se encuentra definida la clase \texttt{Cloud} que proporciona las funciones más básicas que ayudan al resto de recursos sistema distribuido a crear sus proveedores. En concreto, las operaciones de puesta en marcha que se proporcionan son \texttt{leader\_start}, \texttt{common\_start} y \texttt{not\_cloud\_start}; la función de monitorización que se proporciona es \texttt{leader\_monitoring}; las funciones de elección de líder que se proporcionan son \texttt{get\_id}, \texttt{set\_id}, \texttt{get\_leader}, \texttt{set\_leader}, \texttt{vm\_set\_id} y \texttt{vm\_set\_leader}; por último, las funciones de parada de \emph{cloud} que se proporcionan son \texttt{shutdown\_vms} y \texttt{delete\_files}. La implementación completa de la clase \texttt{Cloud} puede verse en el anexo X.

Además de proporcionar estas funciones, la clase \texttt{Cloud} necesita de otras para realizar su trabajo. En particular, cada proveedor debe implementar dos funciones. La primera de ellas es la función \texttt{start\_cloud} que se usará para iniciar el \emph{cloud}. La segunda es la función \texttt{obtain\_vm\_data} que se utilizará para obtener datos de las máquinas virtuales que forman el \emph{cloud}.

Para monitorizar el \emph{cloud} concreto, también será necesario pasarle una función como argumento a las funciones \texttt{leader\_start} y \texttt{leader\_monitoring}. Esta función implementa una monitorización que incluye los aspectos particulares de cada tipo de \emph{cloud}.
