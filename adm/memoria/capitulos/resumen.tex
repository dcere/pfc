\chapter{Resumen ejecutivo}
\label{cap:resumen}

{\sf

A la hora de ejecutar trabajos en un entorno distribuido la aproximación clásica ha sido bien el uso de un cluster de ordenadores o bien el uso de la computación en malla o grid. Con la proliferación de entornos cloud durante estos últimos años y su facilidad de uso, parece que una nueva opción se abre para la ejecución de este tipo de trabajos.\\

La computación en cloud presta al usuario capacidad de cálculo, aplicaciones software, acceso a los datos, gestión de los datos y almacenamiento de recursos sin requerir del usuario un conocimiento de la localización y otro tipo de detalles de la infraestructura del cloud. Asimismo, frente a las dos aproximaciones clásicas, la computación en cloud ofrece la capacidad de añadir dinámicamente tantos nodos en los que ejecutar trabajos como el usuario quiera. Esta capacidad de aprovisionamiento de nodos de ejecución de manera tan dinámica supone un reto a la hora de administrar el cloud. Además de la administración, la puesta en marcha de una infraestructura de tipo cloud es también un proceso laborioso.\\

Por ello, en este proyecto hemos diseñado una solución capaz de automatizar tanto la puesta en marcha de un sistema de ejecución de trabajos distribuidos como la administración y mantenimiento del mismo. Para ello se han estudiado entornos clásicos de ejecución de trabajos como Condor y Torque, entornos de ejecución de trabajos en cloud como AppScale y herramientas clásicas de administración de sistemas como Puppet. Extendiendo la funcionalidad de Puppet hemos desarrollado un sistema capaz de poner en marcha de forma automática un entorno de ejecución de trabajos y encargarse de su posterior mantenimiento. Para verificar la solución diseñada se ha hecho uso del laboratorio 1.03b que el Departamento de Informática e Ingeniería de Sistemas posee. Dicho laboratorio ofrece un entorno el el que simular, a pequeña escala, funcionalidades similares a las que obtendríamos en un cloud a gran escala.\\

Como resultado de este proyecto se presenta una solución que simplifica de manera notable la puesta en marcha de un sistema de ejecución de trabajos distribuidos así como la administración del mismo.
}
