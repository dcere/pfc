\chapter{Resumen}
\label{cap:resumen}


A la hora de ejecutar trabajos en un entorno distribuido, la aproximación clásica ha sido bien el uso de un \emph{cluster} de ordenadores o bien el uso de la computación en malla o \emph{grid}. Con la proliferación de entornos \emph{cloud} durante estos últimos años y su facilidad de uso, una nueva opción se abre para la ejecución de este tipo de trabajos.\\

De hecho, la ejecución de trabajos distribuidos es uno de los principales usos dentro del ámbito de los sistemas \emph{cloud}. Sin embargo, la administración de este tipo de sistemas dista de ser sencilla: cuestiones como la puesta en marcha del sistema, el aprovisionamiento de nodos, las modificaciones del sistema y la evolución y actualización del mismo suponen una tarea intensa y pesada.\\

En vista de lo cual, en este proyecto se ha diseñado una solución capaz de automatizar la administración de sistemas \emph{cloud}, y en particular de un sistema de ejecución de trabajos distribuidos. Para ello se han estudiado entornos clásicos de ejecución de trabajos como Torque y entornos de ejecución de trabajos en \emph{cloud} como AppScale. Además, se han estudiado herramientas clásicas de configuración automática de sistemas como Puppet y CFEngine. El objetivo principal de estas herramientas de configuración de sistemas es la gestión del nodo. En este proyecto se ha extendido la funcionalidad de una de estas herramientas -- Puppet -- añadiéndole la capacidad de gestión de sistemas \emph{cloud}.\\

Como resultado de este proyecto se presenta una solución capaz de administrar de forma automática sistemas de ejecución de trabajos distribuidos. La validación de esta solución se ha llevado a cabo sobre los entornos de ejecución de trabajos Torque y AppScale y también, para mostrar su carácter genérico, sobre una arquitectura de servicios web de tres niveles.
