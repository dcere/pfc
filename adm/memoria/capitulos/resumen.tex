\chapter{Resumen}
\label{cap:resumen}


A la hora de ejecutar trabajos en un entorno distribuido la aproximación clásica ha sido bien el uso de un \emph{cluster} de ordenadores o bien el uso de la computación en malla o \emph{grid}. Con la proliferación de entornos \emph{cloud} durante estos últimos años y su facilidad de uso, parece que una nueva opción se abre para la ejecución de este tipo de trabajos.\\

La computación en \emph{cloud} proporciona capacidad de cálculo, aplicaciones \emph{software} y acceso, gestión y almacenamiento de datos sin requerir del usuario un conocimiento de los detalles internos de la infraestructura del \emph{cloud}. Asimismo, frente a las dos aproximaciones clásicas de ejecución de trabajos, la computación en \emph{cloud} ofrece la capacidad de añadir nodos para la ejecución de trabajos de forma más dinámica. Esta capacidad de aprovisionamiento de nodos de ejecución de manera tan dinámica supone todo un reto a la hora de administrar el \emph{cloud}. Además de la administración, la puesta en marcha de una infraestructura de tipo \emph{cloud} es también un proceso laborioso.\\

En vista de lo cual, en este proyecto se ha diseñado una solución capaz de automatizar tanto la puesta en marcha como la administración y el mantenimiento de un sistema de ejecución de trabajos distribuidos. Para ello se han estudiado entornos clásicos de ejecución de trabajos como Condor y Torque, entornos de ejecución de trabajos en \emph{cloud} como AppScale y herramientas clásicas de administración de sistemas como Puppet y CFEngine. El objetivo principal de estas herramientas de adminisitración de sistemas es la gestión del nodo. En este proyecto se ha extendido la funcionalidad de una de estas herramientas -- Puppet -- añadiéndole la capacidad de gestión del \emph{cloud}. Para verificar la solución diseñada se ha hecho uso del laboratorio 1.03b del Departamento de Informática e Ingeniería de Sistemas. Dicho laboratorio ofrece un entorno el el que simular, a pequeña escala, funcionalidades similares a las que obtendríamos en un \emph{cloud} a gran escala.\\

Como resultado de este proyecto se presenta una extensión a la herramienta de configuración Puppet como solución para simplificar de manera notable la puesta en marcha de un sistema de ejecución de trabajos distribuidos así como la administración del mismo.
