\chapter{Arquitectura Web}
\label{web:web}


%%%%%%%%%%%%%%%%%%%%%%%%%%%%%%%%%%%%%%%%%%%%%%%%%%%%%%%%%%%%%%%%%%%%%%%%%%%%%%%%
\section{Balanceador de carga}


Usaremos nginx como balanceador de carga, y no como servidor web, que es la manera más habitual de verlo en funcionamiento.


%%
\subsection{Instalación}

Para instalar nginx, haz lo siguiente:

\begin{bashcode}
_: apt-get update
_: apt-get upgrade
_: apt-get install nginx
\end{bashcode}
\\


%%
\subsection{Configuración}

Para configurar nginx debemos añadir la parte de balanceo de carga al fichero de configuración \texttt{/etc/nginx/nginx.conf}. Como este fichero no es excesivamente grande, se muestra en su totalidad con la parte modificada resaltada:

\begin{bashcode}
user www-data;
worker_processes  1;

error_log  /var/log/nginx/error.log;
pid        /var/run/nginx.pid;

events {
    worker_connections  1024;
    # multi_accept on;
}

http {
    include       /etc/nginx/mime.types;

    access_log	/var/log/nginx/access.log;

    sendfile        on;
    #tcp_nopush     on;

    #keepalive_timeout  0;
    keepalive_timeout  65;
    tcp_nodelay        on;

    gzip  on;
    gzip_disable "MSIE [1-6]\.(?!.*SV1)";

    include /etc/nginx/conf.d/*.conf;
    include /etc/nginx/sites-enabled/*;


    ### Modified
    upstream web_servers {
      server 155.210.155.73:4567;
      server 155.210.155.178:4567;
    }

    server {
      listen 155.210.155.175:80;
      location / {
        proxy_pass http://web_servers;
      }
    }
    ### End of modification


}
\end{bashcode}
\\


%%
\subsection{Ejecución}

Para iniciar nginx haremos uso del script localizado en \texttt{/etc/init.d}. Dicho script puede ser usado tanto para iniciarlo:

\begin{bashcode}
_: /etc/init.d/nginx start
\end{bashcode}
\\

como para pararlo:

\begin{bashcode}
_: /etc/init.d/nginx stop
\end{bashcode}
\\


%%
\subsection{Versiones}

\begin{tabular}{|c|c|}
   \hline
   Software & Versión \\ \hline
   nginx & 0.7.65 \\ \hline
\end{tabular}


%%%%%%%%%%%%%%%%%%%%%%%%%%%%%%%%%%%%%%%%%%%%%%%%%%%%%%%%%%%%%%%%%%%%%%%%%%%%%%%%


\section{Servidor web}


Como servidor web usaremos WEBrick, que es el servidor web que viene por defecto con la instalación de Ruby. Instalando Ruby instalaremos a la vez el servidor web.


%%
\subsection{Instalación}

Lo primero que hay que hacer es instalar Ruby y rubygems. Para ello, consulta el apéndice \ref{comun:ruby} de instalación de Ruby.

Una vez instalado Ruby y rubygems, instalaremos ruby-dev:

\begin{bashcode}
_: apt-get install ruby1.8-dev
\end{bashcode}
\\

Y ahora instalaremos el soporte necesario para interactuar con la base de datos:

\begin{bashcode}
_: apt-get install libmysqlclient-dev
_: gem install mysql
\end{bashcode}
\\

La aplicación web será desarrollada usando Sinatra. Antes de comenzar con la instalación, comprueba que tu versión de rubygems es igual o superior a la 1.3.6. Esto puede hacerse fácilmente de la siguiente manera:

\begin{bashcode}
_: gem --version
1.3.6
\end{bashcode}
\\

Para instalar Sinatra, haremos lo siguiente:

\begin{bashcode}
_: gem install sinatra
Successfully installed rack-1.4.1
Successfully installed rack-protection-1.2.0
Successfully installed tilt-1.3.3
Successfully installed sinatra-1.3.2
4 gems installed
Installing ri documentation for rack-1.4.1...
...
Installing ri documentation for sinatra-1.3.2...
Installing RDoc documentation for rack-1.4.1...
...
Installing RDoc documentation for sinatra-1.3.2...
\end{bashcode}
\\

Nota: Puede llevar un tiempo hasta que el proceso de instalación muestre cosas por pantalla.


%%
\subsection{Ejecución}

Una vez que la instalación ha finalizado, vamos a crear la primera aplicación web. Guarda el siguiente fichero como \texttt{web.rb}:

\begin{rubycode}
require 'rubygems'
require 'sinatra'

get '/' do
  'Hello world!'
end
\end{rubycode}

y lanza el servidor web:

\begin{bashcode}
_: ruby web.rb
\end{bashcode}
\\

Nota: Para salir Ctrl + C.

En tu navegador web preferido ve a la dirección \texttt{http://localhost:4567} y encontrarás la aplicación web que acabamos de crear.


%%
\subsection{Añadiendo soporte para la base de datos}

Para interactuar con la base de datos usaremos ActiveRecord. Este componente es parte de Ruby On Rails, pero también existe como una \emph{gem} independiente. Vamos a instalarla:

\begin{bashcode}
_: gem install activerecord
\end{bashcode}
\\

Ahora vamos a crear una segunda aplicación web. Guarda el siguiente fichero como \texttt{web2.rb}:

\begin{rubycode}
require 'rubygems'
require 'sinatra'
require 'active_record'

class Article < ActiveRecord::Base
end

get '/' do
   'Hello there!'
end
\end{rubycode}

y lanza el servidor web como antes:

\begin{bashcode}
_: ruby web2.rb
\end{bashcode}
\\

En tu navegador web ve a la dirección \texttt{http://localhost:4567} y encontrarás la aplicación web. Muestra lo mismo que la primera aplicación web, pero hemos incluido (aunque no usado) el soporte para la base de datos.


%%
\subsection{Versiones}

\begin{tabular}{|c|c|}
   \hline
   Software & Versión \\ \hline
   Ruby & 1.8.7 \\ \hline
   rubygems & 1.8.21 \\ \hline
   Sinatra & 1.3.2 \\ \hline
   ActiveRecord & 3.2.3 \\ \hline
\end{tabular}


%%%%%%%%%%%%%%%%%%%%%%%%%%%%%%%%%%%%%%%%%%%%%%%%%%%%%%%%%%%%%%%%%%%%%%%%%%%%%%%%


\section{Base de datos}


Como servidor de base de datos usaremos MySQL.


%%
\subsection{Instalación}

Para instalar MySQL, haz lo siguiente:

\begin{bashcode}
_: apt-get install mysql-server
\end{bashcode}
\\


%%
\subsection{Configuración}

Para configurar los ajustes básicos hay que editar el fichero \texttt{/etc/mysql/my.cnf}. Por ejemplo, si vamos a aceptar conexiones de otra máquina, hay que modificar el parámetro \texttt{bind\_address}. En nuestro caso, lo modificaremos para aceptar conexiones del servidor web:

\begin{bashcode}
bind_address = 155.210.155.73
\end{bashcode}
\\

Nota: Si estás usando Ubuntu 10.04 debido a un \emph{bug} \href{http://ubuntuforums.org/showthread.php?t=1479310}{es mejor que comentes toda la línea}. Además nosotros usaremos dos servidores web, así que mejor la comentamos:

\begin{bashcode}
#bind_address = 155.210.155.73
\end{bashcode}
\\

Vamos a reiniciar el servidor para que los cambios surtan efecto:

\begin{bashcode}
_: /usr/bin/service mysql restart
\end{bashcode}
\\

Para comprobar que ha sido instalado correctamente, podemos hacer lo siguiente:

\begin{bashcode}
_: mysql -u root -p     # Introduce MySQL's root password
Welcome to the MySQL monitor.  Commands end with ; or \g.
Your MySQL connection id is 34
Server version: 5.1.61-0ubuntu0.10.04.1 (Ubuntu)
...
Type 'help;' or '\h' for help. Type '\c' to clear the current input statement.

mysql> CREATE DATABASE mydb;
\end{bashcode}
\\


%%
\subsection{Ejecución}

Para ejecutar el servidor de bases de datos usaremos el programa \texttt{service} localizado en \texttt{/usr/bin/service}. Sirve tanto como para iniciarlo:

\begin{bashcode}
_: /usr/bin/service mysql start
\end{bashcode}
\\

como para pararlo:
\begin{bashcode}
_: /usr/bin/service mysql stop
\end{bashcode}
\\


%%
\subsection{Versiones}

\begin{tabular}{|c|c|}
   \hline
   Software & Versión \\ \hline
   mysql & 5.1.61 \\ \hline
\end{tabular}
