\chapter{Servidor web}
\label{web:servidor}

Como servidor web usaremos WEBrick, que es el servidor web que viene por defecto con la instalación de Ruby. Instalando Ruby instalaremos a la vez el servidor web.


%%
\section{Instalación}

Lo primero que hay que hacer es instalar Ruby y RubyGems. Para ello, consulta el apéndice \ref{comun:ruby} de instalación de Ruby.

Una vez instalado Ruby y RubyGems, instalaremos el paquete \texttt{ruby-dev}:

\begin{bashcode}
_: apt-get install ruby1.8-dev
\end{bashcode}
\\

Y ahora instalaremos el soporte necesario para interactuar con la base de datos:

\begin{bashcode}
_: apt-get install libmysqlclient-dev
_: gem install mysql
\end{bashcode}
\\

La aplicación web será desarrollada usando Sinatra. Antes de comenzar con la instalación, comprueba que tu versión de RubyGems es igual o superior a la 1.3.6. Esto puede hacerse fácilmente de la siguiente manera:

\begin{bashcode}
_: gem --version
1.3.6
\end{bashcode}
\\

Para instalar Sinatra, haremos lo siguiente:

\begin{bashcode}
_: gem install sinatra
Successfully installed rack-1.4.1
Successfully installed rack-protection-1.2.0
Successfully installed tilt-1.3.3
Successfully installed sinatra-1.3.2
4 gems installed
Installing ri documentation for rack-1.4.1...
...
Installing ri documentation for sinatra-1.3.2...
Installing RDoc documentation for rack-1.4.1...
...
Installing RDoc documentation for sinatra-1.3.2...
\end{bashcode}
\\

Nota: Puede llevar un tiempo hasta que el proceso de instalación muestre cosas por pantalla.


%%
\section{Ejecución}

Una vez que la instalación ha finalizado, vamos a crear la primera aplicación web. Guarda el siguiente fichero como \texttt{web.rb}:

\begin{rubycode}
require 'rubygems'
require 'sinatra'

get '/' do
  'Hello world!'
end
\end{rubycode}

y lanza el servidor web:

\begin{bashcode}
_: ruby web.rb
\end{bashcode}
\\

Nota: Para salir Ctrl + C.

En tu navegador web preferido ve a la dirección \texttt{http://localhost:4567} y encontrarás la aplicación web que acabamos de crear.


%%
\section{Añadiendo soporte para la base de datos}

Para interactuar con la base de datos usaremos ActiveRecord. Este componente es parte de Ruby On Rails, pero también existe como una \emph{gem} independiente. Vamos a instalarla:

\begin{bashcode}
_: gem install activerecord
\end{bashcode}
\\

Ahora vamos a crear una segunda aplicación web. Guarda el siguiente fichero como \texttt{web2.rb}:

\begin{rubycode}
require 'rubygems'
require 'sinatra'
require 'active_record'

class Article < ActiveRecord::Base
end

get '/' do
   'Hello there!'
end
\end{rubycode}

y lanza el servidor web como antes:

\begin{bashcode}
_: ruby web2.rb
\end{bashcode}
\\

En tu navegador web ve a la dirección \texttt{http://localhost:4567} y encontrarás la aplicación web. Muestra lo mismo que la primera aplicación web, pero hemos incluido (aunque no usado) el soporte para la base de datos.


%%
\section{Versiones}

\begin{tabular}{|c|c|}
   \hline
   Software & Versión \\ \hline
   Ruby & 1.8.7 \\ \hline
   RubyGems & 1.8.21 \\ \hline
   Sinatra & 1.3.2 \\ \hline
   ActiveRecord & 3.2.3 \\ \hline
\end{tabular}
