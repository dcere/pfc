\chapter{Balanceador de carga}
\label{web:balanceador}

Usaremos nginx como balanceador de carga, y no como servidor web, que es la manera más habitual de verlo en funcionamiento.


%%
\section{Instalación}

Para instalar nginx, haz lo siguiente:

\begin{bashcode}
_: apt-get update
_: apt-get upgrade
_: apt-get install nginx
\end{bashcode}
\\


%%
\section{Configuración}

Para configurar nginx debemos añadir la parte de balanceo de carga al fichero de configuración \texttt{/etc/nginx/nginx.conf}. Como este fichero no es excesivamente grande, se muestra en su totalidad con la parte modificada resaltada:

\begin{bashcode}
user www-data;
worker_processes  1;

error_log  /var/log/nginx/error.log;
pid        /var/run/nginx.pid;

events {
    worker_connections  1024;
    # multi_accept on;
}

http {
    include       /etc/nginx/mime.types;

    access_log	/var/log/nginx/access.log;

    sendfile        on;
    #tcp_nopush     on;

    #keepalive_timeout  0;
    keepalive_timeout  65;
    tcp_nodelay        on;

    gzip  on;
    gzip_disable "MSIE [1-6]\.(?!.*SV1)";

    include /etc/nginx/conf.d/*.conf;
    include /etc/nginx/sites-enabled/*;


    ### Modified
    upstream web_servers {
      server 155.210.155.73:4567;
      server 155.210.155.178:4567;
    }

    server {
      listen 155.210.155.175:80;
      location / {
        proxy_pass http://web_servers;
      }
    }
    ### End of modification


}
\end{bashcode}
\\


%%
\section{Ejecución}

Para iniciar nginx haremos uso del script localizado en \texttt{/etc/init.d}. Dicho script puede ser usado tanto para iniciarlo:

\begin{bashcode}
_: /etc/init.d/nginx start
\end{bashcode}
\\

como para pararlo:

\begin{bashcode}
_: /etc/init.d/nginx stop
\end{bashcode}
\\


%%
\section{Versiones}

\begin{tabular}{|c|c|}
   \hline
   Software & Versión \\ \hline
   nginx & 0.7.65 \\ \hline
\end{tabular}
