\chapter{MCollective}
\label{comun:mcollective}


%%
\section{Instalación}

El primer paso es instalar rubygems. Para instalarlo, consulta el apéndice \ref{comun:ruby} de instalación de Ruby y rubygems.

Descargamos los paquetes mcollective, mcollective-client y mcollective-common de \url{http://downloads.puppetlabs.com/mcollective/}:

\begin{bashcode}
_: wget http://downloads.puppetlabs.com/mcollective/\
mcollective_1.2.1-1_all.deb
_: wget http://downloads.puppetlabs.com/mcollective/\
mcollective-client_1.2.1-1_all.deb
_: wget http://downloads.puppetlabs.com/mcollective/\
mcollective-common_1.2.1-1_all.deb
\end{bashcode}

Dependiendo de si el nodo va a ser administrador o administrado tendremos que instalar unos u otros paquetes:

\begin{tabular}{|c|c|c|}
   \hline
   Paquete & Nodo administrador & Nodo administrado \\ \hline
   mcollective-common & X & X \\ \hline
   mcollective-client & X &   \\ \hline
   mcollective &  & X \\ \hline
\end{tabular}
\\

Podemos instalar los paquetes de la siguiente manera:

\begin{bashcode}
_: dpkg -i mcollective*.deb
\end{bashcode}

Una vez instalado MCollective, instalamos la librería \texttt{libstomp-ruby}:

\begin{bashcode}
_: apt-get install libstomp-ruby
\end{bashcode}


%%
\section{Configuración}

A continuación hay que editar el fichero de configuración de MCollective. En los nodos administradores (clientes) se encuentra en \texttt{/etc/mcollective/client.cfg} mientras que en los nodos administrados (servidores) se encuentra en \texttt{/etc/mcollective/server.cfg}. Nos aseguramos de que contengan los siguientes valores:

\begin{bashcode}
plugin.stomp.host = 155.210.155.ABC # Direccion IP del MQ broker
plugin.stomp.port = 61613
plugin.stomp.user = mcollective
plugin.stomp.password = mcollective
\end{bashcode}


%%
\section{Comprobación de la instalación}

En el servidor stomp lanza el siguiente comando:

\begin{bashcode}
_: /usr/sbin/service rabbitmq-server start
\end{bashcode}

En los nodos administrados (servidores) lanza éste:

\begin{bashcode}
_: /usr/sbin/service mcollective start
\end{bashcode}

Y en uno de los nodos administradores (clientes) lanzaremos el comando \texttt{mc-ping}. La salida obtenida al ejecutar el comando debería ser similar a la siguiente:

\begin{bashcode}
_: mc-ping
155.210.155.177                       time=46.06 ms
---- ping statistics ----
1 replies max: 46.06 min: 46.06 avg: 46.06
\end{bashcode}


%%
\section{Versiones}

\begin{tabular}{|c|c|}
   \hline
   Software & Versión \\ \hline
   Ubuntu & 10.04 \\ \hline
   MCollective & 1.2.1-1 \\ \hline
   libstomp-ruby & 1.8 (1.0.4-1) \\ \hline
\end{tabular}
