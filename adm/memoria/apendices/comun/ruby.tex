\chapter{Ruby}
\label{comun:ruby}

Antes de poder programar en el lenguaje de programación Ruby deberemos instalar una serie de paquetes. Además de lo necesario para programar en Ruby también instalaremos una serie de herramientas de ayuda.


%%
\section{Instalación}

Empezaremos instalando Ruby, IRB (Interactive Ruby Shell) y RDoc, la herramienta de generación de documentación de Ruby:

\begin{bashcode}
_: apt-get install ruby irb rdoc
\end{bashcode}

A continuación instalaremos RubyGems. Para ello, descarga el paquete más actual de \href{http://rubyforge.org/frs/?group_id=126}{rubyforge}. Durante el resto de la instalación usaremos como ejemplo el paquete 1.8.10, pero los pasos son análogos para cualquier otra versión.

Descomprimimos el paquete:

\begin{bashcode}
_: tar xvf rubygems-1.8.10.tgz
\end{bashcode}

E instalamos RubyGems:

\begin{bashcode}
_: cd rubygems-1.8.10
_: ruby setup.rb
\end{bashcode}

Una vez instalado, comprobamos la versión:

\begin{bashcode}
_: gem --version
1.8.10
\end{bashcode}

Y la actualizamos a la última versión disponible. Es posible que antes de hacer este paso haya que actualizar el sistema operativo.

\begin{bashcode}
_: gem update --system
_: gem --version
1.8.24
\end{bashcode}

En caso de que necesites actualizar el sistema operativo, se hace de esta manera:

\begin{bashcode}
_: apt-get upgrade
\end{bashcode}


%%
\section{Problemas}

Puede que durante la ejecución de programas Ruby, o la instalación de \emph{gems} te encuentres con el siguiente error: \texttt{`require': no such file to load -- mkmf (LoadError)}. La manera de solucionarlo es instalando el paquete \texttt{ruby1.8-dev}:

\begin{bashcode}
_: apt-get install ruby1.8-dev
\end{bashcode}

También puede aparecer el error \texttt{no such file to load -- net/https (LoadError)}. Para solucionarlo, instala \texttt{libopenssl-ruby}:

\begin{bashcode}
_: apt-get install libopenssl-ruby
\end{bashcode}


%%
\section{Versiones}

\begin{tabular}{|c|c|}
   \hline
   Software & Versión \\ \hline
   Ubuntu & 10.04 \\ \hline
   Ruby & 1.8.7 \\ \hline
   RubyGems & 1.8.10 (o superior) \\ \hline
\end{tabular}
