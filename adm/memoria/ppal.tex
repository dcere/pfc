\documentclass[11pt, twoside]{book}

\ifx\pdfoutput\undefined
% we are running LaTeX, not pdflatex
\usepackage{graphicx}
\else
% we are running pdflatex, so convert .eps files to .pdf
\usepackage[pdftex]{graphicx}
\usepackage{epstopdf}
\fi

%\usepackage[cp1252]{inputenc} % soportes para acentos 
\usepackage[spanish,activeacute,english]{babel}  % texto autogenerado en espaol
%\usepackage[latin1]{inputenc}
\usepackage{ucs}
\usepackage[utf8]{inputenc} % Para linux, utf8
\usepackage{fontenc} 
%\usepackage[T2A]{fontenc} 
\usepackage[T1]{fontenc} 
\usepackage{geometry} % figuras

\usepackage{titlesec} % Soporte para modificar apariencia de los titulos 
%\usepackage{fancyhdr}       % Soporte para los encabezados de pagina 
\usepackage{anysize}        % Soporte para el comando \marginsize 
\usepackage{url}
\usepackage{multirow}
\usepackage{subfig}
%\usepackage{fancyvrb}
%\usepackage{verbatim}
\usepackage{listings}
\usepackage{color}

%\usepackage[nottoc]{tocbibind} 
\usepackage{hyperref} 
\usepackage{listings} 
\usepackage{float} 
\usepackage{amsmath} 

\usepackage{enumerate}


% CAPTIONS
%\usepackage[font=small,format=plain,labelfont=bf,up,textfont=it,up]{caption}
\usepackage{caption}
\captionsetup{font=footnotesize,labelfont=bf}

% HEADERS
\usepackage{fancyhdr}
\pagestyle{fancy}
 \fancyhead{}
 \fancyhead[RE]{\footnotesize{ \leftmark}}
 \fancyhead[LO]{\footnotesize{ \rightmark}}
\fancyfoot{}
\fancyfoot[LE]{\thepage}
\fancyfoot[RO]{\thepage}

% Code for creating empty pages
% No headers on empty pages before new chapter
\makeatletter
\def\cleardoublepage{\clearpage\if@twoside \ifodd\c@page\else
    \hbox{}
    \thispagestyle{plain}
    \newpage
    \if@twocolumn\hbox{}\newpage\fi\fi\fi}
\makeatother \clearpage{\pagestyle{plain}\cleardoublepage}

\makeatletter
\def\cleardoublepageempty{\clearpage\if@twoside \ifodd\c@page\else
    \hbox{}
    \thispagestyle{empty}
    \newpage
    \if@twocolumn\hbox{}\newpage\fi\fi\fi}
\makeatother \clearpage{\pagestyle{plain}\cleardoublepage}

%%%%%%%%%%%%%%%%%%%%%%%%%%%%%%%%%%%%%%%%%%%%%%%%%%%%%%%%%%%% 
% Modificamos el aspecto de los titulos 
%%%%%%%%%%%%%%%%%%%%%%%%%%%%%%%%%%%%%%%%%%%%%%%%%%%%%%%%%%%% 
\newcommand{\bigrule}{\titlerule[0.5mm]} 

%\usepackage[Lenny]{fncychap}

\titleformat{\chapter}[display] 
{\bfseries\Huge} 
{%\titlerule 
\filright 
\Huge\chaptertitlename\ 
\Huge\thechapter 
} 
{1mm} 
{\filright} 
[\vspace{0.5mm} \bigrule] 

\bibliographystyle{plain}


% So the code looks nice
\definecolor{gray97}{gray}{.97}
\definecolor{gray75}{gray}{.75}
\definecolor{gray45}{gray}{.45}
 
\usepackage{listings}
\lstset{ frame=Ltb,
     framerule=0pt,
     aboveskip=0.5cm,
     framextopmargin=3pt,
     framexbottommargin=3pt,
     framexleftmargin=0.4cm,
     framesep=0pt,
     rulesep=.4pt,
     backgroundcolor=\color{gray97},
     rulesepcolor=\color{black},
     %
     stringstyle=\ttfamily,
     showstringspaces = false,
     basicstyle=\small\ttfamily,
     commentstyle=\color{gray45},
     keywordstyle=\bfseries,
     %
     numbers=left,
     numbersep=15pt,
     numberstyle=\tiny,
     numberfirstline = false,
     breaklines=true,
   }
 
% minimizar fragmentado de listados
\lstnewenvironment{listing}[1][]
   {\lstset{#1}\pagebreak[0]}{\pagebreak[0]}
 
\lstdefinestyle{consola}
   {basicstyle=\scriptsize\bf\ttfamily,
    backgroundcolor=\color{gray75},
   }
 
% Till here for code (lang=?)


% Numeramos hasta 3 niveles y los incluimos en la tabla de contenidos
\setcounter{secnumdepth}{3}
\setcounter{tocdepth}{3}

% Para que ponga 'tabla' en vez de 'cuadro'
% y la bibliografía se llame 'Bibliografía'
\addto\captionsspanish{%
\def\bibname{Bibliografía}
\def\tablename{Tabla}%
\def\listtablename{Índice de tablas}
}


\begin{document}

\selectlanguage{spanish}

\renewcommand{\listtablename}{Índice de tablas}
\renewcommand{\bibname}{Bibliografía}

\begin{titlepage} 
\begin{center} 
 
\includegraphics*[height=3.5cm]{imagenes/unizar.jpg}\\ 

\vspace*{1.5cm} 
{\large Proyecto Fin de Carrera}\\ 
\vspace*{0.2cm} 
{\large Ingeniería en Informática}\\ 
\vspace*{1.5cm} 
{\huge \textbf{Diseño e implementación de un sistema de ejecución de trabajos distribuidos\\}}
\vspace*{2cm} 
{\Large \textbf{David Ceresuela Palomera\\}}
\vspace*{2cm} 
{\normalsize Director: Javier Celaya}\\ 
\vspace*{1.5cm} 
{\normalsize Departamento de Informática e Ingeniería de Sistemas}\\ 
{Centro Politécnico Superior}\\ 
{Universidad de Zaragoza}\\ 
\vspace*{3.5cm} 
{\normalsize Curso 2011/2012}\\ 
{\normalsize Junio 2012}\\ 
\end{center} 
\end{titlepage} 

\cleardoublepageempty

\selectlanguage{spanish}

\frontmatter % numeración romana, capítulos aparecen en la tabla de contenidos

\chapter{Resumen}
\label{cap:resumen}


A la hora de ejecutar trabajos en un entorno distribuido la aproximación clásica ha sido bien el uso de un \emph{cluster} de ordenadores o bien el uso de la computación en malla o \emph{grid}. Con la proliferación de entornos \emph{cloud} durante estos últimos años y su facilidad de uso, parece que una nueva opción se abre para la ejecución de este tipo de trabajos.\\

De hecho, la ejecución de trabajos distribuidos es uno de los principales usos dentro del ámbito de los sistemas \emph{cloud}. Sin embargo, la administración de este tipo de sistemas dista de ser sencilla: cuestiones como la puesta en marcha del sistema, el aprovisionamiento de nodos, las modificaciones del sistema y la evolución y actualización del mismo suponen una tarea intensa y pesada.\\

En vista de lo cual, en este proyecto se ha diseñado una solución capaz de automatizar la administración de sistemas \emph{cloud}, y en particular de un sistema de ejecución de trabajos distribuidos. Para ello se han estudiado entornos clásicos de ejecución de trabajos como Torque y entornos de ejecución de trabajos en \emph{cloud} como AppScale. Además, se han estudiado herramientas clásicas de configuración automática de sistemas como Puppet y CFEngine. El objetivo principal de estas herramientas de configuración de sistemas es la gestión del nodo. En este proyecto se ha extendido la funcionalidad de una de estas herramientas -- Puppet -- añadiéndole la capacidad de gestión de sistemas \emph{cloud}.\\

Como resultado de este proyecto se presenta una solución capaz de administrar de forma automática sistemas de ejecución de trabajos distribuidos. La validación de esta solución se ha llevado a cabo sobre los entornos de ejecución de trabajos Torque y AppScale y también, para mostrar su carácter genérico, sobre una arquitectura de servicios web de tres niveles.


\tableofcontents
\listoftables
\listoffigures

%chapters
\mainmatter % numeración arábiga

\chapter{Introducción}
\label{cap:introduccion}


Durante los últimos años la computación en la nube ha ido ganando importancia de manera progresiva. La capacidad de usar la computación como un servicio permite a los usuarios finales de una aplicación acceder a ésta a través de un navegador web, una aplicación móvil o un cliente de escritorio mientras que la lógica de la aplicación y los datos se encuentran en servidores situados en una localización remota. Las aplicaciones en la nube tratan de proporcionar al usuario el mismo servicio y rendimiento que las aplicaciones instaladas localmente en su ordenador.\\

A lo largo de este proyecto se verán tres ejemplos de infraestructuras distribuidas. La primera de ellas es la infraestrucutra de ejecución de trabajos. Este tipo de infraestructuras está especializada en la ejecución de grandes cargas de trabajo paralelizable e intensivo en computación. Son por lo tanto idóneas para ser usadas en la computación de altas prestaciones. Dentro de esta infraestructura distribuida los ejemplos más claros que podemos encontrar son Condor y Torque.

La segunda infraestructura es la de servicios web en tres capas. Este tipo de infraestructuras tiene tres niveles claramente diferenciados: balanceo o distribución de carga, servidor web y base de datos. El primer nivel de esta arquitectura es el encargado de distribuir la carga (las peticiones web) a los elementos del segundo nivel. Éstos procesarán las peticiones web y para ello puede que tengan que consultar o modificar ciertos datos. Los datos de la aplicación se encuentran en el tercer nivel, y  por consiguiente, cada vez que uno de los elementos del segundo nivel necesite leer información de la base de datos o modificarla, accederá a este tercer nivel. En esta infraestructura no hay un ejemplo claro, pero cualquier página web profesional de hoy en día se sustenta en una arquitectura similar a ésta.

La tercera y última es AppScale, una implementación \emph{open source} del App Engine de Google. App Engine permite alojar aplicaciones web en la infraestructura que Google posee. Además de presentar esta funcionalidad AppScale también ofrece las APIs de EC2, MapReduce y Neptune. La API de EC2 añade a las aplicaciones la capacidad de interactuar con máquinas alojadas en Amazon EC2. La API de MapReduce permite escribir aplicaciones que hagan uso del \emph{framework} (o paradigma?) MapReduce. La última API, Neptune, añade a App Engine la capacidad de usar los nodos de la infraestructura para ejecutar trabajos. Los trabajos más representativos que puede ejecutar son: de entrada, de salida y MPI. El trabajo de entrada sirve para subir ficheros (generalmente el código que se ejecutará) a la infraestructura, el de salida para traer ficheros (generalmente los resultados obtenidos después de la ejecución) y el de MPI para ejecutar un trabajo MPI.

La infraestructura necesaria para dar soporte a todas estas APIs ya no es tan sencilla como en los dos casos anteriores. De hecho, las anteriores infraestructuras estarían contenidas en ésta. Hay dos maneras de definir la infraestructura de AppScale. En la primera de ellas se define un despliegue por defecto, en el que un nodo toma el rol de \emph{controller} y el resto de nodos toman el rol de \emph{servers}. El nodo \emph{controller} es el que carga con la responsabilidad de la coordinación y los nodos \emph{servers} son los que llevan cabo la mayor parte del trabajo. La segunda manera de definir la infraestructura es hacerlo a través de un despliegue personalizado. En este despliegue podemos definir con más precisión los roles que queremos que desempeñe un nodo. Entre todos los posibles roles que AppScale ofrece, los más interesantes desde nuestro punto de vista son: \emph{master}, \emph{appengine}, \emph{database}, \emph{login} y \emph{open}.\\

De igual manera, las herramientas de administración de sistemas (o herramientas de gestión de configuración) también han experimentado un considerable avance. Con entornos cada vez más heterogéneos y complejos la administración de sistemas de manera manual ya no es una opción. Entre todo el conjunto de  herramientas destacan de manera especial Puppet y CFEngine. Puppet es una herramienta de gestión de configuración basada en un lenguaje  declarativo: el usuario especifica qué estado debe alcanzarse y Puppet se encarga de hacerlo. CFEngine permite al usuario un control más fino de cómo se hacen las cosas, ganando velocidad a costa de perder elementos de más alto nivel.\\

Sin embargo, estas herramientas de gestión de la configuración carecen de la funcionalidad requerida para administrar infraestructuras distribuidas.\\

Si tomamos la administración de un cloud como la administración de las MV que forman los nodos del cloud nos damos cuenta de que la administración es puramente software.\\


\section{Contexto del proyecto}

Para la realización de este proyecto de fin de carrera se ha hecho uso del laboratorio 1.03b de investigación que el Departamento de Informática e Ingeniería de Sistemas posee en la Escuela de Ingeniería y Arquitectura de la Universidad de Zaragoza. Los ordenadores que forman este laboratorio poseen procesadores con soporte de virtualización, lo que permite la creación de diversas máquinas virtuales. La creación de los distintos tipos de cloud que representan cada una de las infraestructuras distribuidas se ha llevado a cabo a través de máquinas virtuales alojadas en distintos ordenadores del laboratorio.\\

En este laboratorio se ha comprobado la validez de la extensión para administración de clouds introducida en la herramienta de gestión de configuraciones Puppet que se ha desarrollado a lo largo del proyecto de fin de carrera.


\section{Objetivos}

El objetivo de este proyecto es proporcionar una herramienta que facilite la puesta en marcha de infraestructuras distribuidas y su posterior mantenimiento. Las tareas principales en las que se puede dividir este proyecto son:

\begin{enumerate}
\item Estudio de algunas de las infraestructuras distribuidas existentes profundizando en la parte relativa a la ejecución de trabajos distribuidos.
\item Análisis de las herramientas de administración de virtualización \emph{hardware}.
\item Investigación de las herramientas de gestión de configuración existentes más relevantes y elección de aquella que mayor facilidad de integración y uso proporcione.
\item Extensión de la herramienta de gestión de configuración para que soporte la puesta en marcha y el mantenimiento de un sistema de ejecución de trabajos distribuidos.
\end{enumerate}


\section{Organización de la memoria}

El resto de este documento queda organizado de la siguiente manera:
\begin{description}
\item[Capítulo \ref{cap:trabajo}] Extensión de Puppet para soporte de infraestructuras distribuidas.
\item[Capítulo \ref{cap:appscale}] Uso de la extensión de Puppet para soporte de infraestructura AppScale.
\item[Capítulo \ref{cap:web}] Uso de la extensión de Puppet para soporte de infraestructura web de tres niveles.
\item[Capítulo \ref{cap:torque}] Uso de la extensión de Puppet para soporte de infraestructura de trabajos distribuidos.
\item[Capítulo \ref{cap:conclusiones}] Conclusiones.
\end{description}


\section{Agradecimientos}

Agradecimientos

\chapter{Trabajo desarrollado}
\label{cap:trabajo}

{\sf

Trabajo, trabajo.
}

\chapter{Conclusiones}
\label{cap:conclusiones}


En este proyecto fin de carrera se ha creado un modelo de recursos distribuidos para facilitar la puesta en marcha y la automatización de distintas infraestructuras distribuidas. \\

Para comprobar la validez del modelo creado se ha extendido la herramienta de gestión de configuración Puppet añadiéndole el recurso sistema distribuido o \emph{cloud}. A la hora de definir el recurso sistema distribuido se han tenido en cuenta las características propias de este tipo de recursos, como la disponibilidad, las prestaciones y las dependencias y los conceptos de iteración y convergencia que las herramientas de configuración proveen. \\

Posteriormente se ha comprobado la extensión realizada haciendo que la herramienta de gestión de configuración Puppet sea capaz de administrar infraestructuras distribuidas de ejecución de trabajos como AppScale y TORQUE. Sin embargo no se han descuidado el resto de aspectos de estas infraestructuras. En concreto, en la infraestructura AppScale se soportan los dos tipos de despliegue y la totalidad de los roles posibles, siendo por lo tanto capaz de gestionar al completo una infraestructura de este tipo, y no sólo la parte de ejecución de trabajos. \\

Además, no se ha comprobado la validez del modelo exclusivamente con infraestructuras de ejecución de trabajos, sino que se ha constatado que también es válido para infraestructuras más generales como por ejemplo la de servicios web en tres niveles. Se puede decir, por consiguiente, que se ha ido más allá del mero cumplimiento de los objetivos del proyecto. \\

En cuanto al trabajo futuro, al ser este un primer prototipo, se encuentra abierto a multiples ampliaciones. El paso más evidente sería integrar este modelo dentro de la herramienta Puppet. Sería especialmente interesante modificar la gramática de descripción de recursos para diferenciar entre los recursos locales y los distribuidos. Posteriormente, se deberían incorporar las funcionalidades básicas de gestión de recursos distribuidos. De esta manera los futuros administradores de este tipo de infraestructuras contarían de manera automática con un conjunto de funciones que ayudan en la administración de sistemas distribuidos en Puppet.

La validez del trabajo realizado se ha comprobado únicamente sobre máquinas virtuales con un sistema operativo con núcleo Linux. Son de sobra conocidas las diferencias que existen entre las distintas distribuciones de Linux, por no hablar ya de las diferencias entre sistemas operativos completamente distintos. Realizar proveedores que sean válidos para cualquier sistema operativo es el segundo paso que debería realizarse. Además, estas máquinas virtuales se alojan en ordenadores de un laboratorio en la EINA; utilizar máquinas virtuales alojadas en proveedores como Amazon o Rackspace supondría un broche de oro. \\

Por último, y a modo de valoración personal, este proyecto me ha permitido aprender una gran cantidad de nuevas tecnologías; conocer una herramienta de gestión de configuración como Puppet, que es de las que más relevancia está tomando; estudiar las infraestructuras de AppScale, TORQUE y de servicios web e introducirme en un nuevo lenguaje de programación como Ruby que aporta conceptos muy interesantes y distintos a los lenguajes estudiados a lo largo de la carrera. Aunque en ocasiones la documentación fuese más escasa de lo deseado, la oportunidad de aprender nuevos conceptos y tecnologías es algo que personalmente valoro muy positivamente de este proyecto.


\addcontentsline{toc}{chapter}{Bibliografía}
\nocite{*}
\bibliographystyle{abbrv}
\bibliography{biblio/references}

% apendices
\appendix

%\include{capitulos/apendice_a}
%\include{capitulos/apendice_b}
%\include{capitulos/apendice_c}


\cleardoublepageempty



\end{document}
