\chapter{Código fuente}
\label{anx:codigo}


\section{generic-module}
\subsection{mcollective\_leader.rb}


\begin{lstlisting}
require 'mcollective'
include MCollective::RPC

# MCollective leader client used to help with leader election algorithm
class MCollectiveLeaderClient < MCollectiveClient
   
   
   # Creates a new MCollective leader client.
   def initialize(client)
      super(client)
   end
   
   
   # Asks all nodes for their ID.
   def ask_id
   
      ids = []
      regex = /^.*ID: (\d+).*$/
   
      puts "Retrieving nodes' ids via MCollective Leader client"
      output = Helpers.rpcresults @mc.ask_id()
      
      puts "Complete output:"
      puts "---------------------------"
      puts "#{output}"
      puts "---------------------------"
      
      output.each_line do |line|
         m = line.match(regex)
         if m
            ids << m[1].to_i()      # Send them as integers
         end
      end
      
      return ids
   
   end
   
   
   # Sends all nodes the leader's ID.
   def new_leader(id)
   
      puts "Sending new leader information via MCollective Leader client"
      output = @mc.new_leader(:leader_id => id)
      return output
   
   end
   
   
end
\end{lstlisting}


\subsection{cloud.rb}


\begin{lstlisting}
class Cloud

   attr_reader :resource

   def initialize(infrastructure, leader, resource, error_function)

      @vm_manager = CloudVM.new(infrastructure, leader, resource, error_function)
      @leader     = leader
      @resource   = resource
      @err        = error_function

   end

   #############################################################################
   # Start cloud functions
   #############################################################################

   # Starting function for leader node.
   def leader_start(cloud_type, vm_ips, vm_ip_roles, vm_img_roles, pm_up,
                    monitor_function)
      
      # We are the leader
      puts "#{MY_IP} is the leader"
      
      # Check wether virtual machines are alive or not
      puts "Checking whether virtual machines are alive..."
      alive = {}
      vm_ips.each do |vm|
         if alive?(vm)
            alive[vm] = true
         else
            alive[vm] = false
            puts "#{vm} is down"
         end
      end
      
      # Monitor the alive machines. Start and configure the dead ones.
      deads = false
      vm_ips.each do |vm|
         if alive[vm]
            # If they are alive, monitor them
            puts "Monitoring #{vm}..."
            @vm_manager.monitor_vm(vm, vm_ip_roles, monitor_function)
            puts "...Monitored"
         else
            # If they are not alive, start and configure them
            puts "Starting #{vm}..."
            @vm_manager.start_vm(vm, vm_ip_roles, vm_img_roles, pm_up)
            puts "...Started"
            deads = true
         end
      end
      
      # Wait for all machines to be started
      unless deads
      
         # If not already started, start the cloud
         unless File.exists?("/tmp/cloud-#{@resource[:name]}")
            
            # Copy important files to all machines
            puts "Copying important files to all virtual machines"
            copy_cloud_files(vm_ips, cloud_type)
         
            # Start the cloud
            if start_cloud(@resource, vm_ips, vm_ip_roles)
               
               # Make cloud nodes manage themselves
               auto_manage(cloud_type)     # Only if cloud was started properly
               
               # Create file
               cloud_file = File.open("/tmp/cloud-#{@resource[:name]}", 'w')
               cloud_file.puts(@resource[:name])
               cloud_file.close
               
               puts "==================="
               puts "== Cloud started =="
               puts "==================="
            else
               puts "Impossible to start cloud"
            end
         end      # unless File
         
      end      # unless deads

   end


   # Starting function for common (non-leader) nodes.
   def common_start()

      # We are not the leader or we have not received our ID yet
      puts "#{MY_IP} is not the leader"
            
      if @leader.id == -1
         
         # If we have not received our ID, let's assume we will be the leader
         @leader.set_id(0)
         @leader.set_leader(0)
         
         puts "#{MY_IP} will be the leader"
         
         # Create your ssh key
         CloudSSH.generate_ssh_key()
         
      else
         
         # If we have received our ID, try to become leader
         puts "Trying to become leader..."
         
         # Get all machines' IDs
         mcc = MCollectiveLeaderClient.new("leader")
         ids = mcc.ask_id()
         
         # See if some other machine is leader
         exists_leader = false
         ids.each do |id|
            if id < @leader.id
               exists_leader = true 
               break
            end
         end
         
         # If there is no leader, we will be the new leader
         if !exists_leader
            mcc.new_leader(@leader.id.to_s())
            puts "...#{MY_IP} will be leader"
            
            # Create your ssh key
            CloudSSH.generate_ssh_key()
         else
            puts "...Some other machine is/should be leader"
         end
         mcc.disconnect
         
         return
      end

   end


   # Starting function for nodes which do not belong to the cloud.
   def not_cloud_start(cloud_type, vm_ips, vm_ip_roles, vm_img_roles, pm_up)
      
      # Try to find one virtual machine that is already running
      vm_ips.each do |vm|
         if alive?(vm)
            # This machine is running
            puts "#{vm} is up"
            vm_leader = vm

            # Inform the user of this machine
            puts "#{vm_leader} is already running"
            puts "Do 'puppet apply manifest.pp' on #{vm_leader}"
            return
         end
      end
      
      # No machines are running
      puts "All virtual machines are stopped"
      puts "Starting one of them..."
   
      # Start one of the virtual machines
      vm = vm_ips[rand(vm_ips.count)]     # Choose one randomly
      puts "Starting #{vm} ..."
      
      @vm_manager.start_vm(vm, vm_ip_roles, vm_img_roles, pm_up)
      
      # That virtual machine will be the "leader" (actually the chosen one)
      vm_leader = vm
      
      # Copy important files to it
      #copy_cloud_files(vm_leader, cloud_type)
      
      puts "#{vm_leader} is being started"
      puts "Once started, do 'puppet apply manifest.pp' on #{vm_leader}"

   end


   # Monitoring function for leader node.
   def leader_monitoring(monitor_function)

      puts "#{MY_IP} is the leader"
      
      # Do monitoring
      deads = []
      vm_ips, vm_ip_roles, vm_img_roles = obtain_vm_data(@resource)
      vm_ips.each do |vm|
         puts "Monitoring #{vm}..."
         unless @vm_manager.monitor_vm(vm, vm_ip_roles, monitor_function)
            deads << vm
         end
         puts "...Monitored"
      end
      
      # Check pool of physical machines
      pm_up, pm_down = check_pool()
      
      if deads.count == 0
         puts "=========================="
         puts "== Cloud up and running =="
         puts "=========================="
      else
         # Raise again the dead machines
         deads.each do |vm|
            @vm_manager.start_vm(vm, vm_ip_roles, vm_img_roles, pm_up)
         end
      end
               
   end


   #############################################################################
   # Stop cloud functions
   #############################################################################

   # Stoping function for leader node.
   def leader_stop(cloud_type, stop_function)

      # Stop cron jobs on all machines
      stop_cron_jobs(cloud_type)
      
      # Stop the cloud with the provided method
      vm_ips, vm_ip_roles, vm_img_roles = obtain_vm_data(@resource)
      stop_function.call(@resource, vm_ip_roles)
      
      # Shutdown and undefine all virtual machines explicitly created for
      # this cloud
      @vm_manager.shutdown_vms()
      
      # Delete cloud related files
      delete_files()
      
      # Note: As all the files deleted so far are located in the /tmp directory
      # only the machines that are still alive need to delete these files.
      # If the machine was shut down, these files will not be there the next
      # time it is started, so there is no need to delete them.
      
      puts "==================="
      puts "== Cloud stopped =="
      puts "==================="

   end

   ## Shuts down virtual machines
   def shutdown_vms()
      @vm_manager.shutdown_vms()
   end


   # Stops cron jobs on all machines.
   def stop_cron_jobs(cloud_type)

      mcc = MCollectiveCronClient.new("cronos")
      string = "init-#{cloud_type}"
      mcc.delete_line(CloudCron::CRON_FILE, string)

   end


   # Deletes cloud files on all machines.
   def delete_files()

      puts "Deleting cloud files on all machines..."
      
      # Create an MCollective client so that we avoid errors that appear
      # when you create more than one client in a short time
      mcc = MCollectiveFilesClient.new("files")
      
      # Delete leader, id, last_id and last_mac files on all machines
      # (leader included)
      mcc.delete_file(CloudLeader::LEADER_FILE)             # Leader ID
      mcc.delete_file(CloudLeader::ID_FILE)                 # ID
      mcc.delete_file(CloudLeader::LAST_ID_FILE)            # Last ID
      mcc.delete_file(CloudVM::LAST_MAC_FILE)               # Last MAC address
      mcc.disconnect       # Now it can be disconnected
      
      # Delete rest of regular files on leader machine
      files = [CloudInfrastructure::DOMAINS_FILE,           # Domains file
               "/tmp/cloud-#{@resource[:name]}"]            # Cloud file
      files.each do |file|
         if File.exists?(file)
            File.delete(file)
         else
            puts "File #{file} does not exist"
         end
      end

   end


   #############################################################################
   # Auxiliar functions
   #############################################################################

   # Checks the pool of physical machines are OK.
   def check_pool()

      machines_up = []
      machines_down = []
      machines = @resource[:pool]
      machines.each do |machine|
         if alive?(machine)
            machines_up << machine
         else
            machines_down << machine
         end
      end
      return machines_up, machines_down
      
   end


   # Checks if this node is the leader
   def leader?()
    
      return @leader.leader?

   end


   # Copies important files to all machines inside <ips>.
   def copy_cloud_files(ips, cloud_type)
   # Use MCollective?
   #  - Without MCollective we are able to send it to both one machine or multiple
   #    machines without changing anything, so not really.

      ips.each do |vm|
      
         if vm != MY_IP
            
            # Cloud manifest
            file = "init-#{cloud_type}.pp"
            path = "/etc/puppet/modules/#{cloud_type}/manifests/#{file}"
            out, success = CloudSSH.copy_remote(path, vm, path)
            unless success
               @err.call "Impossible to copy #{file} to #{vm}"
            end

         end
         
      end
      
   end


   # Makes the cloud auto-manageable through crontab files.
   def auto_manage(cloud_type)

      cron_file = "crontab-#{cloud_type}"
      path = "/etc/puppet/modules/#{cloud_type}/files/cron/#{cron_file}"
      
      if File.exists?(path)
         file = File.open(path, 'r')
         line = file.read().chomp()
         file.close
         
         # Add the 'puppet apply manifest.pp' line to the crontab file
         mcc = MCollectiveCronClient.new("cronos")
         mcc.add_line(CloudCron::CRON_FILE, line)
         mcc.disconnect
      else
         @err.call "Impossible to find cron file at #{path}"
      end
      
   end


   #############################################################################
   # Helper functions
   #############################################################################


   # Checks if a machine is alive.
   def alive?(ip)

      ping = "ping -q -c 1 -w 4"
      result = `#{ping} #{ip}`
      return $?.exitstatus == 0

   end


end
\end{lstlisting}


\subsection{cloudmonitor.rb}


\begin{lstlisting}
# Generic monitor functions for a distributed infrastructure
module CloudMonitor

   PING = "ping -q -c 1 -w 4"

   # Checks if the <vm> machine is alive.
   def self.ping(vm)

      result = `#{PING} #{vm}`
      if $?.exitstatus == 0
         puts "[CloudMonitor] #{vm} is up"
         return true
      else
         puts "[CloudMonitor] #{vm} is down"
         return false
      end
      
   end


   # Checks if MCollective is installed in <vm>.
   def self.mcollective_installed(user, vm)
      
      installed = true
      
      # Client configuration file
      client_file = "/etc/mcollective/client.cfg"
      command = "cat #{client_file} > /dev/null 2> /dev/null"
      out, success = CloudSSH.execute_remote(command, user, vm)
      unless success
         puts "[CloudMonitor] #{client_file} does not exist on #{user}@#{vm}"
         installed = false
      end

      # Server configuration file
      server_file = "/etc/mcollective/server.cfg"
      command = "cat #{server_file} > /dev/null 2> /dev/null"
      out, success = CloudSSH.execute_remote(command, user, vm)
      unless success
         puts "[CloudMonitor] #{server_file} does not exist on #{user}@#{vm}"
         installed = false
      end
      
      return installed
      
   end


   # Checks if MCollective is running in <vm>.
   def self.mcollective_running(user, vm)
      
      command = "ps aux | grep -v grep | grep mcollective"
      out, success = CloudSSH.execute_remote(command, user, vm)
      unless success
         puts "MCollective is not running on #{vm}"
         command = "/usr/bin/service mcollective start"
         out, success = CloudSSH.execute_remote(command, user, vm)
         unless success
            puts "[CloudMonitor] Impossible to start mcollective on #{user}@#{vm}"
            return false
         else
            puts "[CloudMonitor] MCollective is running now on #{vm}"
            return true
         end
      else
         puts "[CloudMonitor] MCollective is running on #{vm}"
         return true
      end
      
   end
   
end
\end{lstlisting}


\subsection{cloudinfrastructure.rb}


\begin{lstlisting}
# Manages the infrastructure of a cloud with KVM support.
class CloudInfrastructure

   # Constants
   VIRSH_CONNECT = "virsh -c qemu:///system"
   DOMAINS_FILE = "/tmp/defined-domains"


   #############################################################################
   # Domain functions
   #############################################################################


   # Writes the virtual machine's domain file.
   def write_domain(virtual_machine, domain_file_path, template_path)

      require 'erb'
      template = File.open(template_path, 'r').read()
      erb = ERB.new(template)
      domain_file = File.open(domain_file_path, 'w')
      domain_file.write(erb.result(virtual_machine.get_binding))
      domain_file.close

   end


   # Defines a domain for a virtual machine on a physical machine.
   def define_domain(domain_file_name, pm_user, pm)

      command = "#{VIRSH_CONNECT} define #{domain_file_name}"
      out, success = CloudSSH.execute_remote(command, pm_user, pm)
      return success
      
   end


   # Starts a domain on a physical machine.
   def start_domain(domain, pm_user, pm)

      command = "#{VIRSH_CONNECT} start #{domain}"
      out, success = CloudSSH.execute_remote(command, pm_user, pm)
      return success
      
   end


   # Saves the virtual machine's domain name in a file.
   def save_domain_name(domain, pm_user, pm)

      command = "echo #{domain} >> #{DOMAINS_FILE}"
      out, success = CloudSSH.execute_remote(command, pm_user, pm)
      return success

   end


   # Shuts down a domain.
   def shutdown_domain(domain, pm_user, pm)

      command = "#{VIRSH_CONNECT} shutdown #{domain}"
      out, success = CloudSSH.execute_remote(command, pm_user, pm)
      return success

   end


   # Undefines a domain.
   def undefine_domain(domain, pm_user, pm)

      command = "#{VIRSH_CONNECT} undefine #{domain}"
      out, success = CloudSSH.execute_remote(command, pm_user, pm)
      return success

   end

end
\end{lstlisting}


\subsection{vm.rb}


\begin{lstlisting}
# Virtual machine class
class VM

   attr_accessor :vm
   
   
   # Creates a description of a virtual machine.
   def initialize(name, uuid, disk, mac, mem, ncpu)
      @vm = {
         :name => "#{name}",
         :uuid => "#{uuid}",
         :disk => "#{disk}",
         :mac  => "#{mac}",
         :mem  => "#{mem}",
         :ncpu => "#{ncpu}"}
   end
   
   
   # Provides binding for ERB templates.
   def get_binding
      binding()
   end
   
end
\end{lstlisting}


\subsection{mcollective\_client.rb}


\begin{lstlisting}
require 'mcollective'
include MCollective::RPC

# MCollective base client
class MCollectiveClient
   
   
   # Creates a new MCollective base client.
   def initialize(client)
      @client = client
      @mc = rpcclient(@client)
   end
   
   
   # Disconnects an MCollective client.
   def disconnect
      puts "Disconnecting MCollective client..."
      @mc.disconnect
   end
   
   
end
\end{lstlisting}


\subsection{gedit.sh}


\begin{lstlisting}
gedit cloud*.rb &
\end{lstlisting}


\subsection{mcollective\_files.rb}


\begin{lstlisting}
require 'mcollective'
include MCollective::RPC

# MCollective files client used to manage files
class MCollectiveFilesClient < MCollectiveClient
   
   
   # Creates a new MCollective files client.
   def initialize(client)
      super(client)
   end
   
   
   # Cretes a new file in <path> containing <content>.
   def create_file(path, content)
   
      puts "Sending path and content via MCollective Files client"
      @mc.create(:path => path, :content => content)
      printrpcstats
   
   end
   
   
   # Appends <content> to the file located at <path>.
   def append_content(path, content)
   
      puts "Sending path and content via MCollective Files client"
      @mc.append(:path => path, :content => content)
      printrpcstats
   
   end
   
   
   # Deletes a file located at <path>.
   def delete_file(path)
   
      puts "Sending path via MCollective Files client"
      @mc.delete(:path => path)
      printrpcstats
   
   end
   
   
end
\end{lstlisting}


\subsection{cloudinterface.rb}


\begin{lstlisting}
class CloudInterface

   def initialize()
   end

   # The functions in this file are defined the same in all providers, but each
   # one implements them in their own way. Thus, the headers cannot be modified.

   # Starts a cloud formed by <vm_ips> performing <vm_ip_roles>.
   def start_cloud(resource, vm_ips, vm_ip_roles)

      puts "Starting the cloud"
      
      # ...

   end


   # Obtains vm data from manifest parameters.
   def obtain_vm_data(resource)

      puts "Obtaining virtual machines' data"
      
      # ...
      
      ips = ["IP_A", "IP_B"]
      ip_roles  = {:rol_a => "IP_A", :rol_b => "IP_B"}
      img_roles = {:rol_a => "/path/to/IMG_A", :rol_b => "/path/to/IMG_B"}
      
      return ips, ip_roles, img_roles
      
   end
   
end
\end{lstlisting}


\subsection{mcollective\_cron.rb}


\begin{lstlisting}
require 'mcollective'
include MCollective::RPC

# MCollective cron client used to manage crontab files
class MCollectiveCronClient < MCollectiveClient
   
   
   # Creates a new MCollective cron client.
   def initialize(client)
      super(client)
   end
   
   
   # Adds the line <line> to the crontab file located at <path>.
   def add_line(path, line)
   
      puts "Sending path and line via MCollective Cron client"
      @mc.add_line(:path => path, :line => line)
      printrpcstats
   
   end
   
   
   # Deletes lines that contain the <string> string in the crontab located at <path>.
   def delete_line(path, string)
   
      puts "Sending path and string via MCollective Cron client"
      @mc.delete_line(:path => path, :string => string)
      printrpcstats
   
   end
   
   
end
\end{lstlisting}


\subsection{cloudcron.rb}


\begin{lstlisting}
# Manages cron files in remote machines
class CloudCron

   CRON_FILE = "/var/spool/cron/crontabs/root"

   def initialize(crontab = CRON_FILE)

      @crontab = crontab

   end


   # Creates a new command line.
   def create_line(time, command, out = "/dev/null", err = "/dev/null")

      return "#{time} #{command} > #{out} 2> #{err}"

   end


   # Adds a line to the crontab file.
   def add_line(line, user, vm)

      #line = "#{@time} #{@command} > #{@out} 2> #{@err}"
      command = "echo \"#{line}\" >> #{@crontab}"
      out, success = CloudSSH.execute_remote(command, user, vm)
      unless success
         puts "Impossible to add line in cron file #{@crontab} at #{vm}"
      end

      return success

   end


   # Deletes a line from the crontab file.
   def delete_line(line, user, vm)

      #line = "#{@time} #{@command} > #{@out} 2> #{@err}"
      command = "sed -i '/#{line}/d' #{@crontab}"
      out, success = CloudSSH.execute_remote(command, user, vm)
      unless success
         puts "Impossible to delete line in cron file #{@crontab} at #{vm}"
      end

      return success

   end


   # Deletes a line containing the word <word> from the crontab file.
   def delete_line_with_word(word, user, vm)

      command = "sed -i '/#{word}/d' #{@crontab}"
      out, success = CloudSSH.execute_remote(command, user, vm)
      unless success
         puts "Impossible to delete line with #{word} in " + 
              "cron file #{@crontab} at #{vm}"
      end

      return success

   end

end
\end{lstlisting}


\subsection{cloudvm.rb}


\begin{lstlisting}
# Manages the virtual machines that form a cloud.
class CloudVM

   # Constants
   LAST_MAC_FILE = "/tmp/cloud-last-mac"


   def initialize(infrastructure, leader, resource, error_function)

      @infrastructure = infrastructure
      @leader         = leader
      @resource       = resource
      @err            = error_function

   end


   # Defines and starts a virtual machine.
   def start_vm(vm, ip_roles, img_roles, pm_up)
      
      # Get virtual machine's MAC address
      puts "Getting VM's MAC address"
      mac_address = get_vm_mac()
      
      # Get virtual machine's image disk
      puts "Getting VM's image disk"
      disk = get_vm_disk(vm, ip_roles, img_roles)
      
      # Define a new virtual machine
      id = rand(10000)      # Choose a number for domain name randomly
      vm_name = "myvm-#{id}"
      vm_uuid = `uuidgen`
      vm_mac  = mac_address
      vm_disk = disk
      vm_mem  = @resource[:vm_mem]
      vm_ncpu = @resource[:vm_ncpu]
      myvm = VM.new(vm_name, vm_uuid, vm_disk, vm_mac, vm_mem, vm_ncpu)
      
      # Write virtual machine's domain file
      domain_file_name = "cloud-%s-%s.xml" % [@resource[:name], vm_name]
      domain_file_path = File.dirname(@resource[:vm_domain]) + 
                         "/" + "#{domain_file_name}"
      template_path = @resource[:vm_domain]
      @infrastructure.write_domain(myvm, domain_file_path, template_path)
      puts "Domain file written"
      
      # Choose a physical machine to host the virtual machine
      pm = pm_up[rand(pm_up.count)] # Choose randomly
      
      # Copy ssh key to physical machine
      puts "Copying ssh key to physical machine"
      pm_user = @resource[:pm_user]
      pm_password = @resource[:pm_password]
      out, success = CloudSSH.copy_ssh_key(pm_user, pm, pm_password)
      
      # Copy the domain definition file to the physical machine
      puts "Copying the domain definition file to the physical machine..."
      domain_file_src = domain_file_path
      domain_file_dst = "/tmp/" + domain_file_name
      
      out, success = CloudSSH.copy_remote(domain_file_src, pm, domain_file_dst,
                                          pm_user)
      if success
         puts "domain definition file copied"
         
         # Delete the local copy
         File.delete(domain_file_src)
      else
         @err.call "Impossible to copy #{vm_name}\'s domain definition file"
      end
      
      # Define the domain in the physical machine
      puts "Defining the domain in the physical machine..."
      unless @infrastructure.define_domain(domain_file_dst, pm_user, pm)
         @err.call "Impossible to define #{vm_name} domain"
      end
      
      # Start the domain
      puts "Starting the domain..."
      unless @infrastructure.start_domain(vm_name, pm_user, pm)
         @err.call "#{vm_name} impossible to start"
      end
      
      # Save the domain's name
      puts "Saving the domain's name..."
      unless @infrastructure.save_domain_name(vm_name, pm_user, pm)
         @err.call "#{vm_name} impossible to save in domains file"
      end

      # Save the new virtual machine's MAC address
      file = File.open(LAST_MAC_FILE, 'w')
      file.puts(mac_address)
      file.close
      
      # Send the new virtual machine's MAC address to all nodes
      mcc = MCollectiveFilesClient.new("files")
      mcc.create_file(LAST_MAC_FILE, mac_address)
      #mcc.disconnect

   end


   # Monitors a virtual machine.
   # Returns false if the machine is not alive.
   def monitor_vm(vm, ip_roles, monitor_function)

      # Check if it is alive
      alive = CloudMonitor.ping(vm)
      unless alive
         @err.call "#{vm} is not alive. Impossible to monitor"
         
         # If it is not alive there is no point in continuing
         return false
      end
      
      # Get user and password
      user = @resource[:vm_user]
      password = @resource[:root_password]
      
      # Send it your ssh key
      # Your key was created when you turned into leader
      puts "Sending ssh key to #{vm}"
      out, success = CloudSSH.copy_ssh_key(user, vm, password)
      if success
         puts "ssh key sent"
      else
         @err.call "ssh key impossible to send"
      end
      
      # Check if MCollective is installed and configured
      mcollective_installed = CloudMonitor.mcollective_installed(user, vm)
      unless mcollective_installed
         @err.call "MCollective is not installed on #{vm}"
      end
      
      # Make sure MCollective is running.
      # We need this to ensure the leader election, so ensuring MCollective
      # is running can not be left to Puppet in their local manifest. It must be
      # done explicitly here and now.
      mcollective_running = CloudMonitor.mcollective_running(user, vm)
      unless mcollective_running
         @err.call "MCollective is not running on #{vm}"
      end
      
      # A node may have different roles
      vm_roles = get_vm_roles(ip_roles, vm)
      
      # Check if they have their ID
      # If they are running, but they do not have their ID:
      #   - Set their ID before they can become another leader.
      #   - Set also the leader's ID.
      success = CloudLeader.vm_check_id(user, vm)
      unless success
      
         # Set their ID (based on the last ID we defined)
         id = @leader.last_id
         id += 1
         CloudLeader.vm_set_id(user, vm, id)
         @leader.set_last_id(id)
         
         # Set the leader's ID
         leader = @leader.leader
         CloudLeader.vm_set_leader(user, vm, leader)
         
         # Send the last ID to all nodes
         mcc = MCollectiveFilesClient.new("files")
         mcc.create_file(CloudLeader::LAST_ID_FILE, id.to_s)
         #mcc.disconnect
      end
      
      # Copy files no matter what or check first if they have them?
      # Use copy_cloud_files if we copy no matter what. Modify it if we check
      # We should copy no matter what in case they have changed
      
      # Depending on the type of cloud we will have to monitor different components.
      # Also, take into account that one node can perform more than one role.
      print "#{vm} will perform the roles: "
      p vm_roles
      vm_roles.each do |role|
         monitor_function.call(@resource, vm, role)
      end
      
      return true
      
   end


   # Shuts down virtual machines.
   def shutdown_vms()

      # Get pool of physical machines
      pms = @resource[:pool]
      
      # Shut down virtual machines
      pms.each do |pm|
      
         pm_user = @resource[:pm_user]
         pm_password = @resource[:pm_password]
         
         # Copy ssh key to physical machine
         puts "Copying ssh key to physical machine"
         out, success = CloudSSH.copy_ssh_key(pm_user, pm, pm_password)
         
         # Bring the defined domains file from the physical machine to this one
         out, success = CloudSSH.get_remote(CloudInfrastructure::DOMAINS_FILE,
                                            pm_user, pm,
                                            CloudInfrastructure::DOMAINS_FILE)
         if success
         
            puts "#{CloudInfrastructure::DOMAINS_FILE} exists in #{pm}"
            
            # Open file
            defined_domains = File.open(CloudInfrastructure::DOMAINS_FILE, 'r')
         
            # Stop nodes
            puts "Shutting down domains"
            defined_domains.each_line do |domain|
               domain.chomp!
               unless @infrastructure.shutdown_domain(domain, pm_user, pm)
                  @err.call "#{domain} impossible to shut down"
               end
            end
            
            # Undefine local domains
            puts "Undefining domains"
            defined_domains.rewind
            defined_domains.each_line do |domain|
               domain.chomp!
               unless @infrastructure.undefine_domain(domain, pm_user, pm)
                  @err.call "#{domain} impossible to undefine"
               end
            end
            
            # Delete the defined domains file on the physical machine
            puts "Deleting defined domains file"
            command = "rm -rf #{CloudInfrastructure::DOMAINS_FILE}"
            out, success = CloudSSH.execute_remote(command, pm_user, pm)
            
            # Delete all the domain files on the physical machine. Check how the
            # name is defined on 'start_vm' function.
            puts "Deleting domain files"
            domain_files = "cloud-%s-*.xml" % [@resource[:name]]
            command = "rm /tmp/#{domain_files}"
            out, success = CloudSSH.execute_remote(command, pm_user, pm)
         
         else
            # Some physical machines might not have any virtual machine defined.
            # For instance, no virtual machine will be defined if they were
            # already defined and running when we started the cloud.
            puts "No #{CloudInfrastructure::DOMAINS_FILE} file found in #{pm}"
         end
         
      end   # pms.each

   end


   #############################################################################
   # Auxiliar functions
   #############################################################################
   private

   # Gets the virtual machine's mac address.
   def get_vm_mac()
      
      if File.exists?(LAST_MAC_FILE)
         file = File.open(LAST_MAC_FILE, 'r')
         mac = MAC_Address.new(file.read().chomp())
         mac_address = mac.next_mac()
         file.close
      else
         mac = MAC_Address.new(@resource[:starting_mac_address])
         mac_address = mac.mac
      end
      
      return mac_address

   end


   # Gets the virtual machine's disk image.
   def get_vm_disk(vm, ip_roles, img_roles)
      
      # What if a machine has different roles? Pick one and that is it
      role = :undefined
      index = 0
      ip_roles.each do |r, ips|
         index_aux = 0      # Reset the index for each role
         ips.each do |ip|
            if vm == ip
               puts "vm: #{vm} == ip: #{ip}"
               role = r
               index = index_aux
               puts "role: #{role}"
            else
               index_aux += 1
            end
         end
      end
      
      puts "Finished iterating. role: #{role}, index: #{index}"
      disk = img_roles[role][index]
      
      return disk

   end


   # Gets all the roles a node has.
   def get_vm_roles(roles, vm)

      # The roles array is a map of roles - IP addresses. The 'IP addresses'
      # value can be either a single value or an array of values.
      
      vm_roles = []
      roles.each do |role, ips|
         if ips == vm
            vm_roles << role
         elsif ips.is_a?(Array) && ips.include?(vm)
            vm_roles << role
         end
      end
      return vm_roles

   end

end
\end{lstlisting}


\subsection{cloudconstants.rb}


\begin{lstlisting}
# Some constants

MY_IP = Facter.value(:ipaddress)
\end{lstlisting}


\subsection{mac.rb}


\begin{lstlisting}
# MAC address
class MAC_Address
   
   attr_reader :mac
   
   # Creates a new MAC_Address object.
   def initialize(value=nil)
      @mac = value ? value: "52:54:00:00:00:00"
   end
   
   
   # Obtains the next MAC address.
   def next_mac
   
      mac_string = @mac.delete(":")
      mac_int = mac_string.to_i(16)
      mac_int += 1
      mac_hex = mac_int.to_s(16)
      result = mac_hex[0..1] + ":" + mac_hex[2..3] + ":" + mac_hex[4..5] + ":" + 
               mac_hex[6..7] + ":" + mac_hex[8..9] + ":" + mac_hex[10..11]
      return result
      
   end
   
   
   # Generates an array of <many> MAC addresses starting from this one.
   # def generate_array(many)
   
   #    result = []
   #    result << @mac
   #    for i in 1..many
   #       mac = MAC_Address.new(result[i - 1])
   #       result << mac.next_mac
   #    end
   #    return result
   
   # end
   
   
end
\end{lstlisting}


\subsection{cloudleader.rb}


\begin{lstlisting}
# Generic leader election methods for a distributed infrastructure
class CloudLeader
   
   ID_FILE      = "/tmp/cloud-id"
   LEADER_FILE  = "/tmp/cloud-leader"
   LAST_ID_FILE = "/tmp/cloud-last-id"
   
   
   attr_reader :id, :leader, :last_id
   
   def initialize(id_file = ID_FILE, leader_file = LEADER_FILE,
                  last_id_file = LAST_ID_FILE)
   
      @id_file      = id_file
      @leader_file  = leader_file
      @last_id_file = last_id_file 
      
      @id      = init_id()
      @leader  = init_leader()
      @last_id = init_last_id()
      
   end
   
   
   #############################################################################
   private
   
   # Gets the node's ID by reading the node's id_file.
   def init_id()
      
      if File.exists?(@id_file)
         id_file = File.open(@id_file, 'r')
         id = id_file.read().chomp().to_i()
         id_file.close
      else
         id = -1
      end
      return id
   
   end
   
   
   # Gets the leader's ID by reading the node's leader_file.
   def init_leader()
   
      if File.exists?(@leader_file)
         leader_file = File.open(@leader_file, 'r')
         leader = leader_file.read().chomp().to_i()
         leader_file.close
      else
         leader = -1
      end
      return leader
   
   end
   
   
   # Gets the last defined ID in the ID file.
   def init_last_id()

      if File.exists?(@last_id_file)
         file = File.open(@last_id_file, 'r')
         id = file.read().chomp().to_i
         file.close
      else
         id = @id
      end
      return id

   end
   
   
   #############################################################################
   public
   
   # Sets the node's ID.
   def set_id(id)
   
      file = File.open(@id_file, 'w')
      file.puts(id)
      file.close
      @id = id
      
   end
   
   
   # Sets the leader's ID in the node.
   def set_leader(leader)
   
      file = File.open(@leader_file, 'w')
      file.puts(leader)
      file.close
      @leader = leader
      
   end
   
   
   # Sets last defined ID in the ID file.
   def set_last_id(id)

      file = File.open(@last_id_file, 'w')
      file.puts(id)
      file.close
      @last_id = id
      
   end
   
   
   # Checks if this node is leader.
   def leader?
      
      return @id == @leader && @id != -1
      
   end
   
   
   #############################################################################
   # Remote ID functions
   #############################################################################
   
   # Checks if the remote node has their ID file.
   def self.vm_check_id(user, vm, id_file = ID_FILE)
   
      command = "cat #{id_file} > /dev/null 2> /dev/null"
      out, success = CloudSSH.execute_remote(command, user, vm)
      return success
      
   end
   
   
   # Sets the node's ID on a remote node.
   def self.vm_set_id(user, vm, id, id_file = ID_FILE)
   
      command = "echo #{id} > #{id_file}"
      out, success = CloudSSH.execute_remote(command, user, vm)
      return success
      
   end


   # Sets the leader's ID on a remote node.
   def self.vm_set_leader(user, vm, leader, leader_file = LEADER_FILE)
   
      command = "echo #{leader} > #{leader_file}"
      out, success = CloudSSH.execute_remote(command, user, vm)
      return success
      
   end
   
end
\end{lstlisting}


\subsection{cloudssh.rb}


\begin{lstlisting}
# Generic ssh functions for a distributed infrastructure
module CloudSSH
   
   SSH_PATH = "/root/cloud/ssh"
   SSH_KEY = "id_rsa"

   # Generates a new ssh key to be used in all machines.
   def self.generate_ssh_key(path = SSH_PATH, file = SSH_KEY)
      
      puts "Creating #{path} directory..."
      result = `mkdir -p #{path}`
      unless $?.exitstatus == 0
         puts "Could not create #{path} directory"
      end
      
      puts "Deleting previous keys..."
      result = `rm -rf #{path}/*`
      unless $?.exitstatus == 0
         puts "Could not create #{path}/#{file} key"
      end
      
      puts "Generating key..."
      result = `ssh-keygen -t rsa -N '' -f #{path}/#{file}`
      unless $?.exitstatus == 0
         puts "Could not create #{path}/#{file} key"
      end
      
      puts "Evaluating agent and adding identity..."
      
      # Must be done in one command
      result = `eval \`ssh-agent\` ; ssh-add #{path}/id_rsa`
      unless $?.exitstatus == 0
         puts "Could not add #{path}/#{file} key"
      end

   end
   
   
   # Copies an ssh key to a machine.
   def self.copy_ssh_key(user, ip, password, path = SSH_PATH, file = SSH_KEY)
   
      command_path = "/etc/puppet/modules/generic-module/provider/"
      identity_file = "#{path}/#{file}.pub"     # Be careful, copy PUBLIC KEY
      if password != ""
         puts "password is not empty, using ssh_copy_id.sh shell script"
         result = `#{command_path}/ssh_copy_id.sh #{user}@#{ip} #{identity_file} #{password}`
         success = $?.exitstatus == 0
      else
         puts "password is empty, using ssh-copy-id command"
         result = `ssh-copy-id -i #{identity_file} #{user}@#{ip}`
         success = $?.exitstatus == 0
      end
      return result, success
   end
   
   
   # Executes a command on a remote machine.
   def self.execute_remote(command, user, ip, path = SSH_PATH, file = SSH_KEY)
   
      result = `ssh #{user}@#{ip} -i #{path}/#{file} '#{command}'`
      exit_code = $?.exitstatus
      success = (exit_code == 0)
      return result, success, exit_code
   end
   
   
   # Copies a file to a remote machine.
   def self.copy_remote(src_file, dst_ip, dst_file, dst_user = "root",
                        path = SSH_PATH, file = SSH_KEY)
   
      result = `scp -i #{path}/#{file} #{src_file} #{dst_user}@#{dst_ip}:#{dst_file}`
      success = $?.exitstatus == 0
      return result, success
   end
   
   
   # Gets a file from a remote machine.
   def self.get_remote(src_file, src_user, src_ip, dst_file,
                       path = SSH_PATH, file = SSH_KEY)
   
      result = `scp -i #{path}/#{file} #{src_user}@#{src_ip}:#{src_file} #{dst_file}`
      success = $?.exitstatus == 0
      return result, success
   
   end
end
\end{lstlisting}


\subsection{ssh\_copy\_id.sh}


\begin{lstlisting}
#!/usr/bin/env expect

#############################
#
# Author : Kowshik Prakasam
#
# An expect script to automatically login to each host as a particular user and 
# install the user's public key using ssh-copy-id
#
# If successful, exits with zero (0) status 
# If unsuccessful, exits with non-zero status
#
# Usage : sshcopy.exp <user@host> <key_file> <password>
# Example : sshcopy.exp root@128.111.55.234 /home/appscale/.appscale/appscale joe
#
#############################

# Procedure to interact with ssh-copy-id command
# Parameter : password 
proc sshcopyid { password } {
  expect {
    # Send password at 'Password' prompt and tell expect to continue(i.e. exp_continue)
    -re "\[P|p]assword:" { exp_send "$password\r"
                           exp_continue }

    #Returning 1 as ssh-copy-id has failed
    -re "^\[P|p]ermission denied*" { return 1}
                           
    #Answering yes to ssh host 
    -nocase  "are you sure you want to continue connecting (yes/no)?" { exp_send "yes\r"
    	    	      	       	   	   	    	       		    exp_continue }
            
    # Tell expect stay in this 'expect' block and for each character that ssh-copy-id prints while doing the copy
    # Reset the timeout counter back to 0
    -re .                { exp_continue  }
    timeout              { return 1      }
    
    #Returning 0 as ssh-copy-id was successful
    eof                  { return 0      }
  }
}

#Parsing command-line arguments
set host [lrange $argv 0 0]
set key_file [lrange $argv 1 1]
set password [lrange $argv 2 2]

#Setting timeout to an arbitrary value of 3 that works well for ssh-copy-id
set timeout 3

# Execute ssh-copy-id command
eval spawn ssh-copy-id -i $key_file $host

#Get the result of ssh-copy-id
set sshcopyid_result [sshcopyid $password]

# If ssh-copy-id was successful
if { $sshcopyid_result == 0 } {
  #Exit with zero status
  exit 0
}

# Error attempting ssh-copy-id, so exit with non-zero status
exit 1
\end{lstlisting}


\section{appscale}
\subsection{appscale.rb}


\begin{lstlisting}
Puppet::Type.newtype(:appscale) do
   @doc = "Manages AppScale clouds formed by KVM virtual machines."

   
   ensurable do

      desc "The cloud's ensure field can assume one of the following values:
   `running`: The cloud is running.
   `stopped`: The cloud is stopped.\n"
   
      newvalue(:stopped) do
         provider.stop
      end

      newvalue(:running) do
         provider.start
      end

   end


   # General parameters
   
   newparam(:name) do
      desc "The cloud name"
      isnamevar
   end

   newparam(:ip_file) do
      desc "The file with the cloud description in YAML format"
   end

   newparam(:img_file) do
      desc "The file containing the qemu image(s). You must either provide " +
           "one image from which all copies shall be made or provide " +
           "an image for every instance"
   end

   newparam(:vm_domain) do
      desc "The XML file with the virtual machine domain definition. " +
           "Libvirt XML format must be used"
   end

   newproperty(:pool, :array_matching => :all) do
      desc "The pool of physical machines"
   end

   
   # Virtual machine parameters
   newparam(:vm_mem) do
      desc "The virtual machine's maximum amopunt of memory. " + 
           "In KiB: 2**10 (or blocks of 1024 bytes)."
      defaultto "1048576"
   end
   
   newparam(:vm_ncpu) do
      desc "The virtual machine's number of CPUs"
      defaultto "1"
   end
   
   
   # Infrastructure parameters

   newparam(:pm_user) do
      desc "The physical machines' user. It must have proper permissions"
      defaultto "dceresuela"
   end

   newparam(:pm_password) do
      desc "The physical machines' password"
      defaultto ""
   end

   newparam(:starting_mac_address) do
      desc "Starting MAC address for new virtual machines"
      defaultto "52:54:00:01:00:00"
   end

   newparam(:vm_user) do
      desc "Virtual machines' user"
      defaultto "root"
   end

   newparam(:root_password) do
      desc "Virtual machines' root password"
      defaultto "root"
   end
   
   
   # AppScale parameters
   
   newproperty(:controller, :array_matching => :all) do
      desc "The controller node"
   end

   newproperty(:servers, :array_matching => :all) do
      desc "The server nodes"
   end


   newproperty(:master, :array_matching => :all) do
      desc "The master node"
   end

   newproperty(:appengine, :array_matching => :all) do
      desc "The appengine nodes"
   end

   newproperty(:database, :array_matching => :all) do
      desc "The database nodes"
   end

   newproperty(:login, :array_matching => :all) do
      desc "The login node"
   end

   newproperty(:open, :array_matching => :all) do
      desc "The open nodes"
   end

   newproperty(:zookeeper, :array_matching => :all) do
      desc "The zookeeper nodes"
   end

   newproperty(:memcache, :array_matching => :all) do
      desc "The memcache nodes"
   end


   newparam(:app_email) do
      desc "AppScale administrator e-mail"
      defaultto "david@gmail.com"
   end
   
   newparam(:app_password) do
      desc "AppScale administrator password"
      defaultto "appscale"
   end
   
end
\end{lstlisting}


\subsection{appscalep.rb}


\begin{lstlisting}
Puppet::Type.type(:appscale).provide(:appscalep) do
   desc "Manages AppScale clouds formed by KVM virtual machines"

   # Require appscale auxiliar files
   require File.dirname(__FILE__) + '/appscale/appscale_yaml.rb'
   require File.dirname(__FILE__) + '/appscale/appscale_parsing.rb'
   require File.dirname(__FILE__) + '/appscale/appscale_functions.rb'
   
   # Require generic files
   require '/etc/puppet/modules/generic-module/provider/mcollective_client.rb'
   Dir["/etc/puppet/modules/generic-module/provider/*.rb"].each { |file| require file }
   
   # Commands needed to make the provider suitable
   commands :ping => "/bin/ping"
   commands :grep => "/bin/grep"
   commands :ps   => "/bin/ps"
   
   # Operating system restrictions
   confine :osfamily => "Debian"

   # Some constants
   #   They are in the generic cloud files

   # Makes sure the cloud is running.
   def start

      cloud = Cloud.new(CloudInfrastructure.new(), CloudLeader.new(), resource,
                        method(:err))
      puts "Starting cloud %s" % [resource[:name]]
      
      # Check existence
      if !exists?
         # Cloud does not exist => Startup operations
         
         # Check pool of physical machines
         puts "Checking pool of physical machines..."
         pm_up, pm_down = cloud.check_pool()
         unless pm_down.empty?
            puts "Some physical machines are down"
            pm_down.each do |pm|
               puts " - #{pm}"
            end
         end
         
         # Obtain the virtual machines' IPs
         puts "Obtaining the virtual machines' IPs..."
         #vm_ips, vm_ip_roles, vm_img_roles = obtain_vm_data(method(:appscale_yaml_ips),
         #                                                   method(:appscale_yaml_images))
         vm_ips, vm_ip_roles, vm_img_roles = obtain_vm_data(cloud.resource)
         
         # Check whether you are one of the virtual machines
         puts "Checking whether this machine is part of the cloud..."
         part_of_cloud = vm_ips.include?(MY_IP)
         if part_of_cloud
            puts "#{MY_IP} is part of the cloud"
            
            # Check if you are the leader
            if cloud.leader?()
               cloud.leader_start("appscale", vm_ips, vm_ip_roles, vm_img_roles,
                                  pm_up, method(:appscale_monitor))
            else
               cloud.common_start()
            end
         else
            puts "#{MY_IP} is not part of the cloud"
            cloud.not_cloud_start("appscale", vm_ips, vm_ip_roles, vm_img_roles,
                                  pm_up)
         end
         
      else
         
         # Cloud exists => Monitoring operations
         puts "Cloud already started"
         
         # Check if you are the leader
         if cloud.leader?()
            cloud.leader_monitoring(method(:appscale_monitor))
         else
            puts "#{MY_IP} is not the leader"      # Nothing to do
         end
      end
      
   end


   # Makes sure the cloud is not running.
   def stop

      cloud = Cloud.new(CloudInfrastructure.new(), CloudLeader.new(), resource,
                        method(:err))
      puts "Stopping cloud %s" % [resource[:name]]

      if cloud.leader?()
         if !exists?
            err "Cloud does not exist"
            return
         end
         if status != :running
            err "Cloud is not running"
            return
         end
         if exists? && status == :running
            
            puts "It is an appscale cloud"
            
            # Here we cannot use leader_stop function because appscale_cloud_stop
            # has a different type of arguments. So instead we will use
            # stop_cron_jobs, shutdown_vms and delete_files directly.
            
            # Stop cron jobs on all machiness
            cloud.stop_cron_jobs("appscale")
            
            # Stop cloud infrastructure
            appscale_cloud_stop(cloud.resource, MY_IP)      # What if we run
            # stop on a different machine than start?

            # Shutdown and undefine all virtual machines explicitly created for this cloud
            cloud.shutdown_vms()
            
            # Delete files
            cloud.delete_files()
            
            # Note: As all the files deleted so far are located in the /tmp directory
            # only the machines that are still alive need to delete these files.
            # If the machine was shut down, these files will not be there the next
            # time it is started, so there is no need to delete them.
            
            puts "==================="
            puts "== Cloud stopped =="
            puts "==================="
         end
      else
         puts "#{MY_IP} is not the leader"      # Nothing to do
      end
   
   end


   def status
      if File.exists?("/tmp/cloud-#{resource[:name]}")
         return :running
      else
         return :stopped
      end
   end


   # Ensure methods
   #def create
   #   return true
   #end
   

   #def destroy
   #   return true
   #end


   def exists?
      return File.exists?("/tmp/cloud-#{resource[:name]}")
   end
   
   
   #############################################################################
   # Properties need methods
   #############################################################################
   def pool
   end
   
   def pm_user
   end
   
   def starting_mac_address
   end
   
   def root_password
   end
   
   # AppScale
   
   def controller
   end

   def servers
   end


   def master
   end

   def appengine
   end

   def database
   end

   def login
   end

   def open
   end

   def zookeeper
   end

   def memcache
   end
   
   
   def app_email
   end
   
   def app_password
   end
   
   
end
\end{lstlisting}


\subsection{appscalep\_helper.rb}


\begin{lstlisting}
################################################################################
# Auxiliar functions for appscale provider
################################################################################

# The functions in this file are defined the same in all providers, but each
# one implements them in their own way. Thus, the headers cannot be modified.

# Starts an AppScale cloud formed by <vm_ips> performing <vm_ip_roles>
def start_cloud(resource, vm_ips, vm_ip_roles)

   puts "Starting the cloud"
   if (resource[:app_email] == nil) || (resource[:app_password] == nil)
      err "Need an AppScale user and password"
      exit
   else
      puts "app_email = #{resource[:app_email]}"
      puts "app_password = #{resource[:app_password]}"
   end
   puts  "Starting an appscale cloud"
   
   # Start appscale cloud
   return appscale_cloud_start(resource, vm_ips, vm_ip_roles,
                               resource[:app_email], resource[:app_password],
                               resource[:root_password])

end


# Obtains vm data from manifest parameters.
def obtain_vm_data(resource)

   # Default deployment
   if resource[:controller] != nil && resource[:servers] != nil

      puts "Default deployment"
      return obtain_appscale_data_default(resource[:controller],
                                          resource[:servers])

   # Custom deployment
   elsif resource[:master] != nil && resource[:appengine] != nil &&
         resource[:database] != nil && resource[:login] != nil &&
         resource[:open] != nil

      puts "Custom deployment"
      return obtain_appscale_data_custom(resource[:master],
                                         resource[:appengine],
                                         resource[:database],
                                         resource[:login],
                                         resource[:open],
                                         resource[:zookeeper],
                                         resource[:memcache])

   end

#   ips, ip_roles = appscale_yaml_ips(resource[:ip_file])
#   img_roles     = appscale_yaml_ips(resource[:img_file])
#   return ips, ip_roles, img_roles
   
end
\end{lstlisting}


\subsection{appscale-run-instances.tcl}


\begin{lstlisting}
#!/usr/bin/env expect

# Description:
#   Interacts with the appscale-run-instances tool
#
# Synopsis:
#   appscale-run-instances.tcl <file> <e-mail> <password>
#
# Arguments:
#   - File: AppScale YAML configuration file.
#   - e-mail: AppScale administration e-mail.
#   - Password: AppScale administration password.
#
# Examples:
#   _$: appscale-run-instances.tcl ips.yaml user@mail.com appscale
#
#
# Author:
#   David Ceresuela


# Procedure to interact with appscale-run-instances command
# Parameter : user
# Parameter : password 
proc runinstances { user password } {
  expect {
    # Send e-mail address
    -re "e-mail address:" { exp_send "$user\r"
                            exp_continue }

    # Send password
    -re "new password:" { exp_send "$password\r"
                          exp_continue }
    
    # Send password again to verify
    -re "again to verify:" { exp_send "$password\r"
                             exp_continue }
    
            
    # Tell expect stay in this 'expect' block and for each character that
    # appscale-run-instances prints while doing the copy
    # Reset the timeout counter back to 0
    -re .                { exp_continue  }
    timeout              { return 1      }
    
    # Returning 0 as appscale-run-instances was successful
    eof                  { return 0      }
  }
}

#Parsing command-line arguments
set yaml     [lrange $argv 0 0]
set user     [lrange $argv 1 1]
set password [lrange $argv 2 2]

#Setting timeout to an arbitrary value of 120 that works well for appscale-run-instances
set timeout 120

# Execute appscale-run-instances command
eval spawn /usr/local/appscale-tools/bin/appscale-run-instances --ips $yaml

#Get the result of appscale-run-instances
set runinstances_result [runinstances $user $password]

# If appscale-run-instances was successful
if { $runinstances_result == 0 } {
  #Exit with zero status
  exit 0
}

# Error attempting appscale-run-instances, so exit with non-zero status
exit 1
\end{lstlisting}


\subsection{appscale\_yaml.rb}


\begin{lstlisting}
require 'yaml'

##
# Obtains the IP addresses from the ip_file file. It does NOT check whether
# the file has the proper format.
# Different roles obtained from AppScale wiki:
#    http://code.google.com/p/appscale/wiki/Placement_Support

def appscale_yaml_ips(path)

   ips = []
   ip_roles = {}
   
   file = File.open(path)
   tree = YAML::parse(file)
   
   if tree != nil
   
      tree = tree.transform
      
      # Default deployment (from appscale-tools/lib/node_layout.rb)
      controller = tree[:controller]
      servers =    tree[:servers]
      
      ip_roles[:controller] = get_elements(controller)
      ip_roles[:servers]    = get_elements(servers)
      
      ips = ips + ip_roles[:controller]
      ips = ips + ip_roles[:servers]
      
      # Custom deployment (from appscale-tools/lib/node_layout.rb)
      master =    tree[:master]
      appengine = tree[:appengine]
      database =  tree[:database]
      login =     tree[:login]
      open =      tree[:open]
      zookeeper = tree[:zookeeper]
      memcache =  tree[:memcache]
      
      ip_roles[:master]    = get_elements(master)
      ip_roles[:appengine] = get_elements(appengine)
      ip_roles[:database]  = get_elements(database)
      ip_roles[:login]     = get_elements(login)
      ip_roles[:open]      = get_elements(open)
      ip_roles[:zookeeper] = get_elements(zookeeper)
      ip_roles[:memcache]  = get_elements(memcache)
      
      ips = ips + ip_roles[:master]
      ips = ips + ip_roles[:appengine]
      ips = ips + ip_roles[:database]
      ips = ips + ip_roles[:login]
      ips = ips + ip_roles[:open]
      ips = ips + ip_roles[:zookeeper]
      ips = ips + ip_roles[:memcache]
      
      ips = ips.uniq
      
      # Delete all the roles that have no IP address associated
      ip_roles.delete_if{ |role, ips| ips == [] }
      
      file.close
      
      return ips, ip_roles
   end
   
end


# Obtains the disk images from the img_file file.
def appscale_yaml_images(path)

   img_roles = {}

   file = File.open(path)
   tree = YAML::parse(file)
   
   if tree != nil

      tree = tree.transform

      # Default deployment (from appscale-tools/lib/node_layout.rb)
      controller = tree[:controller]
      servers    = tree[:servers]
      
      # Custom deployment (from appscale-tools/lib/node_layout.rb)
      master    = tree[:master]
      appengine = tree[:appengine]
      database  = tree[:database]
      login     = tree[:login]
      open      = tree[:open]
      zookeeper = tree[:zookeeper]
      memcache  = tree[:memcache]
      
      # Maybe we have been given only an image for all virtual machines
      all = tree[:all]
      
      if all == nil

         # Default deployment
         img_roles[:controller] = get_elements(controller)
         img_roles[:servers]    = get_elements(servers)
         
         # Custom deployment
         img_roles[:master]    = get_elements(master)
         img_roles[:appengine] = get_elements(appengine)
         img_roles[:database]  = get_elements(database)
         img_roles[:login]     = get_elements(login)
         img_roles[:open]      = get_elements(open)
         img_roles[:zookeeper] = get_elements(zookeeper)
         img_roles[:memcache]  = get_elements(memcache)
         
         # Delete all the roles that have no disk image associated
         img_roles.delete_if{ |role, img| img == [] }
         
      else
         img_roles[:all] = get_elements(all)
      end
      
      file.close
      
      return img_roles
   end
   
end


# Writes a hash in a file using the YAML format.
# The hash should resemble something like
# {:controller => ["192.168.1.1"], :servers => ["192.168.1.2", "192.168.1.3"]}
# if you are writing the default deployment IP addresses.
# The custom deployment is done in a similar fashion.
# Either case, the key-value pairs must follow the {symbol => array} pattern.
def appscale_write_yaml_file(hash, path)

   # Get and clean the hash
   hash_yaml = hash.to_yaml(:Indent => 0).to_s
   hash_yaml = hash_yaml.gsub("!ruby/symbol ", ":")
   hash_yaml = hash_yaml.gsub("!ruby/sym ", ":")
   
   # Write to file
   file = File.open(path, 'w')
   file.write(hash_yaml)
   file.close

end


################################################################################
# Auxiliar functions
################################################################################

# Transforms the given elements to an array.
def get_elements(array)

   elements = []
   if array != nil
      elements = array.to_a
   end
   return elements

end
\end{lstlisting}


\subsection{appscale\_functions.rb}


\begin{lstlisting}
# Starts an AppScale cloud.
def appscale_cloud_start(resource, app_ips, app_roles,
                         app_email=nil, app_password=nil, root_password=nil)

   require 'expect'
   
   # Check arguments
   puts "appscale_cloud_start called with:"
   puts "  - app_email == #{app_email}"
   puts "  - app_password == #{app_password}"
   puts "  - root_password == #{root_password}"
   
   script_keys = "appscale-add-keypair.tcl"
   script_run  = "appscale-run-instances.tcl"
   #ips_yaml = resource[:ip_file]
   script_path = "/etc/puppet/modules/appscale/lib/puppet/provider/appscale/appscale"

   # Write ips.yaml file
   puts "Writing AppScale ips_yaml file"
   puts "Hash received: "
   p app_roles
   ips_yaml = "/etc/puppet/modules/appscale/files/auto-ips.yaml"
   appscale_write_yaml_file(app_roles, ips_yaml)
   puts "AppScale ips_yaml file written"

   # Add key pairs
   puts "About to add key pairs"
   result = `#{script_path}/#{script_keys} #{ips_yaml} #{root_password}`
   if $?.exitstatus == 0
      puts "Key pairs added"
      puts "result = |||#{result}|||"
   else
      err "Impossible to add key pairs"
      return false
   end
   
   # Run instances
   puts "About to run instances"
   puts "This may take a while (~ 5 min), so please be patient"
   result = `#{script_path}/#{script_run} #{ips_yaml} #{app_email} #{app_password}`
   if $?.exitstatus == 0
      puts "Instances running"
      puts "result = |||#{result}|||"
   else
      err "Impossible to run appscale instances"
      return false
   end
   
   # Start monitoring
   puts "Start monitoring"
   app_ips.each do |vm|

      # Get all vm's roles
      roles = app_roles.select { |r, ips| ips.include?(vm) }      # Be careful,
      # Ruby 1.8.7 returns an array instead of a hash, so we get something like
      # [[:appengine, [2, 3, 4]], [:database, [3, 4]]] which is an array of
      # arrays with the role in the first place of the innermost arrays.
      
      # Monitor every one of them
      roles.each do |role_array|
         role = role_array[0]
         puts "Calling appscale_monitor on #{vm} as #{role}"
         appscale_monitor(resource, vm, role)
      end
   end

   return true
   
end


# Stops an AppScale cloud.
def appscale_cloud_stop(resource, vm)
   
   user = resource[:vm_user]
   
   # It is being monitored with puppet through a crontab file, so we have to
   # stop monitoring first
   cloud_cron = CloudCron.new()
   word = "basic"
   cloud_cron.delete_line_with_word(word, user, vm)
   
   # Terminate instances
   command = "/root/appscale-tools/bin/appscale-terminate-instances"
   CloudSSH.execute_remote(command, user, vm)
   
end


# Monitors a virtual machine belonging to an AppScale cloud.
def appscale_monitor(resource, vm, role)

   command = "ps aux"
   user = resource[:vm_user]
   out, success = CloudSSH.execute_remote(command, user, vm)
   unless success
      err "[AppScale monitor] Impossible to execute #{command} in #{vm}"
      return
   end
   
   # AppMonitoring calls god, so look for god processes who look like this:
   # /usr/bin/ruby1.8 /usr/bin/god -c /root/appscale/AppMonitoring/config/global.god
   if out.include? "/usr/bin/god -c /root/appscale/AppMonitoring"
   
      # AppMonitoring is running
      puts "[AppScale monitor] AppMonitoring is running"

      # Check the appscale controller is running
      if out.include? "AppController/djinnServer.rb"
         
         # Everything looks right so let AppScale handle it
         puts "[AppScale monitor] AppController is running"
      
      else
         
         # If AppMonitoring is running but the AppController is not,
         # AppMonitoring will take care
         puts "[AppScale monitor] AppMonitoring will take care of AppController"
         
      end

   else
   
      # AppMonitoring is not running
      puts "[AppScale monitor] AppMonitoring is not running"
   
      if out.include? "AppController/djinnServer.rb"

         # AppController is running
         puts "[AppScale monitor] AppController is running"
         
      end
      
      # Try to start the AppMonitoring
      puts "[AppScale monitor] Starting AppMonitoring on #{vm}"
      command = "/etc/init.d/appscale-monitoring start"
      out, success = CloudSSH.execute_remote(command, user, vm)
      if success
         puts "[AppScale monitor] Successfully started AppMonitoring"
      else
         err "[AppScale monitor] Impossible to start AppMonitoring in #{vm}"
         return
      end
      
      # Since the AppMonitoring will take care of the AppController, we do
      # not have to start the AppController
      
      
   end
   
   # If we have made it so far, let's try to apply the manifest
   
   # Copy the manifest
   puts "Copying manifest"
   path = "/etc/puppet/modules/appscale/files/appscale-manifests/basic.pp"
   out, success = CloudSSH.copy_remote(path, vm, "/tmp")
   unless success
      err "[AppScale monitor] Impossible to copy basic manifest to #{vm}"
      return
   end
   
#   # Apply the manifest
#   puts "Applying manifest"
#   command = "puppet apply /tmp/basic.pp"
#   out, success = CloudSSH.execute_remote(command, user, vm)
#   unless success
#      err "[AppScale monitor] Impossible to run puppet in #{vm}"
#      return
#   end
#   
#   # Analyze the output
#   if out.include? "should be directory (noop)"
#      err "[AppScale monitor] Missing directory in #{vm}"
#      puts out
#      return
#   end

   # Monitor AppScale with puppet: basic manifest
   cloud_cron = CloudCron.new()
   cron_time = "*/1 * * * *"
   cron_command = "puppet apply /tmp/basic.pp"
   cron_out = "/root/appscale.out"
   cron_err = "/root/appscale.err"
   line = cloud_cron.create_line(cron_time, cron_command, cron_out, cron_err)
   unless cloud_cron.add_line(line, user, vm)
      err "[AppScale monitor] Impossible to put basic.pp in crontab in #{vm}"
      return
   end
   
   # Execute crontab
   command = "crontab /var/spool/cron/crontabs/root"
   out, success = CloudSSH.execute_remote(command, user, vm)
   if success
      puts "[AppScale monitor] Executed crontab in #{vm}"
   else
      err "[AppScale monitor] Impossible to execute crontab in #{vm}"
      return
   end
  
end


################################################################################
# Auxiliar functions
################################################################################

def err(text)
   puts "\e[31m#{text}\e[0m"
end

\end{lstlisting}


\subsection{appscale\_parsing.rb}


\begin{lstlisting}
# Appscale parsing functions to obtain virtual machine's data from manifest's
# arguments.
# All different roles have been obtained from the node_layout file located at
# appscale-tools/lib/node_layout.rb
# There will be at most 1 controller, 1 master, and 1 login node.
# There will be at most N servers, appengine, database, open, zookeeper and
# memcache nodes.

################################################################################
# Default deployment
################################################################################

# Obtains the IP addresses and disk images from the resource[:controller],
# and resource[:servers] arguments.
# appscale { 'myappscale'
#    controller => ["155.210.155.73",
#                   "/var/tmp/dceresuela/lucid-appscale-tr1.img"],
#    servers    => ["/etc/puppet/modules/appscale/files/servers-ips.txt",
#                   "/etc/puppet/modules/appscale/files/servers-imgs.txt"],
#    pool       => ["155.210.155.70"],
#    ensure     => running,
# }
def obtain_appscale_data_default(controller, servers)

   ips, ip_roles = appscale_parse_ips_default(controller, servers)
   img_roles     = appscale_parse_images_default(controller, servers)
   
   return ips, ip_roles, img_roles

end


# Obtains the IP addresses from the resource[:controller], resource[:server] and
# resource[:database] arguments.
# appscale { 'myappscale'
#    controller => ["155.210.155.73",
#                   "/var/tmp/dceresuela/lucid-appscale-tr1.img"],
#    servers    => ["/etc/puppet/modules/appscale/files/servers-ips.txt",
#                   "/etc/puppet/modules/appscale/files/servers-imgs.txt"],
#    pool       => ["155.210.155.70"],
#    ensure     => running,
# }
def appscale_parse_ips_default(controller, servers)

   ips_index = 0     # Because the IPs are in the first component of the
                     # controller and server arrays, and that is array[0]
   ips = []
   ip_roles = {}

   
   # Get the IPs that are under the "controller" and "servers" attributes
   ip_roles[:controller] = Array[controller[ips_index].chomp]
   ip_roles[:servers]    = get_from_file(servers[ips_index])
   
   # Add the IPs to the array
   ips = ips + ip_roles[:controller]
   ips = ips + ip_roles[:servers]
   
   ips = ips.uniq
   
   return ips, ip_roles
   
end


# Obtains the disk images from the resource[:controller], and resource[:servers]
# arguments.
# appscale { 'myappscale'
#    controller => ["155.210.155.73",
#                   "/var/tmp/dceresuela/lucid-appscale-tr1.img"],
#    servers    => ["/etc/puppet/modules/appscale/files/servers-ips.txt",
#                   "/etc/puppet/modules/appscale/files/servers-imgs.txt"],
#    pool       => ["155.210.155.70"],
#    ensure     => running,
# }
def appscale_parse_images_default(controller, servers)

   img_index = 1     # Because the images are in the second component of the
                     # controller and server arrays, and that is array[1]
   img_roles = {}

   
   # Get the disk images that are under the "controller" and "servers"
   # attributes
   img_roles[:controller] = Array[controller[img_index].chomp]
   img_roles[:servers]    = get_from_file(servers[img_index])
   
   return img_roles
   
end


################################################################################
# Custom deployment
################################################################################

# Obtains the IP addresses and disk images from the resource[:master],
# resource[:appengine], resource[:database], etc. arguments.
# appscale { 'myappscale'
#    master    => ["155.210.155.73",
#                  "/var/tmp/dceresuela/lucid-appscale-tr1.img"],
#    appengine => ["/etc/puppet/modules/appscale/files/appengine-ips.txt",
#                  "/etc/puppet/modules/appscale/files/appengine-imgs.txt"],
#    database  => ["/etc/puppet/modules/appscale/files/database-ips.txt",
#                  "/etc/puppet/modules/appscale/files/database-imgs.txt"],
#    login     => ["155.210.155.73",
#                  "/var/tmp/dceresuela/lucid-appscale-tr1.img"],
#    open      => ["/etc/puppet/modules/appscale/files/open-ips.txt",
#                  "/etc/puppet/modules/appscale/files/open-imgs.txt"],
#    pool      => ["155.210.155.70"],
#    ensure    => running,
# }
def obtain_appscale_data_custom(master, appengine, database, login, open,
                                zookeeper, memcache)

   puts "Parsing custom AppScale IPs"
   ips, ip_roles = appscale_parse_ips_custom(master, appengine, database, login,
                      open, zookeeper, memcache)
   puts "Parsing custom AppScale imgs"
   img_roles     = appscale_parse_images_custom(master, appengine, database,
                      login, open, zookeeper, memcache)
   
   return ips, ip_roles, img_roles

end


# Obtains the IP addresses from the resource[:master], resource[:appengine],
# resource[:database], etc. arguments.
# appscale { 'myappscale'
#    ...
#    master    => ["155.210.155.73",
#                  "/var/tmp/dceresuela/lucid-appscale-tr1.img"],
#    appengine => ["/etc/puppet/modules/appscale/files/appengine-ips.txt",
#                  "/etc/puppet/modules/appscale/files/appengine-imgs.txt"],
#    database  => ["/etc/puppet/modules/appscale/files/database-ips.txt",
#                  "/etc/puppet/modules/appscale/files/database-imgs.txt"],
#    login     => ["155.210.155.73",
#                  "/var/tmp/dceresuela/lucid-appscale-tr1.img"],
#    open      => ["/etc/puppet/modules/appscale/files/open-ips.txt",
#                  "/etc/puppet/modules/appscale/files/open-imgs.txt"],
#    ...
# }
def appscale_parse_ips_custom(master, appengine, database, login, open,
                              zookeeper, memcache)

   ips_index = 0     # Because the IPs are in the first component of the
                     # controller and server arrays, and that is array[0]
   ips = []
   ip_roles = {}

   
   # Get the IPs that are under the "master", "appengine", "database", "login",
   # "open", "zookeeper" and "memcache" attributes
   ip_roles[:master]    = Array[master[ips_index].chomp]      unless master.nil?
   ip_roles[:appengine] = get_from_file(appengine[ips_index]) unless appengine.nil?
   ip_roles[:database]  = get_from_file(database[ips_index])  unless database.nil?
   ip_roles[:login]     = Array[login[ips_index].chomp]       unless login.nil?
   ip_roles[:open]      = get_from_file(open[ips_index])      unless open.nil?
   ip_roles[:zookeeper] = get_from_file(zookeeper[ips_index]) unless zookeeper.nil?
   ip_roles[:memcache]  = get_from_file(memcache[ips_index])  unless memcache.nil?

   # Add the IPs to the array
   ips = ips + (ip_roles[:master]    || []) # If it was nil, add the empty array
   ips = ips + (ip_roles[:appengine] || []) # (add nothing). If it was not nil,
   ips = ips + (ip_roles[:database]  || []) # add the IP addresses.
   ips = ips + (ip_roles[:login]     || [])
   ips = ips + (ip_roles[:open]      || [])
   ips = ips + (ip_roles[:zookeeper] || [])
   ips = ips + (ip_roles[:memcache]  || [])
   
   ips = ips.uniq
   
   return ips, ip_roles
   
end


# Obtains the disk images from the resource[:master], resource[:appengine],
# resource[:database], etc. arguments.
# appscale { 'myappscale'
#    ...
#    master    => ["155.210.155.73",
#                  "/var/tmp/dceresuela/lucid-appscale-tr1.img"],
#    appengine => ["/etc/puppet/modules/appscale/files/appengine-ips.txt",
#                  "/etc/puppet/modules/appscale/files/appengine-imgs.txt"],
#    database  => ["/etc/puppet/modules/appscale/files/database-ips.txt",
#                  "/etc/puppet/modules/appscale/files/database-imgs.txt"],
#    login     => ["155.210.155.73",
#                  "/var/tmp/dceresuela/lucid-appscale-tr1.img"],
#    open      => ["/etc/puppet/modules/appscale/files/open-ips.txt",
#                  "/etc/puppet/modules/appscale/files/open-imgs.txt"],
#    ...
# }
def appscale_parse_images_custom(master, appengine, database, login, open,
                                 zookeeper, memcache)

   img_index = 1     # Because the images are in the second component of the
                     # controller and server arrays, and that is array[1]
   img_roles = {}

   
   # Get the disk images that are under the "master", "appengine", "database",
   # "login", "open", "zookeeper" and "memcache" attributes
   img_roles[:master]    = Array[master[img_index].chomp]      unless master.nil?
   img_roles[:appengine] = get_from_file(appengine[img_index]) unless appengine.nil?
   img_roles[:database]  = get_from_file(database[img_index])  unless database.nil?
   img_roles[:login]     = Array[login[img_index].chomp]       unless login.nil?
   img_roles[:open]      = get_from_file(open[img_index])      unless open.nil?
   img_roles[:zookeeper] = get_from_file(zookeeper[img_index]) unless zookeeper.nil?
   img_roles[:memcache]  = get_from_file(memcache[img_index])  unless memcache.nil?

   return img_roles
   
end


################################################################################
# Auxiliar functions
################################################################################

# Gets all the file lines in an array.
def get_from_file(path)

   array = []
   file = File.open(path)
   if file != nil
      array = file.readlines.map(&:chomp)    # Discard the final '\n'
      file.close
   end

   return array

end
\end{lstlisting}


\subsection{appscale-add-keypair.tcl}


\begin{lstlisting}
#!/usr/bin/env expect

# Description:
#   Interacts with the appscale-add-keypair tool
#
# Synopsis:
#   appscale-add-keypair.tcl <password>
#
# Arguments:
#   - File: AppScale YAML configuration file.
#   - Password: Root password for all machines.
#
# Examples:
#   _$: appscale-add-keypair.tcl ips.yaml my_password_is_abcd
#
#
# Author:
#   David Ceresuela


# Procedure to interact with appscale-add-keypair command
# Parameter : password 
proc addkeypair { password } {
  expect {
    # Send password
    -re "SSH password of root:" { exp_send "$password\r"
                                  exp_continue }
    
    # Tell expect stay in this 'expect' block and for each character that
    # appscale-add-keypair prints while doing the copy
    # Reset the timeout counter back to 0
    -re .                { exp_continue  }
    timeout              { return 1      }
    
    # Returning 0 as appscale-add-keypair was successful
    eof                  { return 0      }
  }
}

#Parsing command-line arguments
set yaml     [lrange $argv 0 0]
set password [lrange $argv 1 1]

#Setting timeout to an arbitrary value of 120 that works well for appscale-add-keypair
set timeout 120

# Execute appscale-add-keypair command
eval spawn /usr/local/appscale-tools/bin/appscale-add-keypair --ips $yaml --auto

#Get the result of appscale-add-keypair
set addkeypair_result [addkeypair $password]

# If appscale-add-keypair was successful
if { $addkeypair_result == 0 } {
  #Exit with zero status
  exit 0
}

# Error attempting appscale-add-keypair, so exit with non-zero status
exit 1
\end{lstlisting}


\subsection{manifiestos locales}
\subsubsection{basic.pp}

\begin{lstlisting}
################################################################################
# AppScale root directory

file { '/root/appscale':
   ensure => 'directory',
   noop   => 'true',
}

file { '/root/appscale/.appscale':
   ensure => 'directory',
   noop   => 'true',
}

################################################################################
# AppScale basic roles

file { '/root/appscale/AppController':
   ensure => 'directory',
   noop   => 'true',
}

file { '/root/appscale/AppDB':
   ensure => 'directory',
   noop   => 'true',
}

file { '/root/appscale/AppLoadBalancer':
   ensure => 'directory',
   noop   => 'true',
}

file { '/root/appscale/AppMonitoring':
   ensure => 'directory',
   noop   => 'true',
}

file { '/root/appscale/AppServer':
   ensure => 'directory',
   noop   => 'true',
}

file { '/root/appscale/AppServer_Java':
   ensure => 'directory',
   noop   => 'true',
}

file { '/root/appscale/Neptune':
   ensure => 'directory',
   noop   => 'true',
}

################################################################################
# AppScale programs

file { '/usr/bin/god':
   ensure => 'present',
   noop   => 'true',
}

file { '/usr/bin/mongrel_rails':
   ensure => 'present',
   noop   => 'true',
}

file { '/usr/sbin/nginx':
   ensure => 'present',
   noop   => 'true',
}
\end{lstlisting}
\section{torque}
\subsection{torque.rb}


\begin{lstlisting}
Puppet::Type.newtype(:torque) do
   @doc = "Manages Torque clouds formed by KVM virtual machines."

   
   ensurable do

      desc "The cloud's ensure field can assume one of the following values:
   `running`: The cloud is running.
   `stopped`: The cloud is stopped.\n"
   
      newvalue(:stopped) do
         provider.stop
      end

      newvalue(:running) do
         provider.start
      end

   end


   # General parameters

   newparam(:name) do
      desc "The cloud name"
      isnamevar
   end

#   newparam(:ip_file) do
#      desc "The file with the cloud description in YAML format"
#   end
#   
#   newparam(:img_file) do
#      desc "The file containing the qemu image(s). You must either provide " +
#           "one image from which all copies shall be made or provide " +
#           "an image for every instance"
#   end

   newparam(:vm_domain) do
      desc "The XML file with the virtual machine domain definition. " +
           "Libvirt XML format must be used"
   end

   newproperty(:pool, :array_matching => :all) do
      desc "The pool of physical machines"
   end
   
   
   # Virtual machine parameters
   newparam(:vm_mem) do
      desc "The virtual machine's maximum amopunt of memory. " + 
           "In KiB: 2**10 (or blocks of 1024 bytes)."
      defaultto "1048576"
   end
   
   newparam(:vm_ncpu) do
      desc "The virtual machine's number of CPUs"
      defaultto "1"
   end
   
   
   # Infrastructure parameters

   newparam(:pm_user) do
      desc "The physical machines' user. It must have proper permissions"
      defaultto "dceresuela"
   end

   newparam(:pm_password) do
      desc "The physical machines' password"
      defaultto ""
   end

   newparam(:starting_mac_address) do
      desc "Starting MAC address for new virtual machines"
      defaultto "52:54:00:01:00:00"
   end

   newparam(:vm_user) do
      desc "Virtual machines' user"
      defaultto "root"
   end

   newparam(:root_password) do
      desc "Virtual machines' root password"
      defaultto "root"
   end


   # Torque parameters
   newproperty(:head, :array_matching => :all) do
      desc "The head node's information"
   end
   
   newproperty(:compute, :array_matching => :all) do
      desc "The compute nodes' information"
   end
   
end
\end{lstlisting}


\subsection{torquep\_helper.rb}


\begin{lstlisting}
################################################################################
# Auxiliar functions for torque provider
################################################################################

# The functions in this file are defined the same in all providers, but each
# one implements them in their own way. Thus, the headers cannot be modified.

# Starts a torque cloud formed by <vm_ips> performing <vm_ip_roles>.
def start_cloud(resource, vm_ips, vm_ip_roles)

   puts "Starting the cloud"
   
   # SSH keys have already been distributed when machines were monitorized,
   # so we do not have to distribute them again
   
   # Start torque cloud
   return torque_cloud_start(resource, vm_ip_roles)

end


# Obtains vm data from manifest parameters.
def obtain_vm_data(resource)

   puts "Obtaining virtual machines' data"
   return obtain_torque_data(resource[:head], resource[:compute])
   
end
\end{lstlisting}


\subsection{torquep.rb}


\begin{lstlisting}
Puppet::Type.type(:torque).provide(:torquep) do
   desc "Manages Torque clouds formed by KVM virtual machines"

   # Require torque auxiliar files
   require File.dirname(__FILE__) + '/torque/torque_yaml.rb'
   require File.dirname(__FILE__) + '/torque/torque_functions.rb'
   require File.dirname(__FILE__) + '/torque/torque_parsing.rb'
   
   # Require generic files
   require '/etc/puppet/modules/generic-module/provider/mcollective_client.rb'
   Dir["/etc/puppet/modules/generic-module/provider/*.rb"].each { |file| require file }

   # Commands needed to make the provider suitable
   commands :ping => "/bin/ping"
   commands :grep => "/bin/grep"
   commands :ps   => "/bin/ps"
   
   # Operating system restrictions
   confine :osfamily => "Debian"

   # Some constants
   #   They are in the generic cloud files

   # Makes sure the cloud is running.
   def start

      cloud = Cloud.new(CloudInfrastructure.new(), CloudLeader.new(), resource,
                        method(:err))
      puts "Starting cloud %s" % [resource[:name]]
      
      # Check existence
      if !exists?
         # Cloud does not exist => Startup operations
         
         # Check pool of physical machines
         puts "Checking pool of physical machines..."
         pm_up, pm_down = cloud.check_pool()
         unless pm_down.empty?
            puts "Some physical machines are down"
            pm_down.each do |pm|
               puts " - #{pm}"
            end
         end
         
         # Obtain the virtual machines' IPs
         puts "Obtaining the virtual machines' IPs..."
         #vm_ips, vm_ip_roles, vm_img_roles = obtain_vm_data(method(:torque_yaml_ips),
         #                                                   method(:torque_yaml_images))
         vm_ips, vm_ip_roles, vm_img_roles = obtain_vm_data(cloud.resource)
         
         
         # Check whether you are one of the virtual machines
         puts "Checking whether this machine is part of the cloud..."
         part_of_cloud = vm_ips.include?(MY_IP)
         if part_of_cloud
            puts "#{MY_IP} is part of the cloud"
            
            # Check if you are the leader
            if cloud.leader?()
               cloud.leader_start("torque", vm_ips, vm_ip_roles, vm_img_roles,
                                  pm_up, method(:torque_monitor))
            else
               cloud.common_start()
            end
         else
            puts "#{MY_IP} is not part of the cloud"
            cloud.not_cloud_start("torque", vm_ips, vm_ip_roles, vm_img_roles,
                                  pm_up)
         end
         
      else
         
         # Cloud exists => Monitoring operations
         puts "Cloud already started"

         # Check if you are the leader
         if cloud.leader?()
            cloud.leader_monitoring(method(:torque_monitor))
         else
            puts "#{MY_IP} is not the leader"      # Nothing to do
         end
      end
      
   end


   # Makes sure the cloud is not running.
   def stop

      cloud = Cloud.new(CloudInfrastructure.new(), CloudLeader.new(), resource,
                        method(:err))
      puts "Stopping cloud %s" % [resource[:name]]
      
      if cloud.leader?()
         if !exists?
            err "Cloud does not exist"
            return
         end
         
         if status != :running
            err "Cloud is not running"
            return
         end
         
         if exists? && status == :running
            puts "It is a torque cloud"
         
            # Stop cloud infrastructure
            cloud.leader_stop("torque", method(:torque_cloud_stop))
         end
      else
         puts "#{MY_IP} is not the leader"      # Nothing to do
      end
   
   end


   def status
      if File.exists?("/tmp/cloud-#{resource[:name]}")
         return :running
      else
         return :stopped
      end
   end


   # Ensure methods
   #def create
   #   return true
   #end
   

   #def destroy
   #   return true
   #end


   def exists?
      return File.exists?("/tmp/cloud-#{resource[:name]}")
   end
   
   
   #############################################################################
   # Properties need methods
   #############################################################################
   def pool
   end
   
   def pm_user
   end
   
   def starting_mac_address
   end
   
   def root_password
   end
   
   def head
   end
   
   def compute
   end
   
end
\end{lstlisting}


\subsection{torque\_parsing.rb}


\begin{lstlisting}
# Torque parsing functions to obtain virual machine's data from manifest's
# arguments.

# Obtains the IP addresses and disk images from the resource[:head] and
# resource[:compute] arguments.
# torque { 'mytorque'
#   ...
#   head => ["155.210.155.73", "/var/tmp/master.img"],
#   compute => ["/files/compute-ip.txt", "/files/compute-img.txt"],
#   ...
# }
def obtain_torque_data(head, compute)

   ips, ip_roles = torque_parse_ips(head, compute)
   img_roles     = torque_parse_images(head, compute)
   return ips, ip_roles, img_roles

end


# Obtains the IP addresses from the resource[:head] and resource[:compute]
# arguments.
# torque { 'mytorque'
#   ...
#   head => ["155.210.155.73", "/var/tmp/master.img"],
#   compute => ["/files/compute-ip.txt", "/files/compute-img.txt"],
#   ...
# }
def torque_parse_ips(head, compute)

   ips_index = 0     # Because the IPs are in the first component of the head
                     # and compute arrays, and that is array[0]
   ips = []
   ip_roles = {}
   
   # Get the IPs that are under the "head" and "compute" attributes
   ip_roles[:head] = []
   ip_roles[:head] << head[ips_index].chomp
   
   path = compute[ips_index]
   ip_roles[:compute] = get_from_file(path)
   
   # Add the IPs to the array
   ips = ips + ip_roles[:head]
   ips = ips + ip_roles[:compute]
   
   ips = ips.uniq
   
   return ips, ip_roles
   
end


# Obtains the disk images from the resource[:head] and resource[:compute]
# arguments.
# torque { 'mytorque'
#   ...
#   head => ["155.210.155.73", "/var/tmp/master.img"],
#   compute => ["/files/compute-ip.txt", "/files/compute-img.txt"],
#   ...
# }
def torque_parse_images(head, compute)

   img_index = 1     # Because the images are in the first component of the head
                     # and compute arrays, and that is array[1]
   img_roles = {}

   
   # Get the disk images that are under the "head" and "compute" attributes
   img_roles[:head] = []
   img_roles[:head] << head[img_index].chomp
   
   path = compute[img_index]
   img_roles[:compute] = get_from_file(path)
   
   return img_roles
   
end


################################################################################
# Auxiliar functions
################################################################################

# Gets all the file lines in an array.
def get_from_file(path)

   array = []
   file = File.open(path)
   if file != nil
      array = file.readlines.map(&:chomp)    # Discard the final '\n'
      file.close
   end

   return array

end
\end{lstlisting}


\subsection{torque\_functions.rb}


\begin{lstlisting}
# Starts a torque cloud.
def torque_cloud_start(resource, torque_roles)

   head    = torque_roles[:head]
   compute = torque_roles[:compute]

   # Start services
   
   # Start head node
   start_head(resource, head)
   
   # Start compute nodes
   compute.each do |vm|
      start_compute(resource, vm, head)
   end
   
   
   # Start monitoring
   torque_monitor(resource, head, :head)
   compute.each do |vm|
      torque_monitor(resource, vm, :compute)
   end
   
   return true
   
end


################################################################################
# Start node functions
################################################################################

# Starts a head node.
def start_head(resource, head)
   
   user = resource[:vm_user]
   puts "Starting trqauthd on head node"
   check_command = "ps aux | grep -v grep | grep trqauthd"
   out, success = CloudSSH.execute_remote(check_command, user, head)
   unless success
      command = "/etc/init.d/trqauthd start > /dev/null 2> /dev/null"
      out, success = CloudSSH.execute_remote(command, user, head)
      unless success
         err "Impossible to start trqauthd in #{head}"
         return false
      end
   end
   
   puts "Starting pbs_server on head node"
   check_command = "ps aux | grep -v grep | grep pbs_server"
   out, success = CloudSSH.execute_remote(check_command, user, head)
   unless success
      command = "/bin/bash /root/cloud/torque/start-pbs-server"
      out, success = CloudSSH.execute_remote(command, user, head)
      unless success
         err "Impossible to start pbs_server in #{head}"
         return false
      end
   end
   
   puts "Starting pbs_sched on head node"
   check_command = "ps aux | grep -v grep | grep pbs_sched"
   out, success = CloudSSH.execute_remote(check_command, user, head)
   unless success
      command = "/bin/bash /root/cloud/torque/start-pbs-sched"
      out, success = CloudSSH.execute_remote(command, user, head)
      unless success
         err "Impossible to start pbs_sched in #{head}"
         return false
      end
   end

end


# Starts a compute node.
def start_compute(resource, compute, head)

   user = resource[:vm_user]
   puts "Starting pbs_mom on compute node"
   check_command = "ps aux | grep -v grep | grep pbs_mom"
   command = "/bin/bash /root/cloud/torque/start-pbs-mom"
   out, success = CloudSSH.execute_remote(check_command, user, compute)
   unless success
      out, success = CloudSSH.execute_remote(command, user, compute)
      unless success
         err "Impossible to start pbs_mom in #{compute}"
         return false
      end
      
      # Add the node to the compute node list on head
      add_compute_node(resource, compute, head)
   end

end


################################################################################
# Monitor node functions
################################################################################

# Monitors a virtual machine belonging to a torque cloud.
def torque_monitor(resource, vm, role)

   if role == :head
      puts "[Torque monitor] Monitoring head"
      monitor_head(resource, vm)
      puts "[Torque monitor] Monitored head"

   elsif role == :compute
      puts "[Torque monitor] Monitoring compute"
      monitor_compute(resource, vm)
      puts "[Torque monitor] Monitored compute"

   else
      puts "[Torque monitor] Unknown role: #{role}"
   end
   
end


# Monitors a head node.
def monitor_head(resource, vm)

   user = resource[:vm_user]

   check_command1 = "ps aux | grep -v grep | grep trqauthd"
   check_command2 = "ps aux | grep -v grep | grep god | grep pbs-server.god"
   check_command3 = "ps aux | grep -v grep | grep god | grep pbs-sched.god"
   
   out1, success1 = CloudSSH.execute_remote(check_command1, user, vm)
   out2, success2 = CloudSSH.execute_remote(check_command2, user, vm)
   out3, success3 = CloudSSH.execute_remote(check_command3, user, vm)
   unless success1 && success2 && success3
      puts "[Torque monitor] God or trqauthd are not running in #{vm}"
      
      # Try to start monitoring again
      puts "[Torque monitor] Starting monitoring head on #{vm}"
      if start_monitor_head(resource, vm)
         puts "[Torque monitor] Successfully started to monitor head on #{vm}"
      else
         err "[Torque monitor] Impossible to monitor head on #{vm}"
      end
   end
   
end


# Monitors a compute node.
def monitor_compute(resource, vm)

   user = resource[:vm_user]

   # Obtain head node's IP
   #vm_ips, vm_ip_roles = torque_yaml_ips(resource[:ip_file])
   vm_ips, vm_ip_roles = torque_parse_ips(resource[:head], resource[:compute])
   head = vm_ip_roles[:head]

   # Check if the node is in the list of compute nodes
   command = "qmgr -c \"list node @localhost\""      # Get a list of all nodes
   out, success = CloudSSH.execute_remote(command, user, head)
   unless success
      err "[Torque monitor] Impossible to obtain node list from #{head}"
   end

   command = "hostname"
   out2, success = CloudSSH.execute_remote(command, user, vm)
   unless success
      err "[Torque monitor] Impossible to obtain hostname for #{vm}"
   end

   # Add the node to the list of compute nodes
   hostname = out2.chomp()
   if out.include? "Node #{hostname}"
      puts "[Torque monitor] #{hostname} (#{vm}) already in head's list node"
   else
      puts "[Torque monitor] Adding #{hostname} to the list of compute nodes"
      add_compute_node(resource, vm, head)
   end

   # Start monitoring
   check_command = "ps aux | grep -v grep | grep god | grep pbs-mom.god"
   out, success = CloudSSH.execute_remote(check_command, user, vm)
   unless success
      puts "[Torque monitor] God is not running in #{vm}"
      
      # Try to start monitoring again
      puts "[Torque monitor] Starting monitoring compute on #{vm}"
      if start_monitor_compute(resource, vm)
         puts "[Torque monitor] Successfully started to monitor compute on #{vm}"
      else
         err "[Torque monitor] Impossible to monitor compute on #{vm}"
      end
   end
   
end


################################################################################
# Stop functions
################################################################################

# Stops a torque cloud.
def torque_cloud_stop(resource, torque_roles)

   head    = torque_roles[:head]
   compute = torque_roles[:compute]
   
   # Stop compute nodes
   compute.each do |vm|
      stop_compute(resource, vm, head)
   end
   
   # Stop head node
   stop_head(resource, head)
   
end


# Stops a head node.
def stop_head(resource, head)
   
   user = resource[:vm_user]
   
   puts "Stopping pbs_sched on head node"
   command = 'pkill -f pbs-sched\(.\)god'    # We are looking for pbs-sched.god
   out, success = CloudSSH.execute_remote(command, user, head)
   if success
      command = "pkill pbs_sched"
      out, success = CloudSSH.execute_remote(command, user, head)
      unless success
         err "Impossible to stop pbs_sched in #{head}"
         return false
      end
   else
      err "Impossible to stop pbs_sched monitoring in #{head}"
   end
   
   puts "Stopping pbs_server on head node"
   command = 'pkill -f pbs-server\(.\)god'   # We are looking for pbs-server.god
   out, success = CloudSSH.execute_remote(command, user, head)
   if success
      command = "pkill pbs_server"
      out, success = CloudSSH.execute_remote(command, user, head)
      unless success
         err "Impossible to stop pbs_server in #{head}"
         return false
      end
   else
      err "Impossible to stop pbs_server monitoring in #{head}"
   end
   
   puts "Stopping trqauthd on head node"
   command = "/etc/init.d/trqauthd stop"     # Do not kill it, stop it
   out, success = CloudSSH.execute_remote(command, user, head)
   unless success
      err "Impossible to stop trqauthd in #{head}"
      return false
   end

end


# Stops a compute node.
def stop_compute(resource, compute, head)

   user = resource[:vm_user]

   puts "Stopping pbs_mom on compute nodes"
   command = 'pkill -f pbs-mom\(.\)god'
   out, success = CloudSSH.execute_remote(command, user, compute)
   if success
      command = "pkill pbs_mom"
      out, success = CloudSSH.execute_remote(command, user, compute)
      unless success
         err "Impossible to stop pbs_mom in #{compute}"
         return false
      end
      
      # Add the node to the compute node list on head
      del_compute_node(resource, compute, head)
   end

end


################################################################################
# Auxiliar start and stop node functions
################################################################################

# Adds a compute node to the list in the head node.
def add_compute_node(resource, vm, head)

   user = resource[:vm_user]

   command = "hostname"
   out, success = CloudSSH.execute_remote(command, user, vm)
   unless success
      err "Impossible to obtain hostname for #{vm}"
      return false
   end
   hostname = out.chomp()
   command = "qmgr -c \"create node #{hostname}\""
   out, success = CloudSSH.execute_remote(command, user, head)
   unless success
      err "Impossible to add #{hostname} as a compute node in #{head}"
      return false
   end
   command = "pbsnodes -c #{hostname}"
   out, success = CloudSSH.execute_remote(command, user, head)
   unless success
      err "Impossible to clear offline from #{hostname} in #{head}"
      return false
   end
   
end


# Deletes a compute node from the list in the head node.
def del_compute_node(resource, vm, head)

   user = resource[:vm_user]
   
   command = "hostname"
   out, success = CloudSSH.execute_remote(command, user, vm)
   unless success
      err "Impossible to obtain hostname for #{vm}"
      return false
   end
   hostname = out
   command = "qmgr -c \"delete node #{hostname}\""
   out, success = CloudSSH.execute_remote(command, user, head)
   unless success
      err "Impossible to delete #{hostname} as a compute node in #{head}"
      return false
   end
   
end


################################################################################
# Auxiliar monitor node functions
################################################################################

# Starts monitoring on head node.
def start_monitor_head(resource, vm)
   
   user = resource[:vm_user]
   god_port = 17165
   
   # The trqauthd script is intelligent enough to be initiated as many times
   # as you want without problem: if it is already started it will not be
   # started again
   command = "/etc/init.d/trqauthd start > /dev/null 2> /dev/null"
   out, success = CloudSSH.execute_remote(command, user, vm)
   unless success
      err "[Torque monitor] Impossible to run /etc/init.d/trqauthd start at #{vm}"
      return false
   end
   
   # Monitor head node pbs_server and pbs_sched processes with god
   
   # pbs_server is up and running
   path = "/etc/puppet/modules/torque/files/torque-god/pbs-server.god"
   command = "mkdir -p /etc/god"
   out, success = CloudSSH.execute_remote(command, user, vm)
   unless success
      err "[Torque monitor] Impossible to create /etc/god at #{vm}"
      return false
   end
   out, success = CloudSSH.copy_remote(path, vm, "/etc/god")
   unless success
      err "[Torque monitor] Impossible to copy #{path} to #{vm}"
      return false
   end
   port = god_port
   command = "god -c /etc/god/pbs-server.god -p #{port}"
   out, success = CloudSSH.execute_remote(command, user, vm)
   unless success
      err "[Torque monitor] Impossible to run god in #{vm}"
      return false
   end
   
   # pbs_sched is up and running
   path = "/etc/puppet/modules/torque/files/torque-god/pbs-sched.god"
   command = "mkdir -p /etc/god"
   out, success = CloudSSH.execute_remote(command, user, vm)
   unless success
      err "[Torque monitor] Impossible to create /etc/god at #{vm}"
      return false
   end
   out, success = CloudSSH.copy_remote(path, vm, "/etc/god")
   unless success
      err "[Torque monitor] Impossible to copy #{path} to #{vm}"
      return false
   end
   port += 1
   command = "god -c /etc/god/pbs-sched.god -p #{port}"
   out, success = CloudSSH.execute_remote(command, user, vm)
   unless success
      err "[Torque monitor] Impossible to run god in #{vm}"
      return false
   end
   
   return true

end


# Starts monitoring on compute node.
def start_monitor_compute(resource, vm)
   
   user = resource[:vm_user]
   god_port = 17165
   
   # Monitor compute node with god: pbs_mom is up and running
   path = "/etc/puppet/modules/torque/files/torque-god/pbs-mom.god"
   command = "mkdir -p /etc/god"
   out, success = CloudSSH.execute_remote(command, user, vm)
   unless success
      err "[Torque monitor] Impossible to create /etc/god at #{vm}"
      return false
   end
   out, success = CloudSSH.copy_remote(path, vm, "/etc/god")
   unless success
      err "[Torque monitor] Impossible to copy #{path} to #{vm}"
      return false
   end
   command = "god -c /etc/god/pbs-mom.god -p #{god_port}"
   out, success = CloudSSH.execute_remote(command, user, vm)
   unless success
      err "[Torque monitor] Impossible to run god in #{vm}"
      return false
   end
   
   return true

end


################################################################################
# Auxiliar functions
################################################################################

def err(text)
   puts "\e[31m#{text}\e[0m"
end

\end{lstlisting}


\subsection{torque\_yaml.rb}


\begin{lstlisting}
require 'yaml'


# Obtains the IP addresses from the ip_file file. It does NOT check whether
# the file has the proper format.
def torque_yaml_ips(path)

   ips = []
   ip_roles = {}

   file = File.open(path)
   tree = YAML::parse(file)
   
   if tree != nil

      tree = tree.transform

      # Deployment: head node + compute nodes
      head      = tree[:head]
      compute   = tree[:compute]
      
      # Get the IPs that are under the "head" and "compute" labels
      ip_roles[:head]      = get_elements(head)
      ip_roles[:compute]   = get_elements(compute)
      
      # Add the IPs to the array
      ips = ips + ip_roles[:head]
      ips = ips + ip_roles[:compute]
      
      ips = ips.uniq
      
      file.close
      
      return ips, ip_roles
   end
   
end


# Obtains the disk images from the img_file file.
def torque_yaml_images(path)

   img_roles = {}

   file = File.open(path)
   tree = YAML::parse(file)
   
   if tree != nil

      tree = tree.transform

      # Deployment: head node + compute nodes
      head      = tree[:head]
      compute   = tree[:compute]
      
      # Maybe we have been given only an image for all virtual machines
      all       = tree[:all]
      
      if all == nil
         img_roles[:head]      = get_elements(head)
         img_roles[:compute]   = get_elements(compute)
      else
         img_roles[:all]       = get_elements(all)
      end
      
      file.close
      
      return img_roles
   end
   
end


# Transforms the given elements to an array.
def get_elements(array)

   elements = []
   if array != nil
      elements = array.to_a
   end
   return elements

end
\end{lstlisting}


\subsection{manifiestos locales}


\subsubsection{pbs-mom.god}


\begin{lstlisting}
God.watch do |w|
  w.name = "torque-pbs-mom"
  w.interval = 30.seconds # default      
  w.start = "/bin/bash /root/cloud/torque/start-pbs-mom"
  w.pid_file = "/var/spool/torque/mom_priv/mom.lock"
    
  #w.behavior(:clean_pid_file)
  
  # determine the state on startup    
  w.transition(:init, { true => :up, false => :start }) do |on|      
    on.condition(:process_running) do |c|        
      c.running = true     
    end    
  end     

  # determine when process has finished starting    
  w.transition([:start, :restart], :up) do |on|      
    on.condition(:process_running) do |c|        
      c.running = true      
    end       
    # failsafe      
    on.condition(:tries) do |c|        
      c.times = 8        
      c.within = 2.minutes        
      c.transition = :start      
    end    
  end     

  # start if process is not running    
  w.transition(:up, :start) do |on|      
    on.condition(:process_exits)    
  end     

  # lifecycle    
  w.lifecycle do |on|      
    on.condition(:flapping) do |c|        
      c.to_state = [:start, :restart]        
      c.times = 5        
      c.within = 1.minute        
      c.transition = :unmonitored        
      c.retry_in = 10.minutes        
      c.retry_times = 5        
      c.retry_within = 2.hours      
    end    
  end
  
end
\end{lstlisting}


\subsubsection{pbs-sched.god}


\begin{lstlisting}
God.watch do |w|
  w.name = "torque-pbs-sched"
  w.interval = 30.seconds # default      
  w.start = "/bin/bash /root/cloud/torque/start-pbs-sched"
  w.pid_file = "/var/spool/torque/sched_priv/sched.lock"
    
  #w.behavior(:clean_pid_file)
  
  # determine the state on startup    
  w.transition(:init, { true => :up, false => :start }) do |on|      
    on.condition(:process_running) do |c|        
      c.running = true     
    end    
  end     

  # determine when process has finished starting    
  w.transition([:start, :restart], :up) do |on|      
    on.condition(:process_running) do |c|        
      c.running = true      
    end       
    # failsafe      
    on.condition(:tries) do |c|        
      c.times = 8        
      c.within = 2.minutes        
      c.transition = :start      
    end    
  end     

  # start if process is not running    
  w.transition(:up, :start) do |on|      
    on.condition(:process_exits)    
  end     

  # lifecycle    
  w.lifecycle do |on|      
    on.condition(:flapping) do |c|        
      c.to_state = [:start, :restart]        
      c.times = 5        
      c.within = 1.minute        
      c.transition = :unmonitored        
      c.retry_in = 10.minutes        
      c.retry_times = 5        
      c.retry_within = 2.hours      
    end    
  end
  
end
\end{lstlisting}


\subsubsection{pbs-server.god}


\begin{lstlisting}
God.watch do |w|
  w.name = "torque-pbs-server"
  w.interval = 30.seconds # default      
  w.start = "/bin/bash /root/cloud/torque/start-pbs-server"
  w.pid_file = "/var/spool/torque/server_priv/server.lock"
    
  #w.behavior(:clean_pid_file)
  
  # determine the state on startup    
  w.transition(:init, { true => :up, false => :start }) do |on|      
    on.condition(:process_running) do |c|        
      c.running = true     
    end    
  end     

  # determine when process has finished starting    
  w.transition([:start, :restart], :up) do |on|      
    on.condition(:process_running) do |c|        
      c.running = true      
    end       
    # failsafe      
    on.condition(:tries) do |c|        
      c.times = 8        
      c.within = 2.minutes        
      c.transition = :start      
    end    
  end     

  # start if process is not running    
  w.transition(:up, :start) do |on|      
    on.condition(:process_exits)    
  end     

  # lifecycle    
  w.lifecycle do |on|      
    on.condition(:flapping) do |c|        
      c.to_state = [:start, :restart]        
      c.times = 5        
      c.within = 1.minute        
      c.transition = :unmonitored        
      c.retry_in = 10.minutes        
      c.retry_times = 5        
      c.retry_within = 2.hours      
    end    
  end
  
end
\end{lstlisting}
\section{web}
\subsection{web.rb}


\begin{lstlisting}
Puppet::Type.newtype(:web) do
   @doc = "Manages web clouds formed by KVM virtual machines."
   
   
   ensurable do

      desc "The cloud's ensure field can assume one of the following values:
   `running`: The cloud is running.
   `stopped`: The cloud is stopped.\n"
   
      newvalue(:stopped) do
         provider.stop
      end

      newvalue(:running) do
         provider.start
      end

   end


   # General parameters
   
   newparam(:name) do
      desc "The cloud name"
      isnamevar
   end
   
#   newparam(:ip_file) do
#      desc "The file with the cloud description in YAML format"
#   end
#   
#   newparam(:img_file) do
#      desc "The file containing the qemu image(s). You must either provide " +
#           "one image from which all copies shall be made or provide " +
#           "an image for every instance"
#   end
   
   newparam(:vm_domain) do
      desc "The XML file with the virtual machine domain definition. " +
           "Libvirt XML format must be used"
   end
   
   newproperty(:pool, :array_matching => :all) do
      desc "The pool of physical machines"
   end

   
   # Virtual machine parameters
   newparam(:vm_mem) do
      desc "The virtual machine's maximum amopunt of memory. " + 
           "In KiB: 2**10 (or blocks of 1024 bytes)."
      defaultto "1048576"
   end
   
   newparam(:vm_ncpu) do
      desc "The virtual machine's number of CPUs"
      defaultto "1"
   end
   
   
   # Infrastructure parameters

   newparam(:pm_user) do
      desc "The physical machines' user. It must have proper permissions"
      defaultto "dceresuela"
   end

   newparam(:pm_password) do
      desc "The physical machines' password"
      defaultto ""
   end

   newparam(:starting_mac_address) do
      desc "Starting MAC address for new virtual machines"
      defaultto "52:54:00:01:00:00"
   end

   newparam(:vm_user) do
      desc "Virtual machines' user"
      defaultto "root"
   end

   newparam(:root_password) do
      desc "Virtual machines' root password"
      defaultto "root"
   end


   # Web parameters
   
   newproperty(:balancer, :array_matching => :all) do
      desc "The balancer node's information"
   end
   
   newproperty(:server, :array_matching => :all) do
      desc "The server nodes' information"
   end
   
   newproperty(:database, :array_matching => :all) do
      desc "The database node's information"
   end

end
\end{lstlisting}


\subsection{webp\_helper.rb}


\begin{lstlisting}
################################################################################
# Auxiliar functions for web provider
################################################################################

# The functions in this file are defined the same in all providers, but each
# one implements them in their own way. Thus, the headers cannot be modified.

# Starts a web cloud formed by <vm_ips> performing <vm_ip_roles>.
def start_cloud(resource, vm_ips, vm_ip_roles)

   puts "Starting the cloud"
   
   # SSH keys have already been distributed when machines were monitorized,
   # so we do not have to distribute them again
   
   # Start web cloud
   return web_cloud_start(resource, vm_ip_roles)

end


# Obtains vm data from manifest parameters.
def obtain_vm_data(resource)

   puts "Obtaining virtual machines' data"
   return obtain_web_data(resource[:balancer], resource[:server],
                          resource[:database])
   
end
\end{lstlisting}


\subsection{webp.rb}


\begin{lstlisting}
Puppet::Type.type(:web).provide(:webp) do
   desc "Manages web clouds formed by KVM virtual machines"

   # Require web auxiliar files
   require File.dirname(__FILE__) + '/web/web_yaml.rb'
   require File.dirname(__FILE__) + '/web/web_functions.rb'
   require File.dirname(__FILE__) + '/web/web_parsing.rb'
   
   # Require generic files
   require '/etc/puppet/modules/generic-module/provider/mcollective_client.rb'
   Dir["/etc/puppet/modules/generic-module/provider/*.rb"].each { |file| require file }
   
   # Commands needed to make the provider suitable
   commands :ping => "/bin/ping"
   commands :grep => "/bin/grep"
   commands :ps   => "/bin/ps"
   
   # Operating system restrictions
   confine :osfamily => "Debian"

   # Some constants
   #   They are in the generic cloud files

   # Makes sure the cloud is running.
   def start
   
      cloud = Cloud.new(CloudInfrastructure.new(), CloudLeader.new(), resource,
                        method(:err))
      puts "Starting cloud %s" % [resource[:name]]
      
      # Check existence
      if !exists?
         # Cloud does not exist => Startup operations
         
         # Check pool of physical machines
         puts "Checking pool of physical machines..."
         pm_up, pm_down = cloud.check_pool()
         unless pm_down.empty?
            puts "Some physical machines are down"
            pm_down.each do |pm|
               puts " - #{pm}"
            end
         end
         
         # Obtain the virtual machines' IPs
         puts "Obtaining the virtual machines' IPs..."
         #vm_ips, vm_ip_roles, vm_img_roles = obtain_vm_data(method(:web_yaml_ips),
         #                                                   method(:web_yaml_images))
         vm_ips, vm_ip_roles, vm_img_roles = obtain_vm_data(cloud.resource)
         
         # Check whether you are one of the virtual machines
         puts "Checking whether this machine is part of the cloud..."
         part_of_cloud = vm_ips.include?(MY_IP)
         if part_of_cloud
            puts "#{MY_IP} is part of the cloud"
            
            # Check if you are the leader
            if cloud.leader?()
               cloud.leader_start("web", vm_ips, vm_ip_roles, vm_img_roles,
                                  pm_up, method(:web_monitor))
            else
               cloud.common_start()
            end
         else
            puts "#{MY_IP} is not part of the cloud"
            cloud.not_cloud_start("web", vm_ips, vm_ip_roles, vm_img_roles,
                                  pm_up)
         end
         
      else
         
         # Cloud exists => Monitoring operations
         puts "Cloud already started"
         
         # Check if you are the leader
         if cloud.leader?()
            cloud.leader_monitoring(method(:web_monitor))
         else
            puts "#{MY_IP} is not the leader"      # Nothing to do
         end
      end
      
   end


   # Makes sure the cloud is not running.
   def stop

      cloud = Cloud.new(CloudInfrastructure.new(), CloudLeader.new(), resource,
                        method(:err))
      puts "Stopping cloud %s" % [resource[:name]]
      
      if cloud.leader?()
         if !exists?
            err "Cloud does not exist"
            return
         end
         
         if status != :running
            err "Cloud is not running"
            return
         end
         
         if exists? && status == :running
            puts "It is a web cloud"
         
            # Stop cloud infrastructure
            cloud.leader_stop("web", method(:web_cloud_stop))
         end
      else
         puts "#{MY_IP} is not the leader"      # Nothing to do
      end
   
   end


   def status
      if File.exists?("/tmp/cloud-#{resource[:name]}")
         return :running
      else
         return :stopped
      end
   end


   # Ensure methods
#   def create
#      return true
#   end
#   

#   def destroy
#      return true
#   end


   def exists?
      return File.exists?("/tmp/cloud-#{resource[:name]}")
   end
   
   
   #############################################################################
   # Properties need methods
   #############################################################################
   def pool
   end
   
   def pm_user
   end
   
   def root_password
   end
   
   def starting_mac_address
   end
   
   def balancer
   end
   
   def server
   end
   
   def database
   end
   
end
\end{lstlisting}


\subsection{web\_parsing.rb}


\begin{lstlisting}
# Web parsing functions to obtain virual machine's data from manifest's
# arguments.

# Obtains the IP addresses and disk images from the resource[:balancer]
# resource[:server] and resource[:database] arguments.
# web { 'myweb'
#   ...
#   balancer => ["155.210.155.175", "/var/tmp/dceresuela/lucid-lb.img"],
#   server   => ["/etc/puppet/modules/web/files/server-ips.txt",
#                "/etc/puppet/modules/web/files/server-imgs.txt"],
#   database => ["155.210.155.177", "/var/tmp/dceresuela/lucid-db.img"],
#   ...
# }
def obtain_web_data(balancer, server, database)

   ips, ip_roles = web_parse_ips(balancer, server, database)
   img_roles     = web_parse_images(balancer, server, database)
   
   return ips, ip_roles, img_roles

end


# Obtains the IP addresses from the resource[:balancer], resource[:server] and
# resource[:database] arguments.
# web { 'myweb'
#   ...
#   balancer => ["155.210.155.175", "/var/tmp/dceresuela/lucid-lb.img"],
#   server   => ["/etc/puppet/modules/web/files/server-ips.txt",
#                "/etc/puppet/modules/web/files/server-imgs.txt"],
#   database => ["155.210.155.177", "/var/tmp/dceresuela/lucid-db.img"],
#   ...
# }
def web_parse_ips(balancer, server, database)

   ips_index = 0     # Because the IPs are in the first component of the
                     # balancer, server and database arrays, and that is
                     # array[0]
   ips = []
   ip_roles = {}
   
   # Get the IPs that are under the "balancer", "server" and "database"
   # attributes
   ip_roles[:balancer] = []
   ip_roles[:balancer] << balancer[ips_index].chomp
   
   path = server[ips_index]
   ip_roles[:server] = get_from_file(path)
   
   ip_roles[:database] = []
   ip_roles[:database] << database[ips_index].chomp
   
   # Add the IPs to the array
   ips = ips + ip_roles[:balancer]
   ips = ips + ip_roles[:server]
   ips = ips + ip_roles[:database]
   
   ips = ips.uniq
   
   return ips, ip_roles
   
end


# Obtains the disk images from the resource[:balancer], resource[:server] and
# resource[:database] arguments.
# web { 'myweb'
#   ...
#   balancer => ["155.210.155.175", "/var/tmp/dceresuela/lucid-lb.img"],
#   server   => ["/etc/puppet/modules/web/files/server-ips.txt",
#                "/etc/puppet/modules/web/files/server-imgs.txt"],
#   database => ["155.210.155.177", "/var/tmp/dceresuela/lucid-db.img"],
#   ...
# }
def web_parse_images(balancer, server, database)

   img_index = 1     # Because the images are in the second component of the
                     # balancer, server and database arrays, and that is
                     # array[1]
   img_roles = {}

   
   # Get the disk images that are under the "balancer", "server" and "database"
   # attributes
   img_roles[:balancer] = []
   img_roles[:balancer] << balancer[img_index].chomp
   
   path = server[img_index]
   img_roles[:server] = get_from_file(path)
   
   img_roles[:database] = []
   img_roles[:database] << database[img_index].chomp
   
   return img_roles
   
end


################################################################################
# Auxiliar functions
################################################################################

# Gets all the file lines in an array.
def get_from_file(path)

   array = []
   file = File.open(path)
   if file != nil
      array = file.readlines.map(&:chomp)    # Discard the final '\n'
      file.close
   end

   return array

end
\end{lstlisting}


\subsection{web\_functions.rb}


\begin{lstlisting}
# Starts a web cloud.
def web_cloud_start(resource, web_roles)

   balancers = web_roles[:balancer]
   servers   = web_roles[:server]
   databases = web_roles[:database]

   # Start services
   
   # Start load balancers => Start nginx
   puts "Starting nginx on load balancers"
   balancers.each do |vm|
      start_balancer(resource, vm)
   end
   
   # Start web servers => Start sinatra application
   puts "Starting ruby web3 on web servers"
   servers.each do |vm|
      start_server(resource, vm)
   end
   
   # Database servers start at boot time, but check whether they have started
   # and if they have not, start them
   puts "Starting mysql on database servers"
   databases.each do |vm|
      start_database(resource, vm)
   end
   
   
   # Start monitoring
   
   # Load balancers
   balancers.each do |vm|
      start_monitor_balancer(resource, vm)
   end
   
   # Web servers
   servers.each do |vm|
      start_monitor_server(resource, vm)
   end
   
   # Database servers
   databases.each do |vm|
      start_monitor_database(resource, vm)
   end
   
   return true
   
end


################################################################################
# Start node functions
################################################################################

# Starts a load balancer.
def start_balancer(resource, vm)
   
   command = "/etc/init.d/nginx start > /dev/null 2> /dev/null"
   user = resource[:vm_user]
   if vm == MY_IP
      result = `#{command}`
      unless $?.exitstatus == 0
         err "Impossible to start balancer in #{vm}"
         return false
      end
   else
      out, success = CloudSSH.execute_remote(command, user, vm)
      unless success
         err "Impossible to start balancer in #{vm}"
         return false
      end
   end

end


# Starts a web server.
def start_server(resource, vm)
   
   # The ruby-web3 file should have been already copied at this point. It should
   # have been copied when installing the web server.
   command = "/etc/init.d/ruby-web3 start"
   user = resource[:vm_user]
   if vm == MY_IP
      result = `#{command}`
      unless $?.exitstatus == 0
         err "Impossible to start server in #{vm}"
         return false
      end
   else
      out, success = CloudSSH.execute_remote(command, user, vm)
      unless success
         err "Impossible to start server in #{vm}"
         return false
      end
   end

end


# Starts a database server.
def start_database(resource, vm)
   
   check_command = "ps aux | grep -v grep | grep mysql"
   command = "/usr/bin/service mysql start"
   user = resource[:vm_user]
   if vm == MY_IP
      result = `#{check_command}`
      unless $?.exitstatus == 0
         result = `#{command}`
         unless $?.exitstatus == 0
            err "Impossible to start database in #{vm}"
            return false
         end
      end
   else
      out, success = CloudSSH.execute_remote(check_command, user, vm)
      unless success
         out, success = CloudSSH.execute_remote(command, user, vm)
         unless success
            err "Impossible to start database in #{vm}"
            return false
         end
      end
   end
   
end


################################################################################
# Monitor node functions
################################################################################

# Monitors a virtual machine belonging to a web cloud.
def web_monitor(resource, vm, role)

   if role == :balancer
      puts "[Web monitor] Monitoring load balancer"
      
      # Run puppet
      unless start_monitor_balancer(resource, vm)
         puts "[Web monitor] Impossible to monitor load balancer on #{vm}"
      end
      puts "[Web monitor] Monitored load balancer"

   elsif role == :server
      puts "[Web monitor] Monitoring web server"
      
      # Run puppet
      unless start_monitor_server(resource, vm)
         puts "[Web monitor] Impossible to monitor web server on #{vm}"
      end
      
      puts "[Web monitor] Monitored web server"
      
   elsif role == :database
      puts "[Web monitor] Monitoring database"
      
      # Check god is running
      check_command = "ps aux | grep -v grep | grep god | grep database.god"
      user = resource[:vm_user]
      out, success = CloudSSH.execute_remote(check_command, user, vm)
      unless success
         puts "[Web monitor] God is not running in #{vm}"
         
         # Try to start monitoring again
         puts "[Web monitor] Starting monitoring database on #{vm}"
         if start_monitor_database(resource, vm)
            puts "[Web monitor] Successfully started to monitor database on #{vm}"
         else
            err "[Web monitor] Impossible to monitor database on #{vm}"
         end
      end
      puts "[Web monitor] Monitored database"

   else
      puts "[Web monitor] Unknown role: #{role}"
   end
   
end


# Starts monitoring on load balancer.
def start_monitor_balancer(resource, vm)

   # Copy the puppet manifest
   path = "/etc/puppet/modules/web/files/web-manifests/balancer.pp"
   out, success = CloudSSH.copy_remote(path, vm, "/tmp")
   unless success
      err "[Web monitor] Impossible to copy balancer manifest to #{vm}"
      return false
   end

   # Monitor load balancer with puppet
   # While god monitoring will be done in a loop this will only be done if
   # explicitly invoked, so we must call 'puppet apply' every time.
#   command = "puppet apply /tmp/balancer.pp"
#   user = resource[:vm_user]
#   out, success = CloudSSH.execute_remote(command, user, vm)
#   unless success
#      err "[Web monitor] Impossible to run puppet in #{vm}"
#      return false
#   end
   
   # We have to ensure that the node will be auto-monitoring itself
   user = resource[:vm_user]
   cloud_cron = CloudCron.new()
   cron_time = "*/1 * * * *"
   cron_command = "puppet apply /tmp/balancer.pp"
   cron_out = "/root/balancer.out"
   cron_err = "/root/balancer.err"
   line = cloud_cron.create_line(cron_time, cron_command, cron_out, cron_err)
   unless cloud_cron.add_line(line, user, vm)
      err "[Web monitor] Impossible to put balancer.pp in crontab in #{vm}"
      return false
   end
   
   # Execute crontab
   command = "crontab /var/spool/cron/crontabs/root"
   out, success = CloudSSH.execute_remote(command, user, vm)
   if success
      puts "[Web monitor] Executed crontab in #{vm}"
   else
      err "[Web monitor] Impossible to execute crontab in #{vm}"
      return
   end
   
   return true
   
end


# Starts monitoring on web server.
def start_monitor_server(resource, vm)

   # Copy the puppet manifests: general, start and stop
   path = "/etc/puppet/modules/web/files/web-manifests/server.pp"
   out, success = CloudSSH.copy_remote(path, vm, "/tmp")
   unless success
      err "[Web monitor] Impossible to copy server manifest to #{vm}"
      return false
   end
   
   path = "/etc/puppet/modules/web/files/web-manifests/server-start.pp"
   out, success = CloudSSH.copy_remote(path, vm, "/tmp")
   unless success
      err "[Web monitor] Impossible to copy server start manifest to #{vm}"
      return false
   end
   
   path = "/etc/puppet/modules/web/files/web-manifests/server-stop.pp"
   out, success = CloudSSH.copy_remote(path, vm, "/tmp")
   unless success
      err "[Web monitor] Impossible to copy server stop manifest to #{vm}"
      return false
   end
   
   # Monitor web server with puppet: installation files and required gems
   user = resource[:vm_user]
   cloud_cron = CloudCron.new()
   cron_time = "*/1 * * * *"
   cron_command = "puppet apply /tmp/server.pp"
   cron_out = "/root/server.out"
   cron_err = "/root/server.err"
   line = cloud_cron.create_line(cron_time, cron_command, cron_out, cron_err)
   unless cloud_cron.add_line(line, user, vm)
      err "[Web monitor] Impossible to put server.pp in crontab in #{vm}"
      return false
   end

   # Monitor web server with puppet: web server is up and running
   cron_time = "*/1 * * * *"
   cron_command = "puppet apply /tmp/server-start.pp"
   cron_out = "/root/server-start.out"
   cron_err = "/root/server-start.err"
   line = cloud_cron.create_line(cron_time, cron_command, cron_out, cron_err)
   unless cloud_cron.add_line(line, user, vm)
      err "[Web monitor] Impossible to put server-start.pp in crontab in #{vm}"
      return false
   end
   
   # Execute crontab
   command = "crontab /var/spool/cron/crontabs/root"
   out, success = CloudSSH.execute_remote(command, user, vm)
   if success
      puts "[Web monitor] Executed crontab in #{vm}"
   else
      err "[Web monitor] Impossible to execute crontab in #{vm}"
      return
   end
   
   return true

end


# Starts monitoring on database.
def start_monitor_database(resource, vm)

   user = resource[:vm_user]

   # Monitor database with god due to puppet vs ubuntu mysql bug
   # http://projects.puppetlabs.com/issues/12773
   # Therefore there is no puppet monitoring, only god
   path = "/etc/puppet/modules/web/files/web-god/database.god"
   command = "mkdir -p /etc/god"
   out, success = CloudSSH.execute_remote(command, user, vm)
   unless success
      err "[Web monitor] Impossible to create /etc/god at #{vm}"
      return false
   end
   out, success = CloudSSH.copy_remote(path, vm, "/etc/god")
   unless success
      err "[Web monitor] Impossible to copy #{path} to #{vm}"
      return false
   end
   command = "god -c /etc/god/database.god"
   out, success = CloudSSH.execute_remote(command, user, vm)
   unless success
      err "[Web monitor] Impossible to run god in #{vm}"
      return false
   end
   
   return true

end


################################################################################
# Stop functions
################################################################################

# Stops a web cloud.
def web_cloud_stop(resource, web_roles)

   balancers = web_roles[:balancer]
   servers   = web_roles[:server]
   databases = web_roles[:database]

   # Stop services
   
   # Stop load balancers => Stop nginx
   puts "Stopping nginx on load balancers"
   balancers.each do |vm|
      stop_balancer(resource, vm)
   end
   
   # Stop web servers => Stop sinatra application
   puts "Stopping ruby web3 on web servers"
   servers.each do |vm|
      stop_server(resource, vm)
   end
   
   # Stop database servers => Stop mysql
   puts "Stopping mysql on database servers"
   databases.each do |vm|
      stop_database(resource, vm)
   end

end


# Stops a load balancer.
def stop_balancer(resource, vm)

   user = resource[:vm_user]

   # It is being monitored with puppet through a crontab file, so we have to
   # stop monitoring first
   cloud_cron = CloudCron.new()
   word = "balancer"
   cloud_cron.delete_line_with_word(word, user, vm)
   
   # Once we have stopped monitoring we stop the load balancer
   command = "/etc/init.d/nginx stop > /dev/null 2> /dev/null"
   out, success = CloudSSH.execute_remote(command, user, vm)
   unless success
      err "Impossible to stop balancer in #{vm}"
      return false
   end

end


# Stops a web server.
def stop_server(resource, vm)

   user = resource[:vm_user]

   # It is being monitored with puppet through a crontab file, so we have to
   # stop monitoring first
   cloud_cron = CloudCron.new()
   word1 = "server"
   word2 = "server-start"
   cloud_cron.delete_line_with_word(word1, user, vm)
   cloud_cron.delete_line_with_word(word2, user, vm)
   
   # Once we have stopped monitoring we stop the web server
   command = "/etc/init.d/ruby-web3 stop > /dev/null 2> /dev/null"
   out, success = CloudSSH.execute_remote(command, user, vm)
   unless success
      err "Impossible to stop web server in #{vm}"
      return false
   end
   
end


# Stops a database server.
def stop_database(resource, vm)

   user = resource[:vm_user]

   command = 'pkill -f database\(.\)god'    # We are looking for database.god
   out, success = CloudSSH.execute_remote(command, user, vm)
   if success
      command = "/usr/bin/service mysql stop"      # Do not kill it, stop it
      out, success = CloudSSH.execute_remote(command, user, vm)
      unless success
         err "Impossible to stop database in #{vm}"
         return false
      end
   else
      err "Impossible to stop database monitoring in #{vm}"
   end

end


################################################################################
# Auxiliar functions
################################################################################

def err(text)
   puts "\e[31m#{text}\e[0m"
end

\end{lstlisting}


\subsection{web\_yaml.rb}


\begin{lstlisting}
require 'yaml'


# Obtains the IP addresses from the ip_file file. It does NOT check whether
# the file has the proper format.
def web_yaml_ips(path)

   ips = []
   ip_roles = {}

   file = File.open(path)
   tree = YAML::parse(file)
   
   if tree != nil

      tree = tree.transform

      # Classic deployment: load balancer + web servers + database
      balancer = tree[:balancer]
      server   = tree[:server]
      database = tree[:database]
      
      # Get the IPs that are under the "balancer", "server" and "database" labels
      ip_roles[:balancer] = get_elements(balancer)
      ip_roles[:server]   = get_elements(server)
      ip_roles[:database] = get_elements(database)
      
      # Add the IPs to the array
      ips = ips + ip_roles[:balancer]
      ips = ips + ip_roles[:server]
      ips = ips + ip_roles[:database]
      
      ips = ips.uniq
      
      file.close
      
      return ips, ip_roles
   end
   
end


# Obtains the disk images from the img_file file.
def web_yaml_images(path)

   img_roles = {}

   file = File.open(path)
   tree = YAML::parse(file)
   
   if tree != nil

      tree = tree.transform

      # Classic deployment: load balancer + web servers + database
      balancer = tree[:balancer]
      server   = tree[:server]
      database = tree[:database]
      
      # Maybe we have been given only an image for all virtual machines
      all      = tree[:all]
      
      if all == nil
         img_roles[:balancer] = get_elements(balancer)
         img_roles[:server]   = get_elements(server)
         img_roles[:database] = get_elements(database)
      else
         img_roles[:all]      = get_elements(all)
      end
      
      file.close
      
      return img_roles
   end
   
end


# Transforms the given elements to an array.
def get_elements(array)

   elements = []
   if array != nil
      elements = array.to_a
   end
   return elements

end
\end{lstlisting}


\subsection{manifiestos locales}


\subsubsection{balancer.pp}


\begin{lstlisting}
package { 'nginx':
   ensure => present,
}

file { '/etc/nginx/nginx.cnf':
   require => Package['nginx'],
}

service { 'nginx':
   ensure => running,
   enable => true,
   hasstatus => true,
   require => Package['nginx'],
}
\end{lstlisting}


\subsubsection{server.pp}

\begin{lstlisting}
package {
   'ruby1.8-dev':        ensure => present;
   'libmysqlclient-dev': ensure => present;
   'rubygems':           ensure => present;
   #'rubygems':           ensure => latest; # It SHOULD work, but let's not force it
   
   # Gems: rubygems, mysql, sinatra, activerecord.
   'mysql':        provider => "gem", ensure => present;
   'sinatra':      provider => "gem", ensure => present;
   'activerecord': provider => "gem", ensure => present;
}
\end{lstlisting}


\subsubsection{server-start.pp}


\begin{lstlisting}
service { 'ruby-web3':
   provider => "debian",
   ensure => running,
}
\end{lstlisting}


\subsubsection{server-stop.pp}


\begin{lstlisting}
service { 'ruby-web3':
   provider => "debian",
   ensure => stopped,
}

\end{lstlisting}


\subsubsection{server.god}


\begin{lstlisting}
God.watch do |w|
  w.name = "ruby-web3"
  w.interval = 30.seconds # default      
  w.start = "/bin/bash /root/cloud/web/server/start-ruby-web3"
  w.pid_file = "/var/run/ruby-web3.pid"
    
  w.behavior(:clean_pid_file)
  
  # determine the state on startup    
  w.transition(:init, { true => :up, false => :start }) do |on|      
    on.condition(:process_running) do |c|        
      c.running = true     
    end    
  end     

  # determine when process has finished starting    
  w.transition([:start, :restart], :up) do |on|      
    on.condition(:process_running) do |c|        
      c.running = true      
    end       
    # failsafe      
    on.condition(:tries) do |c|        
      c.times = 8        
      c.within = 2.minutes        
      c.transition = :start      
    end    
  end     

  # start if process is not running    
  w.transition(:up, :start) do |on|      
    on.condition(:process_exits)    
  end     

  # lifecycle    
  w.lifecycle do |on|      
    on.condition(:flapping) do |c|        
      c.to_state = [:start, :restart]        
      c.times = 5        
      c.within = 1.minute        
      c.transition = :unmonitored        
      c.retry_in = 10.minutes        
      c.retry_times = 5        
      c.retry_within = 2.hours      
    end    
  end
  
end
\end{lstlisting}


\subsubsection{database.pp}


\begin{lstlisting}
# Package, file and service
package { 'mysql-server':
   ensure => present,
}

file { '/etc/mysql/my.cnf':
   require => Package['mysql-server'],
}

service { 'mysql':
   ensure => running,
   enable => true,
   hasstatus => true,
   require => Package["mysql-server"],
   hasrestart => true,
   restart => "/usr/bin/service mysql restart"
}

# User and group
user { "mysql":
   ensure => present,
   gid => "mysql",
   shell => "/bin/false",
   require => Group["mysql"],
}

group { "mysql":
   ensure => present,
}
\end{lstlisting}


\subsubsection{database.god}

\begin{lstlisting}
#
# Borrowed from http://thewebfellas.com/blog/2008/2/12/a-simple-faith-monitoring-by-god
#

God.watch do |w|
  w.name = "mysql"
  w.interval = 30.seconds # default      
  w.start = "service mysql start"
  w.stop = "service mysql stop"
  w.restart = "service mysql restart"
  w.start_grace = 20.seconds
  w.restart_grace = 20.seconds
  w.pid_file = "/var/lib/mysql/lucid-db.pid"
    
  w.behavior(:clean_pid_file)

  # determine the state on startup    
  w.transition(:init, { true => :up, false => :start }) do |on|      
    on.condition(:process_running) do |c|        
      c.running = true     
    end    
  end     

  # determine when process has finished starting    
  w.transition([:start, :restart], :up) do |on|      
    on.condition(:process_running) do |c|        
      c.running = true      
    end       
    # failsafe      
    on.condition(:tries) do |c|        
      c.times = 8        
      c.within = 2.minutes        
      c.transition = :start      
    end    
  end     

  # start if process is not running    
  w.transition(:up, :start) do |on|      
    on.condition(:process_exits)    
  end     

  # lifecycle    
  w.lifecycle do |on|      
    on.condition(:flapping) do |c|        
      c.to_state = [:start, :restart]        
      c.times = 5        
      c.within = 1.minute        
      c.transition = :unmonitored        
      c.retry_in = 10.minutes        
      c.retry_times = 5        
      c.retry_within = 2.hours      
    end    
  end
  
end
\end{lstlisting}


%\subsubsection{}

%\begin{lstlisting}

%\end{lstlisting}




