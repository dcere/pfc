\chapter{AppScale}
\label{anx:appscale}


En este anexo se realiza un análisis más detallado de algunos aspectos de AppScale tales como la instalación de AppScale, los distintos tipos de roles que puede tomar un nodo y los distintos tipos de despliegue que podemos especificar, las bases de datos e infraestructuras que permite usar y el desarrollo y organización del \emph{software}.


%%%%%%%%%%%%%%%%%%%%%%%%%%%%%%%%%%%%%%%%%%%%%%%%%%%%%%%%%%%%%%%%%%%%%%%%%%%%%%%%
\section{Roles y despliegue}
\label{anx:appscale-roles}

A continuación se explica de manera más detallada los distintos roles que puede desempeñar un nodo dentro de la arquitectura AppScale. \\

En primer lugar tenemos los roles simples:

\begin{description}
\item[\texttt{Shadow}]: Comprueba el estado en el que se encuentran el resto de nodos y se asegura de que están ejecutando los servicios que deben.
\item[\texttt{Load balancer}]: Servicio web que lleva a los usuarios a las aplicaciones. Posee también una página en la que informa del estado de todas las máquinas desplegadas.
\item[\texttt{AppEngine}]: Versión modificada de los SDKs de Google App Engine. Además de alojar las aplicaciones web añaden la capacidad de almacenar y recuperar datos de bases de datos que soporten la API de Google Datastore.
\item[\texttt{Database}]: Ejecuta los servicios necesarios para alojar a la base de datos elegida.
\item[\texttt{Login}]: La máquina principal que lleva a los usuarios a las aplicaciones App Engine. Difiere del Load Balancer en que esta es la única máquina desde la que se pueden hacer funciones administrativas. Puede haber muchas máquinas que hagan la función de Load Balancer pero sólo habrá una que haga función de Login.
\item[\texttt{Memcache}]: Proporciona soporte para almacenamiento en caché para las aplicaciones App Engine.
\item[\texttt{Zookeeper}]: Aloja los metadatos necesarios para hacer transacciones en las bases de datos.
\item[\texttt{Open}]: No ejecuta nada por defecto, pero está disponible en caso de que sea necesario. Estas máquinas son las utilizadas para ejecutar trabajos MPI.
\end{description}

Además de los roles simples también se proporcionan unos roles agregados que agrupan a varios de los roles simples y que facilitan la descripción de la arquitectura:

\begin{description}
\item[\texttt{Controller}]: Shadow, Load Balancer, Database, Login y Zookeeper.
\item[\texttt{Servers}]: App Engine, Database y Load Balancer.
\item[\texttt{Master}]: Shadow, Load Balancer y Zookeeper.
\end{description}

Estos roles pueden usarse en dos tipos de despliegue: por defecto y personalizado. En un despliegue por defecto los nodos únicamente pueden ser de tipo \texttt{Controller} y \texttt{Servers}. En un despliegue personalizado los nodos pueden ser de cualquier tipo del resto de roles.


%%%%%%%%%%%%%%%%%%%%%%%%%%%%%%%%%%%%%%%%%%%%%%%%%%%%%%%%%%%%%%%%%%%%%%%%%%%%%%%%
\section{Bases de datos}


AppScale no obliga a trabajar con una base de datos determinada, sino que es capaz de trabajar con varias bases de datos. Cada una de estas bases de datos implementa la interfaz AppScale DB, para comunicarse con las aplicaciones del App Engine. Las bases de datos que actualmente soporta AppScale son:

\begin{description}
\item[\texttt{Cassandra}] \cite{appscale-cassandra}: Una base de datos híbrida entre BigTable \cite{appscale-bigtable} y Dynamo \cite{appscale-dynamo}. Es una base de datos de nodos idénticos (\emph{peer-to-peer}) con una alta escalabilidad y rendimiento.
\item[\texttt{HBase}] \cite{appscale-hbase}: Una implementación de BigTable realizada en Java. Se ejecuta sobre Hadoop \cite{appscale-hadoop}.
\item[\texttt{Hypertable}] \cite{appscale-hypertable}: Otra implementación de BigTable, esta vez realizada en C++. Al igual que \texttt{HBase} se ejecuta sobre Hadoop. Tiene como objetivo el alto rendimiento.
\item[\texttt{MongoDB}] \cite{appscale-mongodb}: Una base de datos centrada en el almacén de documentos. Se comporta bien en operaciones de adición (\emph{append}).
\item[\texttt{MemcacheDB}] \cite{appscale-memcachedb}: Una base de datos clave-valor distribuida diseñada para ser persistente. Se basa en MemcacheD \cite{appscale-memcached}, una memoria cache distribuida, y usa la misma API que ella. La persistencia la obtiene a través de BerkeleyDB \cite{appscale-berkeleydb}.
\item[\texttt{MySQL Cluster}] \cite{appscale-mysql}: Es usada como una base de datos clave-valor dentro de AppScale.
\item[\texttt{Voldemort}] \cite{appscale-voldemort}: Es una base de datos clave-valor de nodos idénticos (\emph{peer-to-peer}). Se basa en Dynamo y gestiona la persistencia a través de BerkeleyDB
\item[\texttt{Redis}] \cite{appscale-redis}: Base de datos clave-valor con durabilidad opcional.
\end{description}

%% Revisar Y redis? se acepta como base de datos?
La elección de la base de datos a la hora de iniciar AppScale se realiza en el comando \texttt{appscale-run-instances}. Los valores que acepta el comando son: \texttt{cassandra}, \texttt{hbase}, \texttt{hypertable}, \texttt{mongodb}, \texttt{memcachedb}, \texttt{mysql} y \texttt{voldemort}. La base de datos que usa por defecto es \texttt{Cassandra}, así que si queremos elegir otra, deberemos especificarlo claramente. Por ejemplo, en el caso de que eligiésemos \texttt{HBase} como base de datos habría que usar el comando \texttt{appscale-run-instances} de la siguiente manera:

\begin{bashcode}
appscale-run-instances --ips ips.yaml --table hbase
\end{bashcode}


%%%%%%%%%%%%%%%%%%%%%%%%%%%%%%%%%%%%%%%%%%%%%%%%%%%%%%%%%%%%%%%%%%%%%%%%%%%%%%%%
\section{Infraestructuras}

AppScale puede ser ejecutada sobre diversos tipos de infraestructuras públicas, privadas o híbridas. Dentro de las infraestructuras públicas AppScale soporta tanto Amazon EC2 \cite{amazon-ec2} como otras infraestructuras públicas administradas mediante Eucalyptus \cite{eucalyptus}, o de manera más general, cualquier infraestructura que sea compatible con las euca2ools \cite{eucalyptus-euca2ools}. Dentro de las infraestructuras privadas soporta tanto un \emph{cluster} de máquinas con Xen o KVM como infraestructuras privadas gestionadas mediante Eucalyptus. Una infraestructura híbrida puede estar formada por una combinación de infraestructura privada e infraestructura pública. Los casos que se pueden dar son:

\begin{itemize}
\item Cloud privado - Cloud público
\item Clod privado X - Cloud privado Y
\item Clod público (zona X) - Cloud público (zona Y)
\item Clod público (empresa X) - Cloud público (empresa Y)
\end{itemize}


%%%%%%%%%%%%%%%%%%%%%%%%%%%%%%%%%%%%%%%%%%%%%%%%%%%%%%%%%%%%%%%%%%%%%%%%%%%%%%%%
\section{Desarrollo y organización}

AppScale es una tecnología que actualmente se encuentra en un proceso activo de desarrollo. Cuando se empezó este proyecto la versión estable era la 1.5 y hasta ahora se ha seguido trabajando en esa versión; más concretamente en la versión 1.5.1. Otras partes de AppScale como Neptune, el lenguaje usado para la ejecución de trabajos MPI, han progresado desde la versión 0.1.1 hasta la versión 0.2.2 sufriendo severas modificaciones a lo largo del proceso. 

Además de los cambios que está experimentando en el \emph{software}, AppScale está realizando cambios en su organización interna. Al empezar este proyecto el código de AppScale se encontraba alojado en Launchpad (\url{https://launchpad.net/appscale/}) y la información en su página de Google (\url{http://code.google.com/p/appscale/}). Ahora se encuentran realizando la transición hacia Github (\url{https://github.com/AppScale/appscale}) y aunque todo el código puede encontrarse allí, la documentación sigue estando en su página de Google.

La complejidad del problema que trata de resolver, el amplio abanico de tecnologías utilizadas y los frecuentes cambios que sufre hacen que sea una tecnología complicada para trabajar con ella. La falta de documentación en alguna de las áreas no ayuda en este aspecto.
