\chapter{Facter y Puppet}
\label{anx:facter-puppet}


%%%%%%%%%%%%%%%%%%%%%%%%%%%%%%%%%%%%%%%%%%%%%%%%%%%%%%%%%%%%%%%%%%%%%%%%%%%%%%%%
\section{Instalación de libopenssl-ruby}

Si nuestro sistema no tiene instalado \texttt{libopenssl-ruby} tenemos que instalarlo:

\begin{bashcode}
_: apt-get install libopenssl-ruby
\end{bashcode}


%%%%%%%%%%%%%%%%%%%%%%%%%%%%%%%%%%%%%%%%%%%%%%%%%%%%%%%%%%%%%%%%%%%%%%%%%%%%%%%%
\section{Creación del entorno de instalación de Facter y Puppet}

Antes de instalar Facter y Puppet, hay que dar valores a ciertas variables:

\begin{bashcode}
_: FACTER_DIR="/root/facter-1.6.4"
_: PUPPET_DIR="/root/puppet-2.7.9"
_: PATH=$PATH:$FACTER_DIR/bin:$PUPPET_DIR/puppet/bin
_: RUBYLIB=$FACTER_DIR/lib:$PUPPET_DIR/lib
_: export PATH RUBYLIB
\end{bashcode}


%% Put a $ to get back highlightning in gedit
%%%%%%%%%%%%%%%%%%%%%%%%%%%%%%%%%%%%%%%%%%%%%%%%%%%%%%%%%%%%%%%%%%%%%%%%%%%%%%%%
\section{Instalación de Facter}

Instalaremos Facter mediante una instalación por fuentes. Para ello primero tenemos que obtener las fuentes:

\begin{bashcode}
_: wget http://puppetlabs.com/downloads/facter/facter-1.6.4.tgz
\end{bashcode}

Una vez conseguidas las fuentes lo descomprimimos:

\begin{bashcode}
_: gzip -d -c facter-1.6.4.tgz | tar xf -
\end{bashcode}

Y posteriormente lo instalamos:

\begin{bashcode}
_: cd facter-1.6.4
_: ruby install.rb
\end{bashcode}


%%%%%%%%%%%%%%%%%%%%%%%%%%%%%%%%%%%%%%%%%%%%%%%%%%%%%%%%%%%%%%%%%%%%%%%%%%%%%%%%
\section{Instalación de Puppet}

Instalaremos Puppet también mediante fuentes. Obtenemos las fuentes:

\begin{bashcode}
_: wget http://puppetlabs.com/downloads/puppet/puppet-2.7.9.tgz
\end{bashcode}

Lo descomprimimos:

\begin{bashcode}
_: gzip -d -c puppet-2.7.9.tgz | tar xf -
\end{bashcode}

Y lo instalamos:

\begin{bashcode}
_: cd puppet-2.7.9
_: ruby install.rb
\end{bashcode}


%%%%%%%%%%%%%%%%%%%%%%%%%%%%%%%%%%%%%%%%%%%%%%%%%%%%%%%%%%%%%%%%%%%%%%%%%%%%%%%%
\section{Configuración de Puppet}

Nos aseguramos de que el fichero \texttt{/etc/puppet/puppet.conf} existe en nuestro sistema. Si no es así, podemos crearlo de la siguiente manera:

\begin{bashcode}
_: puppetmasterd --genconfig > /etc/puppet/puppet.conf
\end{bashcode}

Tenemos que asegurarnos de que existen tanto un usuario como un grupo \texttt{puppet}. Si no existen, los creamos:

\begin{bashcode}
_: groupadd puppet
_: useradd -g puppet puppet
\end{bashcode}

Por último tenemos que asegurarnos de que el directorio \texttt{/var/lib/puppet/rrd} pertenece al usuario y grupo \texttt{puppet}. Si no pertenece, cambiamos los permisos:

\begin{bashcode}
_: chown puppet /var/lib/puppet/rrd
_: chgrp puppet /var/lib/puppet/rrd
\end{bashcode}

Nota: Si este directorio todavía no existe seguimos adelante con la instalación y ya revisitaremos este punto.


%%%%%%%%%%%%%%%%%%%%%%%%%%%%%%%%%%%%%%%%%%%%%%%%%%%%%%%%%%%%%%%%%%%%%%%%%%%%%%%%
\section{Comprobación de la instalación}

Vamos a comprobar de manera rápida que la instalación de Facter y Puppet fue exitosa:

\begin{bashcode}
_: facter
[Muestra datos del sistema]
_: puppet describe -s user
[Muestra una descripcion del tipo usuario]
\end{bashcode}

Ahora creamos un manifiesto simple en el fichero \texttt{manifest.pp}:

\begin{rubycode}
file {'testfile':
  path    => '/tmp/testfile',
  ensure  => present,
  mode    => 0640,
  content => "I'm a test file.",
}
\end{rubycode}

Lo aplicamos:

\begin{bashcode}
_: puppet apply manifest.pp
\end{bashcode}

Y comprobamos el contenido del fichero \texttt{/tmp/testfile} que Puppet debería haber creado:

\begin{bashcode}
_: cat /tmp/testfile
I'm a test file.
\end{bashcode}

Ahora es momento de volver al directorio \texttt{/var/lib/puppet/rrd} y comprobar que tanto el usuario como el grupo de ese directorio tienen como valor \texttt{puppet}.


%%%%%%%%%%%%%%%%%%%%%%%%%%%%%%%%%%%%%%%%%%%%%%%%%%%%%%%%%%%%%%%%%%%%%%%%%%%%%%%%
\subsection{Comprobación del maestro de Puppet}

Para comprobar que el maestro de puppet puede ejecutarse como un demonio tenemos que hacer:

\begin{bashcode}
_: puppet master
Could not prepare for execution: Got 1 failure(s) while initializing: \
change from directory to file failed: 
Could not set 'file on ensure: Is a directory - /var/lib/puppet/facts
\end{bashcode}

Si aparece este fallo tenemos que ir al fichero de configuración \texttt{/etc/puppet/puppet.conf} y comentar la línea con la propiedad \texttt{factdest}. Esta es la solución propuesta para el \href{https://projects.puppetlabs.com/issues/9491}{bug}.

Si lo ejecutamos de nuevo no debería dar más problemas:

\begin{bashcode}
_: puppet master
\end{bashcode}


%%%%%%%%%%%%%%%%%%%%%%%%%%%%%%%%%%%%%%%%%%%%%%%%%%%%%%%%%%%%%%%%%%%%%%%%%%%%%%%%
\subsection{Comprobación del agente de Puppet}

Nota: Supondremos que el maestro de Puppet se está ejecutando en una máquina que tiene por nombre \texttt{puppet.example.com}. La máquina desde la que se ejecuta el agente tiene por nombre \texttt{node1.example.com}

Desde la máquina que tiene el agente, lanzamos la siguiente orden:

\begin{bashcode}
_: puppet agent --server=puppet.example.com \
--no-daemonize --verbose   # Usa nombres de dominio completos
warning: peer certificate won't be verified in this SSL session
info: Caching certificate for ca
warning: peer certificate won't be verified in this SSL session
warning: peer certificate won't be verified in this SSL session
info: Creating a new SSL certificate request for appscale-image1.cloud.net
info: Certificate Request fingerprint (md5): ... 
warning: peer certificate won't be verified in this SSL session
[Se queda colgado]
[Ctrl + C]
\end{bashcode}

Vamos a la máquina en la que se está ejecutando el maestro y hacemos lo siguiente:

\begin{bashcode}
_: puppet cert --sign node1.example.com
notice: Signed certificate request for node1.example.com
notice: Removing file Puppet::SSL::CertificateRequest node1.example.com at \
'/etc/puppet/ssl/ca/requests/node1.example.com.pem'
\end{bashcode}

Y de vuelta a la máquina del agente:

\begin{bashcode}
_: puppet agent --server=puppet.example.com --no-daemonize --verbose
warning: peer certificate won't be verified in this SSL session
info: Caching certificate for node1.example.com
notice: Starting Puppet client version 2.7.9
info: Caching certificate_revocation_list for ca
info: Caching catalog for node1.example.com
info: Applying configuration version '1331813472'
info: Creating state file /var/lib/puppet/state/state.yaml
notice: Finished catalog run in 0.03 seconds
[Se queda colgado]
[Ctrl + C]
notice: Caught INT; calling stop
\end{bashcode}

Si no tienes una configuración preparada para tu nodo te aparecerá un error, pero no te preocupes: Puppet está funcionando correctamente. El error será similar a éste:

\begin{bashcode}
_: puppet agent --server=puppet.example.com --no-daemonize --verbose
notice: Starting Puppet client version 2.7.9
info: Caching certificate_revocation_list for ca
err: Could not retrieve catalog from remote server: Error 400 on SERVER: \
Could not find default node or by name with 'node1.example.com, node1' \
on node node1.example.com
notice: Using cached catalog
err: Could not retrieve catalog; skipping run
[Se queda colgado]
[Ctrl + C]
notice: Caught INT; calling stop
\end{bashcode}


%%%%%%%%%%%%%%%%%%%%%%%%%%%%%%%%%%%%%%%%%%%%%%%%%%%%%%%%%%%%%%%%%%%%%%%%%%%%%%%%
\subsection{Comprobación del sistema Puppet}

Creamos el fichero \texttt{/etc/puppet/manifests/site.pp} en la máquina del maestro con el siguiente contenido:

\begin{rubycode}
# Create "/tmp/testfile" if it doesn't exist.
class test_class {
    file { "/tmp/testfile":
       ensure => present,
       mode   => 644,
       owner  => root,
       group  => root
    }
}

# tell puppet on which client to run the class
node 'node1' {            # Notice it is node1 and not node1.example.com
    include test_class
}

node 'puppet' {           # Notice it is puppet and not puppet.example.com
}
\end{rubycode}

Y desde la máquina del agente tecleamos esto:

\begin{bashcode}
_: puppet agent --server=puppet.example.com --no-daemonize --verbose
\end{bashcode}

Comprobamos que en la máquina del agente se ha creado el fichero \texttt{/tmp/testfile}:

\begin{bashcode}
_: cat /tmp/testfile
I'm a test file.
\end{bashcode}


%%%%%%%%%%%%%%%%%%%%%%%%%%%%%%%%%%%%%%%%%%%%%%%%%%%%%%%%%%%%%%%%%%%%%%%%%%%%%%%%
\section{Problemas}

Durante la ejecución de Puppet pueden ocurrir varios errores. Como la información de Puppet puede ser en ocasiones algo críptica se muestra a continuación una colección de los errores más comunes y su solución:

\begin{bashcode}
_: puppet master
err: /File[/var/lib/puppet/facts]/ensure: change from directory to file \
failed: Could not set 'file on ensure: Is a directory - /var/lib/puppet/facts
\end{bashcode}

Solución: Comenta la propiedad factdest en el fichero de configuración de Puppet en la máquina del maestro.

\begin{bashcode}
_: puppet agent --server=puppet.example.com --no-daemonize --verbose
err: Could not retrieve catalog from remote server: Error 400 on SERVER: \
Could not find default node or by name with ...
\end{bashcode}

Solución: Necesitas incluir el nombre del agente en el fichero \texttt{site.pp} de la máquina del maestro. O puedes incluir un nodo por defecto:

\begin{rubycode}
node 'default' {
}
\end{rubycode}

\begin{bashcode}
_: puppet master
err: Could not prepare for execution: Retrieved certificate does not match \
private key; please remove certificate from server and regenerate it with \
the current key
\end{bashcode}

Solución: Cambia el nombre del directorio \texttt{/etc/puppet/ssl} a \texttt{/etc/puppet/ssl.old} y prueba de nuevo.

\begin{bashcode}
_: puppet agent --server=puppet.example.com --no-daemonize --verbose
err: Could not send report: SSL_connect returned=1 errno=0 state=SSLv3 read \
server certificate B: certificate verify failed. This is often because the \
time is out of sync on the server or client
\end{bashcode}

Solución: Compleja, mejor mira \href{http://projects.puppetlabs.com/projects/1/wiki/Certificates_And_Security}{aquí}.

\begin{bashcode}
_: puppet master
err: Could not run: Could not create PID file: /var/lib/puppet/run/master.pid
\end{bashcode}

Solución: El maestro de Puppet ya se está ejecutando.

\begin{bashcode}
_: puppet apply manifest.pp
err: Could not evaluate: Puppet::Util::Log requires a message
\end{bashcode}

Solución: Puppet ha salido de manera abrupta. Si estás ejecutando un proveedor asegúrate de que sales de él con \texttt{return} y no con \texttt{exit}.


%%%%%%%%%%%%%%%%%%%%%%%%%%%%%%%%%%%%%%%%%%%%%%%%%%%%%%%%%%%%%%%%%%%%%%%%%%%%%%%%
\section{Versiones instaladas}

\begin{table}[!htbp]
\centering
   \begin{tabular}{|c|c|}
      \hline
      \textbf{Software} & \textbf{Versión} \\ \hline
      Ubuntu & 10.04 \\ \hline
      Ruby   & 1.8.7 \\ \hline
      Facter & 1.6.4 \\ \hline
      Puppet & 2.7.9 \\ \hline
      libopenssl-ruby &	4.2 \\ \hline
   \end{tabular}
\caption{Versiones instaladas de Facter y Puppet.}
\label{table:puppet-versions}
\end{table}
