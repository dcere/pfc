\chapter{AppScale}
\label{anx:appscale}


En este anexo se realiza un análisis más detallado de algunos aspectos de AppScale tales como la instalación de AppScale, los distintos tipos de roles que puede tomar un nodo y los distintos tipos de despliegue que podemos especificar, las bases de datos e infraestructuras que permite usar y el desarrollo y organización del \emph{software}.


\section{Instalación}

Antes de instalar AppScale es muy recomendable actualizar el sistema operativo. Para ello añadimos las siguientes líneas al archivo \texttt{/etc/apt/sources.list}:

\begin{lstlisting}
deb http://archive.ubuntu.com/ubuntu  lucid main restricted universe multiverse
deb http://archive.ubuntu.com/ubuntu  lucid-updates main restricted universe
deb http://security.ubuntu.com/ubuntu lucid-security main restricted universe
\end{lstlisting}

Y actualizamos el sistema operativo:

\begin{bashcode}
_: apt-get update
_: apt-get -y upgrade
\end{bashcode}

Para gestionar una infraestructura AppScale es necesario instalar una serie de programas agrupados bajo el nombre de appscale-tools. Para ello, se debe descargar el archivo \emph{tarball} de \url{http://code.google.com/p/appscale/downloads/list}. Una vez descargado, se instala:

\begin{bashcode}
_: tar xzvf appscale-tools.tar.gz
_: cd appscale-tools
_: sudo bash debian/appscale_build.sh
...
AppScale tools installation completed successfully!
\end{bashcode}

Sin olvidarse de añadir el directorio de las appscale-tools a nuestro \emph{path}:

\begin{bashcode}
_: export PATH=${PATH}:/usr/local/appscale-tools/bin
\end{bashcode}
%% Put a $ to get back highlightning in gedit

Una vez instaladas las appscale-tools instalaremos AppScale. Como lo que vamos a hacer va a ser clonar un repositorio, tenemos que asegurarnos de tener las herramientas necesarias, en este caso \texttt{git}:

\begin{bashcode}
_: apt-get -y install git-core
\end{bashcode}

y cuando esté instalado clonamos el repositorio:

\begin{bashcode}
_: cd /root/
_: git clone git://github.com/AppScale/appscale.git
\end{bashcode}

Accedemos a la carpeta recién creada y ejecutamos el \emph{script} de instalación. La instalación tarda aproximadamente una hora:

\begin{bashcode}
_: cd appscale
_: bash debian/appscale_build.sh
...
AppScale installation completed successfully!
\end{bashcode}

Nota: Si no quieres instalar todo AppScale puedes comentar las partes que no quieras en el \emph{script} de \texttt{appscale\_install.sh}. Por ejemplo, si no quieres que se instale la base de datos Voldemort, puedes comentar las líneas que hacen referencia a ella:

\begin{lstlisting}
...
all)
# scratch install of appscale including post script.
installappscaleprofile
. /etc/profile.d/appscale.sh
installgems
postinstallgems
installsetuptools
installhaproxy
postinstallhaproxy
...
installcassandra
postinstallcassandra
###installvoldemort        # No queremos Voldemort
###postinstallvoldemort    # No queremos Voldemort
installhbase
postinstallhbase
...
\end{lstlisting}


%%%%%%%%%%%%%%%%%%%%%%%%%%%%%%%%%%%%%%%%%%%%%%%%%%%%%%%%%%%%%%%%%%%%%%%%%%%%%%%%
\section{Comprobación de la instalación}

Para comprobar que AppScale se ha instalado correctamente lanzaremos una instancia. En primer lugar creamos un archivo llamado \texttt{ips.yaml} con el siguiente contenido (sustituye la dirección IP por la de tu máquina):

\begin{lstlisting}
--- 
:controller: 155.210.155.73
\end{lstlisting}

A continuación lanza la instancia:

\begin{bashcode}
_: appscale-run-instances --ips ips.yaml
\end{bashcode}

Si en tu navegador web vas a la dirección http://155.210.155.73 (la que corresponda en tu caso) deberías ver la página de inicio de sesión de AppScale.


%%%%%%%%%%%%%%%%%%%%%%%%%%%%%%%%%%%%%%%%%%%%%%%%%%%%%%%%%%%%%%%%%%%%%%%%%%%%%%%%
\section{Versiones instaladas}

\begin{table}[!htbp]
\centering
   \begin{tabular}{|c|c|}
      \hline
      \textbf{Software} & \textbf{Versión} \\ \hline
      Ubuntu & 10.04 \\ \hline
      AppScale & 1.5.1 \\ \hline
      appscale-tools & 1.5 \\ \hline
   \end{tabular}
\caption{Versiones instaladas de AppScale.}
\label{table:puppet-versions}
\end{table}
