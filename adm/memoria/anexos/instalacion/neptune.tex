\chapter{Neptune}
\label{anx:neptune}


Neptune es un lenguaje específico de dominio que permite al usuario ejecutar código en la plataforma AppScale.


%%%%%%%%%%%%%%%%%%%%%%%%%%%%%%%%%%%%%%%%%%%%%%%%%%%%%%%%%%%%%%%%%%%%%%%%%%%%%%%%
\section{Instalación}

El primer paso consiste en la instalación de RubyGems. Para ello, consulta el apéndice \ref{anx:ruby} de instalación de Ruby y RubyGems. Cuando ese paso ya esté completado usaremos las RubyGems para instalar neptune:

\begin{bashcode}
_: gem install neptune
\end{bashcode}

Nota: Puede que en vez de \texttt{gem} tengas que usar \texttt{gem1.8} dependiendo de cómo estén configuradas las RubyGems.


%%%%%%%%%%%%%%%%%%%%%%%%%%%%%%%%%%%%%%%%%%%%%%%%%%%%%%%%%%%%%%%%%%%%%%%%%%%%%%%%
\section{Comprobación de la instalación}

Puedes verificar que neptune se ha instalado satisfactoriamente mirando si existe el ejecutable en \texttt{/usr/bin/neptune} y \texttt{/usr/lib/ruby/gems/1.8/gems/neptune-0.1.4/bin/neptune}.


%%%%%%%%%%%%%%%%%%%%%%%%%%%%%%%%%%%%%%%%%%%%%%%%%%%%%%%%%%%%%%%%%%%%%%%%%%%%%%%%
\section{Versiones instaladas}

\begin{table}[!htbp]
\centering
   \begin{tabular}{|c|c|}
      \hline
      \textbf{Software} & \textbf{Versión} \\ \hline
      Ubuntu & 10.04 \\ \hline
      Ruby & 1.8.7 \\ \hline
      RubyGems & 1.8.10 (o superior) \\ \hline
      neptune & 0.1.4 \\ \hline
   \end{tabular}
\caption{Versión instalada de neptune.}
\label{table:neptune-versions}
\end{table}
