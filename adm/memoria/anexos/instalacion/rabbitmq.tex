\chapter{RabbitMQ}
\label{anx:rabbitmq}


RabbitMQ es un \emph{middleware} de mensajería que implementa el estándar AMQP (Advanced Message Queuing Protocol).


%%%%%%%%%%%%%%%%%%%%%%%%%%%%%%%%%%%%%%%%%%%%%%%%%%%%%%%%%%%%%%%%%%%%%%%%%%%%%%%%
\section{Instalación de RabbitMQ}

Empezaremos añadiendo la línea

\begin{bashcode}
deb http://www.rabbitmq.com/debian/ testing main
\end{bashcode}

a nuestro fichero \texttt{/etc/apt/sources.list} de la siguiente manera:

\begin{bashcode}
_: echo "deb http://www.rabbitmq.com/debian/ testing main" >> \
/etc/apt/sources.list
\end{bashcode}

Para evitar avisos acerca de paquetes que no han sido firmados, podemos añadir la clave pública de RabbitMQ a nuestra lista de claves:

\begin{bashcode}
_: wget http://www.rabbitmq.com/rabbitmq-signing-key-public.asc
_: apt-key add rabbitmq-signing-key-public.asc
\end{bashcode}

Actualizamos el sistema con los nuevos cambios:

\begin{bashcode}
_: apt-get update
\end{bashcode}

E instalamos el paquete de la manera habitual:

\begin{bashcode}
_: apt-get install rabbitmq-server
\end{bashcode}

La instalación incluirá automáticamente los paquetes de Erlang necesarios para ejecutar RabbitMQ.


%%%%%%%%%%%%%%%%%%%%%%%%%%%%%%%%%%%%%%%%%%%%%%%%%%%%%%%%%%%%%%%%%%%%%%%%%%%%%%%%
\section{Instalación de los \emph{plugins} para el soporte de AMQP y STOMP}

RabbitMQ no basta para proporcionar el servicio de paso de mensajes que MCollective requiere. Para que sea capaz de proporcionarlo necesitamos instalar unos \emph{plugins}. Para ello, vamos al fichero \texttt{/usr/lib/rabbitmq/lib/rabbitmq\_server-2.7.1/plugins} y comprobamos si existen \texttt{amqp\_client-2.7.1.ez} y \texttt{rabbitmq\_stomp-2.7.1.ez}. Si no existen, los instalamos:

\begin{bashcode}
_: wget -q http://www.rabbitmq.com/releases/plugins/v2.7.1/\
amqp_client-2.7.1.ez
_: wget -q http://www.rabbitmq.com/releases/plugins/v2.7.1/\
rabbitmq_stomp-2.7.1.ez
\end{bashcode}

Nota: Los nombres de los \emph{plugins} pueden variar dependiendo de la versión instalada.

Activamos los \emph{plugins}:

\begin{bashcode}
_: rabbitmq-plugins list
_: rabbitmq-plugins enable amqp_client           # El nombre puede cambiar
_: rabbitmq-plugins enable rabbitmq_stomp        # El nombre puede cambiar
\end{bashcode}

y reiniciamos el servidor:

\begin{bashcode}
_: /usr/sbin/service rabbitmq-server restart
\end{bashcode}


%%%%%%%%%%%%%%%%%%%%%%%%%%%%%%%%%%%%%%%%%%%%%%%%%%%%%%%%%%%%%%%%%%%%%%%%%%%%%%%%
\section{Configuración de la cuenta de MCollective}

Añadimos el usuario \texttt{mcollective} y la contraseña \texttt{mcollective}:
\begin{bashcode}
_: rabbitmqctl add_user mcollective mcollective
\end{bashcode}

Establecemos los permisos necesarios:

\begin{bashcode}
_: rabbitmqctl set_permissions -p / mcollective "^amq.gen-.*" ".*" ".*"
\end{bashcode}

Y por seguridad borramos la cuenta de invitado:

\begin{bashcode}
_: rabbitmqctl delete_user guest
\end{bashcode}


%%%%%%%%%%%%%%%%%%%%%%%%%%%%%%%%%%%%%%%%%%%%%%%%%%%%%%%%%%%%%%%%%%%%%%%%%%%%%%%%
\section{Comprobación de la instalación}

Para comprobar que hemos instalado correctamente RabbitMQ podemos hacer lo siguiente:

\begin{bashcode}
_: invoke-rc.d rabbitmq-server start
Starting rabbitmq-server: SUCCESS
rabbitmq-server.
\end{bashcode}


%%%%%%%%%%%%%%%%%%%%%%%%%%%%%%%%%%%%%%%%%%%%%%%%%%%%%%%%%%%%%%%%%%%%%%%%%%%%%%%%
\section{Versiones instaladas}

\begin{table}[!htbp]
\centering
   \begin{tabular}{|c|c|}
      \hline
      \textbf{Software} & \textbf{Versión} \\ \hline
      Ubuntu & 10.04 \\ \hline
      RabbitMQ & 2.7.1 \\ \hline
      Erlang & R13B03 (erts-5.7.4) \\ \hline
   \end{tabular}
\caption{Versiones instaladas de RabbitMQ y Erlang.}
\label{table:rabbitmq-versions}
\end{table}
