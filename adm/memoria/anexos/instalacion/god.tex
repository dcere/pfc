\chapter{God}
\label{anx:god}


En ocasiones, debido a fallos de Puppet o porque es más sencillo para la monitorización de procesos simples, se ha usado la herramienta de monitorización God.


%%%%%%%%%%%%%%%%%%%%%%%%%%%%%%%%%%%%%%%%%%%%%%%%%%%%%%%%%%%%%%%%%%%%%%%%%%%%%%%%
\section{Instalación}

El primer paso consiste en la instalación de RubyGems. Para ello, consulta el anexo \ref{anx:inst-ruby} de instalación de Ruby y RubyGems. Cuando ese paso ya esté completado instalaremos los paquetes \texttt{ruby1.8-dev} y \texttt{libopenssl-ruby}:

\begin{bashcode}
_: apt-get install ruby1.8-dev
_: apt-get install libopenssl-ruby
\end{bashcode}

Cuando estos paquetes ya se hayan instalado, procedemos a instalar God:

\begin{bashcode}
_: gem install god
\end{bashcode}


%%%%%%%%%%%%%%%%%%%%%%%%%%%%%%%%%%%%%%%%%%%%%%%%%%%%%%%%%%%%%%%%%%%%%%%%%%%%%%%%
\section{Comprobación de la instalación}

Puedes verificar que God se ha instalado satisfactoriamente comprobando la versión instalada:

\begin{bashcode}
_: god --version
Version 0.12.1
\end{bashcode}


%%%%%%%%%%%%%%%%%%%%%%%%%%%%%%%%%%%%%%%%%%%%%%%%%%%%%%%%%%%%%%%%%%%%%%%%%%%%%%%%
\section{Versiones instaladas}

\begin{table}[!htbp]
\centering
   \begin{tabular}{|c|c|}
      \hline
      \textbf{Software} & \textbf{Versión} \\ \hline
      Ubuntu & 10.04 \\ \hline
      Ruby & 1.8.7 \\ \hline
      RubyGems & 1.8.10 (o superior) \\ \hline
      god & 0.12.1 \\ \hline
   \end{tabular}
\caption{Versión instalada de god.}
\label{table:god-versions}
\end{table}
