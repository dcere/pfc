\chapter{Libvirt}
\label{comun:libvirt}

%%%%%%%%%%%%%%%%%%%%%%%%%%%%%%%%%%%%%%%%%%%%%%%%%%%%%%%%%%%%%%%%%%%%%%%%%%%%%%%%
\section{Instalación}

Instalaremos \texttt{libvirt-ruby} mediante apt-get:

\begin{bashcode}
_: apt-get install libvirt-ruby
\end{bashcode}


%%%%%%%%%%%%%%%%%%%%%%%%%%%%%%%%%%%%%%%%%%%%%%%%%%%%%%%%%%%%%%%%%%%%%%%%%%%%%%%%
\section{Versiones instaladas}

\begin{table}[!htbp]
\centering
   \begin{tabular}{|c|c|}
      \hline
      \textbf{Software} & \textbf{Versión} \\ \hline
      Ubuntu & 10.04 \\ \hline
      Ruby & 1.8.7 \\ \hline
      libvirt-ruby & 0.0.7-1 \\ \hline
   \end{tabular}
\caption{Versión instalada de libvirt.}
\label{table:libvirt-versions}
\end{table}
