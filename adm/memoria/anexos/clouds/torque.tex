\chapter{TORQUE}
\label{anx:torque}


%%%%%%%%%%%%%%%%%%%%%%%%%%%%%%%%%%%%%%%%%%%%%%%%%%%%%%%%%%%%%%%%%%%%%%%%%%%%%%%%
\section{Instalación de TORQUE}


%%%%%%%%%%%%%%%%%%%%%%%%%%%%%%%%%%%%%%%%%%%%%%%%%%%%%%%%%%%%%%%%%%%%%%%%%%%%%%%%
\subsection{Nodo maestro}

Primero instalaremos \texttt{libssl-dev}:

\begin{bashcode}
_: apt-get install libssl-dev
\end{bashcode}

Una vez hecho esto debemos seguir las instrucciones del enlace \url{http://www.adaptivecomputing.com/resources/docs/torque/4-0-1/help.htm#topics/1-installConfig/installing.htm}, que básicamente son:

\begin{bashcode}
_: tar -xzvf torque-4.0.3.tar.gz
_: cd torque-4.0.3/
_: echo '/usr/local/lib' > /etc/ld.so.conf.d/torque.conf
_: ldconfig
_: ./configure
_: make
_: make install
\end{bashcode}


%%%%%%%%%%%%%%%%%%%%%%%%%%%%%%%%%%%%%%%%%%%%%%%%%%%%%%%%%%%%%%%%%%%%%%%%%%%%%%%%
\subsection{Nodos de computación}

Para crear los paquetes necesarios para los nodos de computación, haz lo siguiente:

\begin{bashcode}
_: cd torque-4.0.3/
_: make packages
\end{bashcode}

Ahora se pueden instalar estos paquetes en los nodos de computación usando los \emph{scripts} creados dentro del directorio \texttt{torque-4.0.3}:

\begin{bashcode}
_: ./torque-package-mom-linux-x86_64.sh --install
\end{bashcode}

No hay que olvidarse de incluir \texttt{/usr/local/lib} en la lista de directorios para buscar las librerías enlazadas dinámicamente:

\begin{bashcode}
_: echo '/usr/local/lib' > /etc/ld.so.conf.d/torque.conf
_: ldconfig
\end{bashcode}

Para habilitar TORQUE como un servicio se debe hacer lo siguiente:

\begin{bashcode}
_: cd torque-4.0.3
_: cp contrib/init.d/debian.pbs_mom /etc/init.d/pbs_mom
_: update-rc.d pbs_mom defaults
\end{bashcode}


%%%%%%%%%%%%%%%%%%%%%%%%%%%%%%%%%%%%%%%%%%%%%%%%%%%%%%%%%%%%%%%%%%%%%%%%%%%%%%%%
\section{Configuración de TORQUE}


%%%%%%%%%%%%%%%%%%%%%%%%%%%%%%%%%%%%%%%%%%%%%%%%%%%%%%%%%%%%%%%%%%%%%%%%%%%%%%%%
\subsection{Nodo maestro}
\label{anx:torque-conf-maestro}

Añade los nodos de computación en el fichero \texttt{/var/spool/torque/server\_priv/nodes}. Por ejemplo:

\begin{lstlisting}
## This is the TORQUE server "nodes" file. 
## 
## To add a node, enter its hostname, optional processor count (np=), 
## and optional feature names.
## 
## Example:
##    host01 np=8 featureA featureB 
##    host02 np=8 featureA featureB
## 
## for more information, please visit:
## 
## http://www.clusterresources.com/torquedocs/nodeconfig.shtml

### Nodes
lucid-tor2
\end{lstlisting}

Posteriormente copia el fichero \texttt{debian.trqauthd} al directorio \texttt{/etc/init.d}:

\begin{bashcode}
_: cd torque-4.0.3/contrib/init.d
_: cp debian.trqauthd /etc/init.d
_: cd /etc/init.d
_: mv debian.trqauthd trqauthd
\end{bashcode}

Puedes empezar el demonio con:

\begin{bashcode}
_: /etc/init.d/trqauthd start
_: /usr/bin/service trqauthd start
\end{bashcode}

y puedes pararlo con:

\begin{bashcode}
_: /etc/init.d/trqauthd stop
_: /usr/bin/service trqauthd stop
\end{bashcode}

Inicializa \texttt{serverdb}:

\begin{bashcode}
_: cd torque-4.0.3/
_: ./torque.setup <user>   # Asegurate de que el usuario <user> existe en cada
                           # nodo, maestro o de computacion.
\end{bashcode}


%%%%%%%%%%%%%%%%%%%%%%%%%%%%%%%%%%%%%%%%%%%%%%%%%%%%%%%%%%%%%%%%%%%%%%%%%%%%%%%%
\subsection{Comprobación de la configuración}


Reiniciamos \texttt{pbs\_server} en el nodo maestro:

\begin{bashcode}
_: qterm -t quick
_: pbs_server
\end{bashcode}

Y \texttt{pbs\_mom} en los nodos de computación:

\begin{bashcode}
_: pbs_mom
\end{bashcode}

Después de esperar durante un tiempo prudencial, el comando \texttt{pbsnodes -a} proporcionará una lista de nodos libres:

\begin{bashcode}
_: pbsnodes -a
lucid-tor2
     state = free
     np = 1
     ntype = cluster
     status = rectime=1338548162,varattr=,jobs=,state=free,netload=2359153,
              gres=,loadave=0.03,ncpus=2,physmem=1023292kb,availmem=1457904kb,
              totmem=1521972kb,idletime=3003,nusers=0,nsessions=0,
              uname=Linux lucid-tor2 2.6.32-33-server
              #70-Ubuntu SMP Thu Jul 7 22:28:30 UTC 2011 x86_64,opsys=linux
     mom_service_port = 15002
     mom_manager_port = 15003
     gpus = 0
\end{bashcode}

Nota: Si el nodo aparece como \texttt{offline}, y únicamente \texttt{offline} (i.e. no como \texttt{down, offline}) se puede liberar con:

\begin{bashcode}
_: pbsnodes -c lucid-tor2
\end{bashcode}

Para ver el estado de las colas podemos ejecutar \texttt{qstat -q}:

\begin{bashcode}
_: qstat -q

server: lucid-tor1

Queue            Memory CPU Time Walltime Node  Run Que Lm  State
---------------- ------ -------- -------- ----  --- --- --  -----
batch              --      --       --      --    0   0 --   E R
                                               ----- -----
                                                   0     0
\end{bashcode}


%%%%%%%%%%%%%%%%%%%%%%%%%%%%%%%%%%%%%%%%%%%%%%%%%%%%%%%%%%%%%%%%%%%%%%%%%%%%%%%%
\subsection{Envío de un trabajo}

Para comprobar que TORQUE es capaz de ejecutar trabajos podemos enviarle un trabajo sencillo. Para ello ejecutaremos el comando \texttt{echo "sleep 30" | qsub} desde el usuario creado al final de la sección \ref{anx:torque-conf-maestro}. Si el trabajo fue correctamente recibido deberá aparecer al usar el comando \texttt{qstat}:

\begin{bashcode}
root@lucid-tor1:~# su - david
david@lucid-tor1:~$ echo "sleep 30" | qsub
0.lucid-tor1
david@lucid-tor1:~$ qstat
Job id                    Name             User            Time Use S Queue
------------------------- ---------------- --------------- -------- - -----
0.lucid-tor1              STDIN            david                  0 Q batch
\end{bashcode}


%%%%%%%%%%%%%%%%%%%%%%%%%%%%%%%%%%%%%%%%%%%%%%%%%%%%%%%%%%%%%%%%%%%%%%%%%%%%%%%%
\section{Versiones instaladas}

\begin{table}[!htbp]
\centering
   \begin{tabular}{|c|c|}
      \hline
      \textbf{Software} & \textbf{Versión} \\ \hline
      Ubuntu & 10.04 \\ \hline
      TORQUE & 4.0.3 \\ \hline
   \end{tabular}
\caption{Versión instalada de TORQUE.}
\label{table:torque-versions}
\end{table}
